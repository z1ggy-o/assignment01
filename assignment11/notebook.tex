
% Default to the notebook output style

    


% Inherit from the specified cell style.




    
\documentclass[11pt]{article}

    
    
    \usepackage[T1]{fontenc}
    % Nicer default font (+ math font) than Computer Modern for most use cases
    \usepackage{mathpazo}

    % Basic figure setup, for now with no caption control since it's done
    % automatically by Pandoc (which extracts ![](path) syntax from Markdown).
    \usepackage{graphicx}
    % We will generate all images so they have a width \maxwidth. This means
    % that they will get their normal width if they fit onto the page, but
    % are scaled down if they would overflow the margins.
    \makeatletter
    \def\maxwidth{\ifdim\Gin@nat@width>\linewidth\linewidth
    \else\Gin@nat@width\fi}
    \makeatother
    \let\Oldincludegraphics\includegraphics
    % Set max figure width to be 80% of text width, for now hardcoded.
    \renewcommand{\includegraphics}[1]{\Oldincludegraphics[width=.8\maxwidth]{#1}}
    % Ensure that by default, figures have no caption (until we provide a
    % proper Figure object with a Caption API and a way to capture that
    % in the conversion process - todo).
    \usepackage{caption}
    \DeclareCaptionLabelFormat{nolabel}{}
    \captionsetup{labelformat=nolabel}

    \usepackage{adjustbox} % Used to constrain images to a maximum size 
    \usepackage{xcolor} % Allow colors to be defined
    \usepackage{enumerate} % Needed for markdown enumerations to work
    \usepackage{geometry} % Used to adjust the document margins
    \usepackage{amsmath} % Equations
    \usepackage{amssymb} % Equations
    \usepackage{textcomp} % defines textquotesingle
    % Hack from http://tex.stackexchange.com/a/47451/13684:
    \AtBeginDocument{%
        \def\PYZsq{\textquotesingle}% Upright quotes in Pygmentized code
    }
    \usepackage{upquote} % Upright quotes for verbatim code
    \usepackage{eurosym} % defines \euro
    \usepackage[mathletters]{ucs} % Extended unicode (utf-8) support
    \usepackage[utf8x]{inputenc} % Allow utf-8 characters in the tex document
    \usepackage{fancyvrb} % verbatim replacement that allows latex
    \usepackage{grffile} % extends the file name processing of package graphics 
                         % to support a larger range 
    % The hyperref package gives us a pdf with properly built
    % internal navigation ('pdf bookmarks' for the table of contents,
    % internal cross-reference links, web links for URLs, etc.)
    \usepackage{hyperref}
    \usepackage{longtable} % longtable support required by pandoc >1.10
    \usepackage{booktabs}  % table support for pandoc > 1.12.2
    \usepackage[inline]{enumitem} % IRkernel/repr support (it uses the enumerate* environment)
    \usepackage[normalem]{ulem} % ulem is needed to support strikethroughs (\sout)
                                % normalem makes italics be italics, not underlines
    

    
    
    % Colors for the hyperref package
    \definecolor{urlcolor}{rgb}{0,.145,.698}
    \definecolor{linkcolor}{rgb}{.71,0.21,0.01}
    \definecolor{citecolor}{rgb}{.12,.54,.11}

    % ANSI colors
    \definecolor{ansi-black}{HTML}{3E424D}
    \definecolor{ansi-black-intense}{HTML}{282C36}
    \definecolor{ansi-red}{HTML}{E75C58}
    \definecolor{ansi-red-intense}{HTML}{B22B31}
    \definecolor{ansi-green}{HTML}{00A250}
    \definecolor{ansi-green-intense}{HTML}{007427}
    \definecolor{ansi-yellow}{HTML}{DDB62B}
    \definecolor{ansi-yellow-intense}{HTML}{B27D12}
    \definecolor{ansi-blue}{HTML}{208FFB}
    \definecolor{ansi-blue-intense}{HTML}{0065CA}
    \definecolor{ansi-magenta}{HTML}{D160C4}
    \definecolor{ansi-magenta-intense}{HTML}{A03196}
    \definecolor{ansi-cyan}{HTML}{60C6C8}
    \definecolor{ansi-cyan-intense}{HTML}{258F8F}
    \definecolor{ansi-white}{HTML}{C5C1B4}
    \definecolor{ansi-white-intense}{HTML}{A1A6B2}

    % commands and environments needed by pandoc snippets
    % extracted from the output of `pandoc -s`
    \providecommand{\tightlist}{%
      \setlength{\itemsep}{0pt}\setlength{\parskip}{0pt}}
    \DefineVerbatimEnvironment{Highlighting}{Verbatim}{commandchars=\\\{\}}
    % Add ',fontsize=\small' for more characters per line
    \newenvironment{Shaded}{}{}
    \newcommand{\KeywordTok}[1]{\textcolor[rgb]{0.00,0.44,0.13}{\textbf{{#1}}}}
    \newcommand{\DataTypeTok}[1]{\textcolor[rgb]{0.56,0.13,0.00}{{#1}}}
    \newcommand{\DecValTok}[1]{\textcolor[rgb]{0.25,0.63,0.44}{{#1}}}
    \newcommand{\BaseNTok}[1]{\textcolor[rgb]{0.25,0.63,0.44}{{#1}}}
    \newcommand{\FloatTok}[1]{\textcolor[rgb]{0.25,0.63,0.44}{{#1}}}
    \newcommand{\CharTok}[1]{\textcolor[rgb]{0.25,0.44,0.63}{{#1}}}
    \newcommand{\StringTok}[1]{\textcolor[rgb]{0.25,0.44,0.63}{{#1}}}
    \newcommand{\CommentTok}[1]{\textcolor[rgb]{0.38,0.63,0.69}{\textit{{#1}}}}
    \newcommand{\OtherTok}[1]{\textcolor[rgb]{0.00,0.44,0.13}{{#1}}}
    \newcommand{\AlertTok}[1]{\textcolor[rgb]{1.00,0.00,0.00}{\textbf{{#1}}}}
    \newcommand{\FunctionTok}[1]{\textcolor[rgb]{0.02,0.16,0.49}{{#1}}}
    \newcommand{\RegionMarkerTok}[1]{{#1}}
    \newcommand{\ErrorTok}[1]{\textcolor[rgb]{1.00,0.00,0.00}{\textbf{{#1}}}}
    \newcommand{\NormalTok}[1]{{#1}}
    
    % Additional commands for more recent versions of Pandoc
    \newcommand{\ConstantTok}[1]{\textcolor[rgb]{0.53,0.00,0.00}{{#1}}}
    \newcommand{\SpecialCharTok}[1]{\textcolor[rgb]{0.25,0.44,0.63}{{#1}}}
    \newcommand{\VerbatimStringTok}[1]{\textcolor[rgb]{0.25,0.44,0.63}{{#1}}}
    \newcommand{\SpecialStringTok}[1]{\textcolor[rgb]{0.73,0.40,0.53}{{#1}}}
    \newcommand{\ImportTok}[1]{{#1}}
    \newcommand{\DocumentationTok}[1]{\textcolor[rgb]{0.73,0.13,0.13}{\textit{{#1}}}}
    \newcommand{\AnnotationTok}[1]{\textcolor[rgb]{0.38,0.63,0.69}{\textbf{\textit{{#1}}}}}
    \newcommand{\CommentVarTok}[1]{\textcolor[rgb]{0.38,0.63,0.69}{\textbf{\textit{{#1}}}}}
    \newcommand{\VariableTok}[1]{\textcolor[rgb]{0.10,0.09,0.49}{{#1}}}
    \newcommand{\ControlFlowTok}[1]{\textcolor[rgb]{0.00,0.44,0.13}{\textbf{{#1}}}}
    \newcommand{\OperatorTok}[1]{\textcolor[rgb]{0.40,0.40,0.40}{{#1}}}
    \newcommand{\BuiltInTok}[1]{{#1}}
    \newcommand{\ExtensionTok}[1]{{#1}}
    \newcommand{\PreprocessorTok}[1]{\textcolor[rgb]{0.74,0.48,0.00}{{#1}}}
    \newcommand{\AttributeTok}[1]{\textcolor[rgb]{0.49,0.56,0.16}{{#1}}}
    \newcommand{\InformationTok}[1]{\textcolor[rgb]{0.38,0.63,0.69}{\textbf{\textit{{#1}}}}}
    \newcommand{\WarningTok}[1]{\textcolor[rgb]{0.38,0.63,0.69}{\textbf{\textit{{#1}}}}}
    
    
    % Define a nice break command that doesn't care if a line doesn't already
    % exist.
    \def\br{\hspace*{\fill} \\* }
    % Math Jax compatability definitions
    \def\gt{>}
    \def\lt{<}
    % Document parameters
    \title{assignment11}
    
    
    

    % Pygments definitions
    
\makeatletter
\def\PY@reset{\let\PY@it=\relax \let\PY@bf=\relax%
    \let\PY@ul=\relax \let\PY@tc=\relax%
    \let\PY@bc=\relax \let\PY@ff=\relax}
\def\PY@tok#1{\csname PY@tok@#1\endcsname}
\def\PY@toks#1+{\ifx\relax#1\empty\else%
    \PY@tok{#1}\expandafter\PY@toks\fi}
\def\PY@do#1{\PY@bc{\PY@tc{\PY@ul{%
    \PY@it{\PY@bf{\PY@ff{#1}}}}}}}
\def\PY#1#2{\PY@reset\PY@toks#1+\relax+\PY@do{#2}}

\expandafter\def\csname PY@tok@w\endcsname{\def\PY@tc##1{\textcolor[rgb]{0.73,0.73,0.73}{##1}}}
\expandafter\def\csname PY@tok@c\endcsname{\let\PY@it=\textit\def\PY@tc##1{\textcolor[rgb]{0.25,0.50,0.50}{##1}}}
\expandafter\def\csname PY@tok@cp\endcsname{\def\PY@tc##1{\textcolor[rgb]{0.74,0.48,0.00}{##1}}}
\expandafter\def\csname PY@tok@k\endcsname{\let\PY@bf=\textbf\def\PY@tc##1{\textcolor[rgb]{0.00,0.50,0.00}{##1}}}
\expandafter\def\csname PY@tok@kp\endcsname{\def\PY@tc##1{\textcolor[rgb]{0.00,0.50,0.00}{##1}}}
\expandafter\def\csname PY@tok@kt\endcsname{\def\PY@tc##1{\textcolor[rgb]{0.69,0.00,0.25}{##1}}}
\expandafter\def\csname PY@tok@o\endcsname{\def\PY@tc##1{\textcolor[rgb]{0.40,0.40,0.40}{##1}}}
\expandafter\def\csname PY@tok@ow\endcsname{\let\PY@bf=\textbf\def\PY@tc##1{\textcolor[rgb]{0.67,0.13,1.00}{##1}}}
\expandafter\def\csname PY@tok@nb\endcsname{\def\PY@tc##1{\textcolor[rgb]{0.00,0.50,0.00}{##1}}}
\expandafter\def\csname PY@tok@nf\endcsname{\def\PY@tc##1{\textcolor[rgb]{0.00,0.00,1.00}{##1}}}
\expandafter\def\csname PY@tok@nc\endcsname{\let\PY@bf=\textbf\def\PY@tc##1{\textcolor[rgb]{0.00,0.00,1.00}{##1}}}
\expandafter\def\csname PY@tok@nn\endcsname{\let\PY@bf=\textbf\def\PY@tc##1{\textcolor[rgb]{0.00,0.00,1.00}{##1}}}
\expandafter\def\csname PY@tok@ne\endcsname{\let\PY@bf=\textbf\def\PY@tc##1{\textcolor[rgb]{0.82,0.25,0.23}{##1}}}
\expandafter\def\csname PY@tok@nv\endcsname{\def\PY@tc##1{\textcolor[rgb]{0.10,0.09,0.49}{##1}}}
\expandafter\def\csname PY@tok@no\endcsname{\def\PY@tc##1{\textcolor[rgb]{0.53,0.00,0.00}{##1}}}
\expandafter\def\csname PY@tok@nl\endcsname{\def\PY@tc##1{\textcolor[rgb]{0.63,0.63,0.00}{##1}}}
\expandafter\def\csname PY@tok@ni\endcsname{\let\PY@bf=\textbf\def\PY@tc##1{\textcolor[rgb]{0.60,0.60,0.60}{##1}}}
\expandafter\def\csname PY@tok@na\endcsname{\def\PY@tc##1{\textcolor[rgb]{0.49,0.56,0.16}{##1}}}
\expandafter\def\csname PY@tok@nt\endcsname{\let\PY@bf=\textbf\def\PY@tc##1{\textcolor[rgb]{0.00,0.50,0.00}{##1}}}
\expandafter\def\csname PY@tok@nd\endcsname{\def\PY@tc##1{\textcolor[rgb]{0.67,0.13,1.00}{##1}}}
\expandafter\def\csname PY@tok@s\endcsname{\def\PY@tc##1{\textcolor[rgb]{0.73,0.13,0.13}{##1}}}
\expandafter\def\csname PY@tok@sd\endcsname{\let\PY@it=\textit\def\PY@tc##1{\textcolor[rgb]{0.73,0.13,0.13}{##1}}}
\expandafter\def\csname PY@tok@si\endcsname{\let\PY@bf=\textbf\def\PY@tc##1{\textcolor[rgb]{0.73,0.40,0.53}{##1}}}
\expandafter\def\csname PY@tok@se\endcsname{\let\PY@bf=\textbf\def\PY@tc##1{\textcolor[rgb]{0.73,0.40,0.13}{##1}}}
\expandafter\def\csname PY@tok@sr\endcsname{\def\PY@tc##1{\textcolor[rgb]{0.73,0.40,0.53}{##1}}}
\expandafter\def\csname PY@tok@ss\endcsname{\def\PY@tc##1{\textcolor[rgb]{0.10,0.09,0.49}{##1}}}
\expandafter\def\csname PY@tok@sx\endcsname{\def\PY@tc##1{\textcolor[rgb]{0.00,0.50,0.00}{##1}}}
\expandafter\def\csname PY@tok@m\endcsname{\def\PY@tc##1{\textcolor[rgb]{0.40,0.40,0.40}{##1}}}
\expandafter\def\csname PY@tok@gh\endcsname{\let\PY@bf=\textbf\def\PY@tc##1{\textcolor[rgb]{0.00,0.00,0.50}{##1}}}
\expandafter\def\csname PY@tok@gu\endcsname{\let\PY@bf=\textbf\def\PY@tc##1{\textcolor[rgb]{0.50,0.00,0.50}{##1}}}
\expandafter\def\csname PY@tok@gd\endcsname{\def\PY@tc##1{\textcolor[rgb]{0.63,0.00,0.00}{##1}}}
\expandafter\def\csname PY@tok@gi\endcsname{\def\PY@tc##1{\textcolor[rgb]{0.00,0.63,0.00}{##1}}}
\expandafter\def\csname PY@tok@gr\endcsname{\def\PY@tc##1{\textcolor[rgb]{1.00,0.00,0.00}{##1}}}
\expandafter\def\csname PY@tok@ge\endcsname{\let\PY@it=\textit}
\expandafter\def\csname PY@tok@gs\endcsname{\let\PY@bf=\textbf}
\expandafter\def\csname PY@tok@gp\endcsname{\let\PY@bf=\textbf\def\PY@tc##1{\textcolor[rgb]{0.00,0.00,0.50}{##1}}}
\expandafter\def\csname PY@tok@go\endcsname{\def\PY@tc##1{\textcolor[rgb]{0.53,0.53,0.53}{##1}}}
\expandafter\def\csname PY@tok@gt\endcsname{\def\PY@tc##1{\textcolor[rgb]{0.00,0.27,0.87}{##1}}}
\expandafter\def\csname PY@tok@err\endcsname{\def\PY@bc##1{\setlength{\fboxsep}{0pt}\fcolorbox[rgb]{1.00,0.00,0.00}{1,1,1}{\strut ##1}}}
\expandafter\def\csname PY@tok@kc\endcsname{\let\PY@bf=\textbf\def\PY@tc##1{\textcolor[rgb]{0.00,0.50,0.00}{##1}}}
\expandafter\def\csname PY@tok@kd\endcsname{\let\PY@bf=\textbf\def\PY@tc##1{\textcolor[rgb]{0.00,0.50,0.00}{##1}}}
\expandafter\def\csname PY@tok@kn\endcsname{\let\PY@bf=\textbf\def\PY@tc##1{\textcolor[rgb]{0.00,0.50,0.00}{##1}}}
\expandafter\def\csname PY@tok@kr\endcsname{\let\PY@bf=\textbf\def\PY@tc##1{\textcolor[rgb]{0.00,0.50,0.00}{##1}}}
\expandafter\def\csname PY@tok@bp\endcsname{\def\PY@tc##1{\textcolor[rgb]{0.00,0.50,0.00}{##1}}}
\expandafter\def\csname PY@tok@fm\endcsname{\def\PY@tc##1{\textcolor[rgb]{0.00,0.00,1.00}{##1}}}
\expandafter\def\csname PY@tok@vc\endcsname{\def\PY@tc##1{\textcolor[rgb]{0.10,0.09,0.49}{##1}}}
\expandafter\def\csname PY@tok@vg\endcsname{\def\PY@tc##1{\textcolor[rgb]{0.10,0.09,0.49}{##1}}}
\expandafter\def\csname PY@tok@vi\endcsname{\def\PY@tc##1{\textcolor[rgb]{0.10,0.09,0.49}{##1}}}
\expandafter\def\csname PY@tok@vm\endcsname{\def\PY@tc##1{\textcolor[rgb]{0.10,0.09,0.49}{##1}}}
\expandafter\def\csname PY@tok@sa\endcsname{\def\PY@tc##1{\textcolor[rgb]{0.73,0.13,0.13}{##1}}}
\expandafter\def\csname PY@tok@sb\endcsname{\def\PY@tc##1{\textcolor[rgb]{0.73,0.13,0.13}{##1}}}
\expandafter\def\csname PY@tok@sc\endcsname{\def\PY@tc##1{\textcolor[rgb]{0.73,0.13,0.13}{##1}}}
\expandafter\def\csname PY@tok@dl\endcsname{\def\PY@tc##1{\textcolor[rgb]{0.73,0.13,0.13}{##1}}}
\expandafter\def\csname PY@tok@s2\endcsname{\def\PY@tc##1{\textcolor[rgb]{0.73,0.13,0.13}{##1}}}
\expandafter\def\csname PY@tok@sh\endcsname{\def\PY@tc##1{\textcolor[rgb]{0.73,0.13,0.13}{##1}}}
\expandafter\def\csname PY@tok@s1\endcsname{\def\PY@tc##1{\textcolor[rgb]{0.73,0.13,0.13}{##1}}}
\expandafter\def\csname PY@tok@mb\endcsname{\def\PY@tc##1{\textcolor[rgb]{0.40,0.40,0.40}{##1}}}
\expandafter\def\csname PY@tok@mf\endcsname{\def\PY@tc##1{\textcolor[rgb]{0.40,0.40,0.40}{##1}}}
\expandafter\def\csname PY@tok@mh\endcsname{\def\PY@tc##1{\textcolor[rgb]{0.40,0.40,0.40}{##1}}}
\expandafter\def\csname PY@tok@mi\endcsname{\def\PY@tc##1{\textcolor[rgb]{0.40,0.40,0.40}{##1}}}
\expandafter\def\csname PY@tok@il\endcsname{\def\PY@tc##1{\textcolor[rgb]{0.40,0.40,0.40}{##1}}}
\expandafter\def\csname PY@tok@mo\endcsname{\def\PY@tc##1{\textcolor[rgb]{0.40,0.40,0.40}{##1}}}
\expandafter\def\csname PY@tok@ch\endcsname{\let\PY@it=\textit\def\PY@tc##1{\textcolor[rgb]{0.25,0.50,0.50}{##1}}}
\expandafter\def\csname PY@tok@cm\endcsname{\let\PY@it=\textit\def\PY@tc##1{\textcolor[rgb]{0.25,0.50,0.50}{##1}}}
\expandafter\def\csname PY@tok@cpf\endcsname{\let\PY@it=\textit\def\PY@tc##1{\textcolor[rgb]{0.25,0.50,0.50}{##1}}}
\expandafter\def\csname PY@tok@c1\endcsname{\let\PY@it=\textit\def\PY@tc##1{\textcolor[rgb]{0.25,0.50,0.50}{##1}}}
\expandafter\def\csname PY@tok@cs\endcsname{\let\PY@it=\textit\def\PY@tc##1{\textcolor[rgb]{0.25,0.50,0.50}{##1}}}

\def\PYZbs{\char`\\}
\def\PYZus{\char`\_}
\def\PYZob{\char`\{}
\def\PYZcb{\char`\}}
\def\PYZca{\char`\^}
\def\PYZam{\char`\&}
\def\PYZlt{\char`\<}
\def\PYZgt{\char`\>}
\def\PYZsh{\char`\#}
\def\PYZpc{\char`\%}
\def\PYZdl{\char`\$}
\def\PYZhy{\char`\-}
\def\PYZsq{\char`\'}
\def\PYZdq{\char`\"}
\def\PYZti{\char`\~}
% for compatibility with earlier versions
\def\PYZat{@}
\def\PYZlb{[}
\def\PYZrb{]}
\makeatother


    % Exact colors from NB
    \definecolor{incolor}{rgb}{0.0, 0.0, 0.5}
    \definecolor{outcolor}{rgb}{0.545, 0.0, 0.0}



    
    % Prevent overflowing lines due to hard-to-break entities
    \sloppy 
    % Setup hyperref package
    \hypersetup{
      breaklinks=true,  % so long urls are correctly broken across lines
      colorlinks=true,
      urlcolor=urlcolor,
      linkcolor=linkcolor,
      citecolor=citecolor,
      }
    % Slightly bigger margins than the latex defaults
    
    \geometry{verbose,tmargin=1in,bmargin=1in,lmargin=1in,rmargin=1in}
    
    

    \begin{document}
    
    
    \maketitle
    
    

    
    \section{Image Denoising}\label{image-denoising}

\textbf{Name}: ZHU GUANGYU\\
\textbf{Student ID}: 20165953\\
\textbf{Github Repo}:
\href{https://github.com/z1ggy-o/cv_assignment/tree/master/assignment11}{assignment11}

\begin{center}\rule{0.5\linewidth}{\linethickness}\end{center}

\subsection{Multi-objective least
squares}\label{multi-objective-least-squares}

In some applications we have \emph{multiple} objectives, all of which we
would like to be small:

\[J_{1} = ||A_{1}x-b_{1}||^{2}, \cdots, J_{k} = ||A_{k}x-b_{k}||^{2} \]

We seek a \emph{single} \(\hat{x}\) that gives a compromise, and makes
them all small, tothe extent possible. We call this the
\emph{multi-objective} least squares problem.

We cannot get a single \(\hat{x}\) which makes all the objectives be
minimum at same time, so we have to come out a compromise plan. A
standard method for finding a value of \(x\) that gives a compromise in
making all the objectves small is to choose \(x\) to minimize a
\emph{weighted sum objective}:

\[ J = \lambda_{1}J_{1} + \cdots + \lambda_{k}J_{k} = \lambda_{1}||A_{1}x-b_{1}||^{2} + \cdots + \lambda_{k}||A_{k}x-b_{k}||^{2}, \]

where \(\lambda\) are positive \emph{weights}, that express our relative
desire for the terms to be small.

\subsubsection{Weighted sum least squares via
stacking}\label{weighted-sum-least-squares-via-stacking}

We can minimize the weighted sum obejctive function by expressing it as
a standard least squares problem, then we can solve it by the method we
use before. Express \(J\) as the norm squared of a single vector:

\[ J = \begin{Vmatrix}
\begin{bmatrix}
\sqrt{\lambda_{1}}(A_{1}x - b_{1}) \\
\vdots \\
\sqrt{\lambda_{k}}(A_{k}x - b_{k})
\end{bmatrix}
\end{Vmatrix}^{2}\]

so we have

\[
J = \begin{Vmatrix}
\begin{bmatrix}\sqrt{\lambda_{1}}A_{1}\\ \vdots \\ \sqrt{\lambda_{k}}A_{k} \end{bmatrix}x -
\begin{bmatrix}\sqrt{\lambda_{1}}b_{1}\\ \vdots \\ \sqrt{\lambda_{k}}b_{k} \end{bmatrix}
\end{Vmatrix}^{2}
=
\begin{Vmatrix}\tilde{A}x - \tilde{b} \end{Vmatrix}^{2}
\]

Now, we have reduced the problem of minimizing the weighted sum least
squares objective to a standard lest squares problem.

Usually, we identify a \emph{primary ojective} \(J_{1}\) that we would
like to be small. We also identify one or more \emph{secondary
objectives} that we would also like to be small. There are many possible
sencondary objectives:

\begin{itemize}
\tightlist
\item
  \(||x||^{2}\): \(x\) should be small.
\item
  \(|| x - x^{prior}||^{2}\): \(x\) should be near \(x^{prior}\).
\item
  \(||Dx||^{2}\), where \(D\) is the first difference matrix: \(x\)
  should be smooth.
\end{itemize}

\subsection{Estination and inversion}\label{estination-and-inversion}

In the broad application area of \emph{estimation}, the goal is to
estimate a set of \(n\) values (also called parameters), the entries of
the n-vector \(x\). We are given a set of \emph{m measurements}, the
entries of an \(m\)-vector \(y\). They are related by

\[ y = Ax + v \]

The \(m\)-vector \(v\) is the \emph{measurement noise}, and is unknown
but presumed to be small. The estimation problem is to make a sensible
guess as to what \(x\) is, given \(y\) and prior knowledge about \(x\).

Of course we cannot expect to find \(x\) exactly when the measurement
noise is nonzero. This is called \emph{approximate inversion}.

If we guess that \(x\) has the value \(\hat{x}\), then we are implicitly
making the guess that \(v\) has the value \(y - A\hat{x}\). If we assume
that \(v\) is small, then a sensible choive for \(\hat{x}\) is the least
squares approximate solution, which minimizes \(||A\hat{x}-y||^{2}\). We
will take this as our primary objective. And we choose secondary
objectives by the informations we know about \(x\).

\subsection{Image Denoising}\label{image-denoising-1}

Image de-nosing is one of the \emph{Inversion} problem. The vector \(x\)
is an image, and the matrix \(A\) gives noise, so \$y = Ax + v \$ is a
noisy image.

Our prior infromation about \(x\) is that it is smooth; neighboring
pixels values are not very different from each other. So we choose
\(||Dx||^2\) be our secondary objective.

Because image is 2D, we form an estimate \(\hat{x}\) image by minimizing
a cost function of the form

\[||Ax - y||^2 + \lambda (||D_{h}x||^{2} + ||D_{v}x||^{2}) \]

Here \(D_{v}\) and \(D_{h}\) are vertical and horizontal differencing
operations.

\subsubsection{Our specific denosing
problem}\label{our-specific-denosing-problem}

In our problem, we do not use matrix \(A\) to add noisy on image, but
just given noisy image \(y\), so our function is

\[ ||x - y||^2 + \lambda (||D_{h}x||^{2} + ||D_{v}x||^{2}) \]

Express it as a norm squared form, we have

\[
\begin{Vmatrix}
\begin{bmatrix} I \\ \sqrt{\lambda}D_{h}\\ \sqrt{\lambda}D_{v} \end{bmatrix}x -
\begin{bmatrix} y \\ 0 \\ 0 \end{bmatrix}
\end{Vmatrix}^{2}
=
\begin{Vmatrix}\tilde{A}x - \tilde{b} \end{Vmatrix}^{2}
\]

Suppose the vector \(x\) has length \(MN\) then \(D_{h}\) is a
\(M(N-1) \times MN\) matrix, \(D_{v}\) be the \((M-1)N \times MN\)
matrix, and \(I\) is a \(M \times N\) identity matrix.

Now we can use standard least mean square method to get our \(\hat{x}\).

\begin{center}\rule{0.5\linewidth}{\linethickness}\end{center}

    \subsection{Implementation}\label{implementation}

\subsubsection{Import packages \& read input
data}\label{import-packages-read-input-data}

    \begin{Verbatim}[commandchars=\\\{\}]
{\color{incolor}In [{\color{incolor}1}]:} \PY{k+kn}{import} \PY{n+nn}{matplotlib}\PY{n+nn}{.}\PY{n+nn}{pyplot} \PY{k}{as} \PY{n+nn}{plt}
        \PY{k+kn}{import} \PY{n+nn}{numpy} \PY{k}{as} \PY{n+nn}{np}
        \PY{k+kn}{import} \PY{n+nn}{math}
        \PY{k+kn}{from} \PY{n+nn}{scipy} \PY{k}{import} \PY{n}{signal}
        \PY{k+kn}{from} \PY{n+nn}{skimage} \PY{k}{import} \PY{n}{io}\PY{p}{,} \PY{n}{color}
        \PY{k+kn}{from} \PY{n+nn}{skimage} \PY{k}{import} \PY{n}{exposure}
        
        \PY{n}{file\PYZus{}image}	\PY{o}{=} \PY{l+s+s1}{\PYZsq{}}\PY{l+s+s1}{cau\PYZhy{}resized.jpg}\PY{l+s+s1}{\PYZsq{}}
        
        \PY{n}{im\PYZus{}color} 	\PY{o}{=} \PY{n}{io}\PY{o}{.}\PY{n}{imread}\PY{p}{(}\PY{n}{file\PYZus{}image}\PY{p}{)}
        \PY{n}{im\PYZus{}gray}  	\PY{o}{=} \PY{n}{color}\PY{o}{.}\PY{n}{rgb2gray}\PY{p}{(}\PY{n}{im\PYZus{}color}\PY{p}{)}
        \PY{n}{im}          \PY{o}{=} \PY{p}{(}\PY{n}{im\PYZus{}gray} \PY{o}{\PYZhy{}} \PY{n}{np}\PY{o}{.}\PY{n}{mean}\PY{p}{(}\PY{n}{im\PYZus{}gray}\PY{p}{)}\PY{p}{)} \PY{o}{/} \PY{n}{np}\PY{o}{.}\PY{n}{std}\PY{p}{(}\PY{n}{im\PYZus{}gray}\PY{p}{)}
        \PY{n}{row}\PY{p}{,} \PY{n}{col}    \PY{o}{=} \PY{n}{im}\PY{o}{.}\PY{n}{shape}
\end{Verbatim}


    \subsubsection{Function for generate noisy
image}\label{function-for-generate-noisy-image}

Here we add some noise on original image by normal distribution with
mean 0 and standard deviation \(\sigma\).

    \begin{Verbatim}[commandchars=\\\{\}]
{\color{incolor}In [{\color{incolor}38}]:} \PY{k}{def} \PY{n+nf}{create\PYZus{}noisy\PYZus{}img}\PY{p}{(}\PY{n}{img}\PY{p}{,} \PY{n}{noise\PYZus{}std}\PY{p}{)}\PY{p}{:}
             \PY{l+s+sd}{\PYZdq{}\PYZdq{}\PYZdq{}Add noise on image}
         \PY{l+s+sd}{    }
         \PY{l+s+sd}{    Arguments:}
         \PY{l+s+sd}{        img(np.matrix): input clear image.}
         \PY{l+s+sd}{        noise\PYZus{}std: noise standard deviation.}
         \PY{l+s+sd}{    Return:}
         \PY{l+s+sd}{        noisy image}
         \PY{l+s+sd}{    \PYZdq{}\PYZdq{}\PYZdq{}}
             
             \PY{n}{row}\PY{p}{,} \PY{n}{col} \PY{o}{=} \PY{n}{im}\PY{o}{.}\PY{n}{shape}
         
             \PY{n}{noise} \PY{o}{=} \PY{n}{np}\PY{o}{.}\PY{n}{random}\PY{o}{.}\PY{n}{normal}\PY{p}{(}\PY{l+m+mi}{0}\PY{p}{,} \PY{n}{noise\PYZus{}std}\PY{p}{,} \PY{p}{(}\PY{n}{row}\PY{p}{,} \PY{n}{col}\PY{p}{)}\PY{p}{)}
             \PY{n}{im\PYZus{}noise} \PY{o}{=} \PY{n}{im} \PY{o}{+} \PY{n}{noise}
             
             \PY{k}{return} \PY{n}{im\PYZus{}noise}
\end{Verbatim}


    \subsubsection{Build cost function}\label{build-cost-function}

As we talked above, to build cost function, we need a \(M \times N\)
identity matrix \(I\), a \(M(N-1) \times MN\) matrix \(D_{h}\), and a
\((M-1)N \times MN\) matrix \(D_{v}\) then combine them to a big matrix
\(\tilde{A}\).

    \begin{Verbatim}[commandchars=\\\{\}]
{\color{incolor}In [{\color{incolor}3}]:} \PY{k}{def} \PY{n+nf}{create\PYZus{}tilde\PYZus{}matrix}\PY{p}{(}\PY{n}{row}\PY{p}{,} \PY{n}{col}\PY{p}{,} \PY{n}{weight}\PY{p}{)}\PY{p}{:}
            \PY{l+s+sd}{\PYZdq{}\PYZdq{}\PYZdq{}Build tilde matrix A in cost function}
        \PY{l+s+sd}{    }
        \PY{l+s+sd}{    Arguments:}
        \PY{l+s+sd}{        row: \PYZsh{}row of image}
        \PY{l+s+sd}{        col: \PYZsh{}column of image}
        \PY{l+s+sd}{        weight: weight of secondary objective}
        \PY{l+s+sd}{    Return:}
        \PY{l+s+sd}{        matrix row: row*(col\PYZhy{}1) + (row\PYZhy{}1)*col + row*col}
        \PY{l+s+sd}{               column: row * col}
        \PY{l+s+sd}{    \PYZdq{}\PYZdq{}\PYZdq{}}
            
            \PY{n}{I} \PY{o}{=} \PY{n}{np}\PY{o}{.}\PY{n}{identity}\PY{p}{(}\PY{n}{row}\PY{o}{*}\PY{n}{col}\PY{p}{)}
            \PY{n}{Dx\PYZus{}weight} \PY{o}{=} \PY{n}{deri\PYZus{}h}\PY{p}{(}\PY{n}{row}\PY{p}{,} \PY{n}{col}\PY{p}{,} \PY{n}{weight}\PY{p}{)}
            \PY{n}{Dy\PYZus{}weight} \PY{o}{=} \PY{n}{deri\PYZus{}v}\PY{p}{(}\PY{n}{row}\PY{p}{,} \PY{n}{col}\PY{p}{,} \PY{n}{weight}\PY{p}{)}
            
            \PY{n}{A} \PY{o}{=} \PY{n}{np}\PY{o}{.}\PY{n}{vstack}\PY{p}{(}\PY{p}{(}\PY{n}{I}\PY{p}{,} \PY{n}{Dx\PYZus{}weight}\PY{p}{,} \PY{n}{Dy\PYZus{}weight}\PY{p}{)}\PY{p}{)}
            
            \PY{k}{return} \PY{n}{A}
        
        
        \PY{k}{def} \PY{n+nf}{deri\PYZus{}h}\PY{p}{(}\PY{n}{row}\PY{p}{,} \PY{n}{col}\PY{p}{,} \PY{n}{weight}\PY{o}{=}\PY{l+m+mi}{1}\PY{p}{)}\PY{p}{:}
            \PY{n}{dh} \PY{o}{=} \PY{n}{np}\PY{o}{.}\PY{n}{zeros}\PY{p}{(}\PY{p}{(}\PY{n}{row}\PY{o}{*}\PY{p}{(}\PY{n}{col}\PY{o}{\PYZhy{}}\PY{l+m+mi}{1}\PY{p}{)}\PY{p}{,} \PY{n}{row}\PY{o}{*}\PY{n}{col}\PY{p}{)}\PY{p}{)}
            
            \PY{k}{for} \PY{n}{i} \PY{o+ow}{in} \PY{n+nb}{range}\PY{p}{(}\PY{n}{row}\PY{o}{*}\PY{p}{(}\PY{n}{col}\PY{o}{\PYZhy{}}\PY{l+m+mi}{1}\PY{p}{)}\PY{p}{)}\PY{p}{:}
                \PY{n}{dh}\PY{p}{[}\PY{n}{i}\PY{p}{]}\PY{p}{[}\PY{n}{i}\PY{p}{]} \PY{o}{=} \PY{o}{\PYZhy{}}\PY{l+m+mi}{1}
                \PY{n}{dh}\PY{p}{[}\PY{n}{i}\PY{p}{]}\PY{p}{[}\PY{n}{i}\PY{o}{+}\PY{n}{row}\PY{p}{]} \PY{o}{=} \PY{l+m+mi}{1}
                
            \PY{n}{dh} \PY{o}{=} \PY{n}{math}\PY{o}{.}\PY{n}{sqrt}\PY{p}{(}\PY{n}{weight}\PY{p}{)} \PY{o}{*} \PY{n}{dh}
            
            \PY{k}{return} \PY{n}{dh}
        
        
        \PY{k}{def} \PY{n+nf}{deri\PYZus{}v}\PY{p}{(}\PY{n}{row}\PY{p}{,} \PY{n}{col}\PY{p}{,} \PY{n}{weight}\PY{o}{=}\PY{l+m+mi}{1}\PY{p}{)}\PY{p}{:}
            \PY{n}{m} \PY{o}{=} \PY{n}{np}\PY{o}{.}\PY{n}{zeros}\PY{p}{(}\PY{p}{(}\PY{n}{row}\PY{o}{\PYZhy{}}\PY{l+m+mi}{1}\PY{p}{,} \PY{n}{row}\PY{p}{)}\PY{p}{)}
            \PY{k}{for} \PY{n}{i} \PY{o+ow}{in} \PY{n+nb}{range}\PY{p}{(}\PY{n}{row}\PY{o}{\PYZhy{}}\PY{l+m+mi}{1}\PY{p}{)}\PY{p}{:}
                \PY{n}{m}\PY{p}{[}\PY{n}{i}\PY{p}{]}\PY{p}{[}\PY{n}{i}\PY{p}{]} \PY{o}{=} \PY{o}{\PYZhy{}}\PY{l+m+mi}{1}
                \PY{n}{m}\PY{p}{[}\PY{n}{i}\PY{p}{]}\PY{p}{[}\PY{n}{i}\PY{o}{+}\PY{l+m+mi}{1}\PY{p}{]} \PY{o}{=} \PY{l+m+mi}{1}
            
            \PY{n}{ident} \PY{o}{=} \PY{n}{np}\PY{o}{.}\PY{n}{identity}\PY{p}{(}\PY{n}{col}\PY{p}{)}
            
            \PY{n}{dv} \PY{o}{=} \PY{n}{np}\PY{o}{.}\PY{n}{kron}\PY{p}{(}\PY{n}{ident}\PY{p}{,} \PY{n}{m}\PY{p}{)}
            \PY{n}{dv} \PY{o}{=} \PY{n}{math}\PY{o}{.}\PY{n}{sqrt}\PY{p}{(}\PY{n}{weight}\PY{p}{)} \PY{o}{*} \PY{n}{dv}
            
            \PY{k}{return} \PY{n}{dv}
\end{Verbatim}


    \paragraph{\texorpdfstring{Generate
\(\tilde{b}\)}{Generate \textbackslash{}tilde\{b\}}}\label{generate-tildeb}

    \begin{Verbatim}[commandchars=\\\{\}]
{\color{incolor}In [{\color{incolor}5}]:} \PY{k}{def} \PY{n+nf}{create\PYZus{}tilde\PYZus{}b}\PY{p}{(}\PY{n}{row}\PY{p}{,} \PY{n}{col}\PY{p}{,} \PY{n}{img}\PY{p}{)}\PY{p}{:}
            \PY{n}{length} \PY{o}{=} \PY{p}{(}\PY{n}{row}\PY{o}{\PYZhy{}}\PY{l+m+mi}{1}\PY{p}{)}\PY{o}{*}\PY{n}{col} \PY{o}{+} \PY{n}{row}\PY{o}{*}\PY{p}{(}\PY{n}{col}\PY{o}{\PYZhy{}}\PY{l+m+mi}{1}\PY{p}{)} \PY{o}{+} \PY{n}{row}\PY{o}{*}\PY{n}{col}
            
            \PY{n}{b} \PY{o}{=} \PY{n}{np}\PY{o}{.}\PY{n}{zeros}\PY{p}{(}\PY{n}{length}\PY{p}{)}
            \PY{n}{img\PYZus{}vec} \PY{o}{=} \PY{p}{[}\PY{p}{]}
            
            \PY{k}{for} \PY{n}{i} \PY{o+ow}{in} \PY{n+nb}{range}\PY{p}{(}\PY{n}{col}\PY{p}{)}\PY{p}{:}
                \PY{n}{column} \PY{o}{=} \PY{n}{img}\PY{p}{[}\PY{p}{:}\PY{p}{,} \PY{n}{i}\PY{p}{]}
                \PY{k}{for} \PY{n}{j} \PY{o+ow}{in} \PY{n+nb}{range}\PY{p}{(}\PY{n}{row}\PY{p}{)}\PY{p}{:}
                    \PY{n}{img\PYZus{}vec}\PY{o}{.}\PY{n}{append}\PY{p}{(}\PY{n}{column}\PY{p}{[}\PY{n}{j}\PY{p}{]}\PY{p}{)}
                
            \PY{k}{for} \PY{n}{k} \PY{o+ow}{in} \PY{n+nb}{range}\PY{p}{(}\PY{n}{row}\PY{o}{*}\PY{n}{col}\PY{p}{)}\PY{p}{:}
                \PY{n}{b}\PY{p}{[}\PY{n}{k}\PY{p}{]} \PY{o}{=} \PY{n}{img\PYZus{}vec}\PY{p}{[}\PY{n}{k}\PY{p}{]}
            
            \PY{k}{return} \PY{n}{b}
\end{Verbatim}


    \paragraph{\texorpdfstring{Compute
\(\hat{x}\)}{Compute \textbackslash{}hat\{x\}}}\label{compute-hatx}

    \begin{Verbatim}[commandchars=\\\{\}]
{\color{incolor}In [{\color{incolor}39}]:} \PY{k}{def} \PY{n+nf}{compute\PYZus{}param}\PY{p}{(}\PY{n}{A}\PY{p}{,} \PY{n}{y}\PY{p}{)}\PY{p}{:}
             \PY{k}{return} \PY{n}{np}\PY{o}{.}\PY{n}{linalg}\PY{o}{.}\PY{n}{lstsq}\PY{p}{(}\PY{n}{A}\PY{p}{,} \PY{n}{y}\PY{p}{,} \PY{n}{rcond}\PY{o}{=}\PY{k+kc}{None}\PY{p}{)}\PY{p}{[}\PY{l+m+mi}{0}\PY{p}{]}
\end{Verbatim}


    \subsubsection{De-noising image}\label{de-noising-image}

Since we have got all the components of the cost function, now we can
compute the approximate inversion \(\hat{x}\)

    \begin{Verbatim}[commandchars=\\\{\}]
{\color{incolor}In [{\color{incolor}47}]:} \PY{k}{def} \PY{n+nf}{denoising}\PY{p}{(}\PY{n}{row}\PY{p}{,} \PY{n}{col}\PY{p}{,} \PY{n}{weight}\PY{p}{,} \PY{n}{img\PYZus{}noisy}\PY{p}{)}\PY{p}{:}
             \PY{l+s+sd}{\PYZdq{}\PYZdq{}\PYZdq{}de\PYZhy{}noising image}
         \PY{l+s+sd}{    }
         \PY{l+s+sd}{    Arguments:}
         \PY{l+s+sd}{        row: \PYZsh{}row of image}
         \PY{l+s+sd}{        col: \PYZsh{}col of image}
         \PY{l+s+sd}{        weight: weight for secondary objective}
         \PY{l+s+sd}{        img\PYZus{}noisy: noisy image}
         \PY{l+s+sd}{    Return:}
         \PY{l+s+sd}{        img\PYZus{}recon: de\PYZhy{}noised image}
         \PY{l+s+sd}{        error: error of this denoising}
         \PY{l+s+sd}{    \PYZdq{}\PYZdq{}\PYZdq{}}
             \PY{c+c1}{\PYZsh{} tranform image to vector}
             \PY{n}{b} \PY{o}{=} \PY{n}{create\PYZus{}tilde\PYZus{}b}\PY{p}{(}\PY{n}{row}\PY{p}{,} \PY{n}{col}\PY{p}{,} \PY{n}{im\PYZus{}noise}\PY{p}{)}
         
             \PY{c+c1}{\PYZsh{} generate matrix A}
             \PY{n}{matrix\PYZus{}A} \PY{o}{=} \PY{n}{create\PYZus{}tilde\PYZus{}matrix}\PY{p}{(}\PY{n}{row}\PY{p}{,} \PY{n}{col}\PY{p}{,} \PY{n}{weight}\PY{p}{)}
         
             \PY{c+c1}{\PYZsh{} compute recon image}
             \PY{n}{img\PYZus{}recon} \PY{o}{=} \PY{n}{compute\PYZus{}param}\PY{p}{(}\PY{n}{matrix\PYZus{}A}\PY{p}{,} \PY{n}{b}\PY{p}{)}
             
             \PY{c+c1}{\PYZsh{} compute error}
             \PY{n}{cost\PYZus{}func} \PY{o}{=} \PY{n}{np}\PY{o}{.}\PY{n}{inner}\PY{p}{(}\PY{n}{matrix\PYZus{}A}\PY{p}{,} \PY{n}{img\PYZus{}recon}\PY{p}{)} \PY{o}{\PYZhy{}} \PY{n}{b}
             \PY{n}{error} \PY{o}{=} \PY{n}{np}\PY{o}{.}\PY{n}{linalg}\PY{o}{.}\PY{n}{norm}\PY{p}{(}\PY{n}{cost\PYZus{}func}\PY{p}{)}\PY{o}{*}\PY{o}{*}\PY{l+m+mi}{2}
             
             \PY{c+c1}{\PYZsh{} transform to matrix shape}
             \PY{n}{img\PYZus{}recon} \PY{o}{=} \PY{p}{(}\PY{n}{img\PYZus{}recon}\PY{o}{.}\PY{n}{reshape}\PY{p}{(}\PY{p}{(}\PY{n}{col}\PY{p}{,} \PY{n}{row}\PY{p}{)}\PY{p}{)}\PY{p}{)}\PY{o}{.}\PY{n}{T}
             
             
             \PY{k}{return} \PY{n}{img\PYZus{}recon}\PY{p}{,} \PY{n}{error}
\end{Verbatim}


    \begin{center}\rule{0.5\linewidth}{\linethickness}\end{center}

\subsection{De-noising Images}\label{de-noising-images}

Now, let's try our \texttt{denoising} function to do some image
de-noising.

\subsubsection{\texorpdfstring{Try different standard deviation
\(\sigma\) and regularization parameter
\(\lambda\)}{Try different standard deviation \textbackslash{}sigma and regularization parameter \textbackslash{}lambda}}\label{try-different-standard-deviation-sigma-and-regularization-parameter-lambda}

First create a function to convenient plotting.

    \begin{Verbatim}[commandchars=\\\{\}]
{\color{incolor}In [{\color{incolor}54}]:} \PY{k}{def} \PY{n+nf}{plot\PYZus{}images}\PY{p}{(}\PY{n}{img}\PY{p}{,} \PY{n}{img\PYZus{}noisy}\PY{p}{,} \PY{n}{img\PYZus{}recon}\PY{p}{)}\PY{p}{:}
             \PY{n}{noise\PYZus{}recon} \PY{o}{=} \PY{n}{img\PYZus{}noisy} \PY{o}{\PYZhy{}} \PY{n}{img\PYZus{}recon}
         
             \PY{n}{p1} \PY{o}{=} \PY{n}{plt}\PY{o}{.}\PY{n}{subplot}\PY{p}{(}\PY{l+m+mi}{2}\PY{p}{,}\PY{l+m+mi}{2}\PY{p}{,}\PY{l+m+mi}{1}\PY{p}{)}
             \PY{n}{p1}\PY{o}{.}\PY{n}{set\PYZus{}title}\PY{p}{(}\PY{l+s+s1}{\PYZsq{}}\PY{l+s+s1}{original image}\PY{l+s+s1}{\PYZsq{}}\PY{p}{)}
             \PY{n}{plt}\PY{o}{.}\PY{n}{imshow}\PY{p}{(}\PY{n}{img}\PY{p}{,} \PY{n}{cmap}\PY{o}{=}\PY{l+s+s1}{\PYZsq{}}\PY{l+s+s1}{gray}\PY{l+s+s1}{\PYZsq{}}\PY{p}{)}
             \PY{n}{plt}\PY{o}{.}\PY{n}{axis}\PY{p}{(}\PY{l+s+s1}{\PYZsq{}}\PY{l+s+s1}{off}\PY{l+s+s1}{\PYZsq{}}\PY{p}{)}
         
             \PY{n}{p2} \PY{o}{=} \PY{n}{plt}\PY{o}{.}\PY{n}{subplot}\PY{p}{(}\PY{l+m+mi}{2}\PY{p}{,}\PY{l+m+mi}{2}\PY{p}{,}\PY{l+m+mi}{2}\PY{p}{)}
             \PY{n}{p2}\PY{o}{.}\PY{n}{set\PYZus{}title}\PY{p}{(}\PY{l+s+s1}{\PYZsq{}}\PY{l+s+s1}{noisy image}\PY{l+s+s1}{\PYZsq{}}\PY{p}{)}
             \PY{n}{plt}\PY{o}{.}\PY{n}{imshow}\PY{p}{(}\PY{n}{img\PYZus{}noisy}\PY{p}{,} \PY{n}{cmap}\PY{o}{=}\PY{l+s+s1}{\PYZsq{}}\PY{l+s+s1}{gray}\PY{l+s+s1}{\PYZsq{}}\PY{p}{)}
             \PY{n}{plt}\PY{o}{.}\PY{n}{axis}\PY{p}{(}\PY{l+s+s1}{\PYZsq{}}\PY{l+s+s1}{off}\PY{l+s+s1}{\PYZsq{}}\PY{p}{)}
         
             \PY{n}{p3} \PY{o}{=} \PY{n}{plt}\PY{o}{.}\PY{n}{subplot}\PY{p}{(}\PY{l+m+mi}{2}\PY{p}{,}\PY{l+m+mi}{2}\PY{p}{,}\PY{l+m+mi}{3}\PY{p}{)}
             \PY{n}{p3}\PY{o}{.}\PY{n}{set\PYZus{}title}\PY{p}{(}\PY{l+s+s1}{\PYZsq{}}\PY{l+s+s1}{reconstruction}\PY{l+s+s1}{\PYZsq{}}\PY{p}{)}
             \PY{n}{plt}\PY{o}{.}\PY{n}{imshow}\PY{p}{(}\PY{n}{img\PYZus{}recon}\PY{p}{,} \PY{n}{cmap}\PY{o}{=}\PY{l+s+s1}{\PYZsq{}}\PY{l+s+s1}{gray}\PY{l+s+s1}{\PYZsq{}}\PY{p}{)}
             \PY{n}{plt}\PY{o}{.}\PY{n}{axis}\PY{p}{(}\PY{l+s+s1}{\PYZsq{}}\PY{l+s+s1}{off}\PY{l+s+s1}{\PYZsq{}}\PY{p}{)}
         
             \PY{n}{p4} \PY{o}{=} \PY{n}{plt}\PY{o}{.}\PY{n}{subplot}\PY{p}{(}\PY{l+m+mi}{2}\PY{p}{,}\PY{l+m+mi}{2}\PY{p}{,}\PY{l+m+mi}{4}\PY{p}{)}
             \PY{n}{p4}\PY{o}{.}\PY{n}{set\PYZus{}title}\PY{p}{(}\PY{l+s+s1}{\PYZsq{}}\PY{l+s+s1}{estimated noise}\PY{l+s+s1}{\PYZsq{}}\PY{p}{)}
             \PY{n}{plt}\PY{o}{.}\PY{n}{imshow}\PY{p}{(}\PY{n}{noise\PYZus{}recon}\PY{p}{,} \PY{n}{cmap}\PY{o}{=}\PY{l+s+s1}{\PYZsq{}}\PY{l+s+s1}{gray}\PY{l+s+s1}{\PYZsq{}}\PY{p}{)}
             \PY{n}{plt}\PY{o}{.}\PY{n}{axis}\PY{p}{(}\PY{l+s+s1}{\PYZsq{}}\PY{l+s+s1}{off}\PY{l+s+s1}{\PYZsq{}}\PY{p}{)}
         
             \PY{n}{plt}\PY{o}{.}\PY{n}{show}\PY{p}{(}\PY{p}{)}
\end{Verbatim}


    \begin{enumerate}
\def\labelenumi{\arabic{enumi}.}
\tightlist
\item
  Try \(\sigma = 0.2\), \(\lambda = 0.25\)
\end{enumerate}

    \begin{Verbatim}[commandchars=\\\{\}]
{\color{incolor}In [{\color{incolor}56}]:} \PY{n}{noise\PYZus{}std} \PY{o}{=} \PY{l+m+mf}{0.2}
         \PY{n}{weight} \PY{o}{=} \PY{l+m+mf}{0.25}
         
         \PY{c+c1}{\PYZsh{} generate noisy image}
         \PY{n}{img\PYZus{}noisy} \PY{o}{=} \PY{n}{create\PYZus{}noisy\PYZus{}img}\PY{p}{(}\PY{n}{im}\PY{p}{,} \PY{n}{noise\PYZus{}std}\PY{p}{)}
         
         \PY{c+c1}{\PYZsh{} denoising}
         \PY{n}{img\PYZus{}recon}\PY{p}{,} \PY{n}{error} \PY{o}{=} \PY{n}{denoising}\PY{p}{(}\PY{n}{row}\PY{p}{,} \PY{n}{col}\PY{p}{,} \PY{n}{weight}\PY{p}{,} \PY{n}{img\PYZus{}noisy}\PY{p}{)}
         
         \PY{n}{plot\PYZus{}images}\PY{p}{(}\PY{n}{im}\PY{p}{,} \PY{n}{img\PYZus{}noisy}\PY{p}{,} \PY{n}{img\PYZus{}recon}\PY{p}{)}
\end{Verbatim}


    \begin{center}
    \adjustimage{max size={0.9\linewidth}{0.9\paperheight}}{output_16_0.png}
    \end{center}
    { \hspace*{\fill} \\}
    
    \begin{enumerate}
\def\labelenumi{\arabic{enumi}.}
\setcounter{enumi}{1}
\tightlist
\item
  Try \(\sigma = 0.5\), \(\lambda = 0.25\)
\end{enumerate}

    \begin{Verbatim}[commandchars=\\\{\}]
{\color{incolor}In [{\color{incolor}57}]:} \PY{n}{noise\PYZus{}std} \PY{o}{=} \PY{l+m+mf}{0.5}
         \PY{n}{weight} \PY{o}{=} \PY{l+m+mf}{0.25}
         
         \PY{n}{img\PYZus{}noisy} \PY{o}{=} \PY{n}{create\PYZus{}noisy\PYZus{}img}\PY{p}{(}\PY{n}{im}\PY{p}{,} \PY{n}{noise\PYZus{}std}\PY{p}{)}
         \PY{n}{img\PYZus{}recon}\PY{p}{,} \PY{n}{error} \PY{o}{=} \PY{n}{denoising}\PY{p}{(}\PY{n}{row}\PY{p}{,} \PY{n}{col}\PY{p}{,} \PY{n}{weight}\PY{p}{,} \PY{n}{img\PYZus{}noisy}\PY{p}{)}
         
         \PY{n}{plot\PYZus{}images}\PY{p}{(}\PY{n}{im}\PY{p}{,} \PY{n}{img\PYZus{}noisy}\PY{p}{,} \PY{n}{img\PYZus{}recon}\PY{p}{)}
\end{Verbatim}


    \begin{center}
    \adjustimage{max size={0.9\linewidth}{0.9\paperheight}}{output_18_0.png}
    \end{center}
    { \hspace*{\fill} \\}
    
    \begin{enumerate}
\def\labelenumi{\arabic{enumi}.}
\setcounter{enumi}{2}
\tightlist
\item
  Try \(\sigma = 0.2\), \(\lambda = 1\)
\end{enumerate}

    \begin{Verbatim}[commandchars=\\\{\}]
{\color{incolor}In [{\color{incolor}58}]:} \PY{n}{noise\PYZus{}std} \PY{o}{=} \PY{l+m+mf}{0.2}
         \PY{n}{weight} \PY{o}{=} \PY{l+m+mi}{1}
         
         \PY{n}{img\PYZus{}noisy} \PY{o}{=} \PY{n}{create\PYZus{}noisy\PYZus{}img}\PY{p}{(}\PY{n}{im}\PY{p}{,} \PY{n}{noise\PYZus{}std}\PY{p}{)}
         \PY{n}{img\PYZus{}recon}\PY{p}{,} \PY{n}{error} \PY{o}{=} \PY{n}{denoising}\PY{p}{(}\PY{n}{row}\PY{p}{,} \PY{n}{col}\PY{p}{,} \PY{n}{weight}\PY{p}{,} \PY{n}{img\PYZus{}noisy}\PY{p}{)}
         
         \PY{n}{plot\PYZus{}images}\PY{p}{(}\PY{n}{im}\PY{p}{,} \PY{n}{img\PYZus{}noisy}\PY{p}{,} \PY{n}{img\PYZus{}recon}\PY{p}{)}
\end{Verbatim}


    \begin{center}
    \adjustimage{max size={0.9\linewidth}{0.9\paperheight}}{output_20_0.png}
    \end{center}
    { \hspace*{\fill} \\}
    
    \begin{enumerate}
\def\labelenumi{\arabic{enumi}.}
\setcounter{enumi}{3}
\tightlist
\item
  Try \(\sigma = 0.5\), \(\lambda = 1\)
\end{enumerate}

    \begin{Verbatim}[commandchars=\\\{\}]
{\color{incolor}In [{\color{incolor}59}]:} \PY{n}{noise\PYZus{}std} \PY{o}{=} \PY{l+m+mf}{0.5}
         \PY{n}{weight} \PY{o}{=} \PY{l+m+mi}{1}
         
         \PY{n}{img\PYZus{}noisy} \PY{o}{=} \PY{n}{create\PYZus{}noisy\PYZus{}img}\PY{p}{(}\PY{n}{im}\PY{p}{,} \PY{n}{noise\PYZus{}std}\PY{p}{)}
         \PY{n}{img\PYZus{}recon}\PY{p}{,} \PY{n}{error} \PY{o}{=} \PY{n}{denoising}\PY{p}{(}\PY{n}{row}\PY{p}{,} \PY{n}{col}\PY{p}{,} \PY{n}{weight}\PY{p}{,} \PY{n}{img\PYZus{}noisy}\PY{p}{)}
         
         \PY{n}{plot\PYZus{}images}\PY{p}{(}\PY{n}{im}\PY{p}{,} \PY{n}{img\PYZus{}noisy}\PY{p}{,} \PY{n}{img\PYZus{}recon}\PY{p}{)}
\end{Verbatim}


    \begin{center}
    \adjustimage{max size={0.9\linewidth}{0.9\paperheight}}{output_22_0.png}
    \end{center}
    { \hspace*{\fill} \\}
    
    Since I do not get memory to compute bigger image, these may not very
clear.

But we still can see from the above images, when \(\sigma\) bigger the
noise is more, when \(\lambda\) get bigger the recon images are becoming
smoother.

    \subsubsection{\texorpdfstring{See how \(\lambda\) affects
result}{See how \textbackslash{}lambda affects result}}\label{see-how-lambda-affects-result}

Let's fix \(\sigma\) and try different \(\lambda s\) to see the result.

    \begin{Verbatim}[commandchars=\\\{\}]
{\color{incolor}In [{\color{incolor}79}]:} \PY{n}{lambdas} \PY{o}{=} \PY{p}{[}\PY{l+m+mf}{0.5}\PY{o}{*}\PY{o}{*}\PY{n}{x} \PY{k}{for} \PY{n}{x} \PY{o+ow}{in} \PY{n+nb}{range}\PY{p}{(}\PY{o}{\PYZhy{}}\PY{l+m+mi}{5}\PY{p}{,} \PY{l+m+mi}{11}\PY{p}{)}\PY{p}{]}
         \PY{n}{img\PYZus{}noisy} \PY{o}{=} \PY{n}{create\PYZus{}noisy\PYZus{}img}\PY{p}{(}\PY{n}{im}\PY{p}{,} \PY{l+m+mf}{0.5}\PY{p}{)}
         
         \PY{n}{error\PYZus{}histo} \PY{o}{=} \PY{p}{[}\PY{p}{]}
         \PY{n}{imgs} \PY{o}{=} \PY{p}{[}\PY{p}{]}
         
         \PY{k}{for} \PY{n}{weight} \PY{o+ow}{in} \PY{n}{lambdas}\PY{p}{:}
             \PY{n}{img\PYZus{}recon}\PY{p}{,} \PY{n}{error} \PY{o}{=}\PY{n}{denoising}\PY{p}{(}\PY{n}{row}\PY{p}{,} \PY{n}{col}\PY{p}{,} \PY{n}{weight}\PY{p}{,} \PY{n}{img\PYZus{}noisy}\PY{p}{)}
             \PY{n}{imgs}\PY{o}{.}\PY{n}{append}\PY{p}{(}\PY{n}{img\PYZus{}recon}\PY{p}{)}
             \PY{n}{error\PYZus{}histo}\PY{o}{.}\PY{n}{append}\PY{p}{(}\PY{n}{error}\PY{p}{)}
\end{Verbatim}


    \begin{Verbatim}[commandchars=\\\{\}]
{\color{incolor}In [{\color{incolor}95}]:} \PY{n}{plt}\PY{o}{.}\PY{n}{figure}\PY{p}{(}\PY{n}{figsize}\PY{o}{=}\PY{p}{(}\PY{l+m+mi}{16}\PY{p}{,} \PY{l+m+mi}{16}\PY{p}{)}\PY{p}{)}
         \PY{k}{for} \PY{n}{i} \PY{o+ow}{in} \PY{n+nb}{range}\PY{p}{(}\PY{n+nb}{len}\PY{p}{(}\PY{n}{error\PYZus{}histo}\PY{p}{)}\PY{p}{)}\PY{p}{:}  
             \PY{n}{p} \PY{o}{=} \PY{n}{plt}\PY{o}{.}\PY{n}{subplot}\PY{p}{(}\PY{l+m+mi}{6}\PY{p}{,}\PY{l+m+mi}{3}\PY{p}{,}\PY{n}{i}\PY{o}{+}\PY{l+m+mi}{1}\PY{p}{)}
             \PY{n}{img\PYZus{}recon}\PY{o}{=} \PY{n}{imgs}\PY{p}{[}\PY{n}{i}\PY{p}{]}
             \PY{n}{p}\PY{o}{.}\PY{n}{set\PYZus{}title}\PY{p}{(}\PY{l+s+s1}{\PYZsq{}}\PY{l+s+s1}{weight = }\PY{l+s+si}{\PYZob{}\PYZcb{}}\PY{l+s+s1}{\PYZsq{}}\PY{o}{.}\PY{n}{format}\PY{p}{(}\PY{n}{lambdas}\PY{p}{[}\PY{n}{i}\PY{p}{]}\PY{p}{)}\PY{p}{)}
             \PY{n}{plt}\PY{o}{.}\PY{n}{imshow}\PY{p}{(}\PY{n}{img\PYZus{}recon}\PY{p}{,} \PY{n}{cmap}\PY{o}{=}\PY{l+s+s1}{\PYZsq{}}\PY{l+s+s1}{gray}\PY{l+s+s1}{\PYZsq{}}\PY{p}{)}
             \PY{n}{plt}\PY{o}{.}\PY{n}{axis}\PY{p}{(}\PY{l+s+s1}{\PYZsq{}}\PY{l+s+s1}{off}\PY{l+s+s1}{\PYZsq{}}\PY{p}{)}
             
         \PY{n}{plt}\PY{o}{.}\PY{n}{show}\PY{p}{(}\PY{p}{)}
\end{Verbatim}


    \begin{center}
    \adjustimage{max size={0.9\linewidth}{0.9\paperheight}}{output_26_0.png}
    \end{center}
    { \hspace*{\fill} \\}
    
    Let's see how the error of of reconstruction changes.

Error function is
\(E(u) = ||u-f||_{2}^{2} + \lambda||\nabla u||_{2}^{2}\)

    \begin{Verbatim}[commandchars=\\\{\}]
{\color{incolor}In [{\color{incolor}98}]:} \PY{n}{plt}\PY{o}{.}\PY{n}{title}\PY{p}{(}\PY{l+s+s2}{\PYZdq{}}\PY{l+s+s2}{Reconstruction error with diff weight}\PY{l+s+s2}{\PYZdq{}}\PY{p}{)}
         \PY{n}{plt}\PY{o}{.}\PY{n}{plot}\PY{p}{(}\PY{n}{lambdas}\PY{p}{,} \PY{n}{error\PYZus{}histo}\PY{p}{,} \PY{l+s+s1}{\PYZsq{}}\PY{l+s+s1}{b\PYZhy{}}\PY{l+s+s1}{\PYZsq{}}\PY{p}{)}
         \PY{n}{plt}\PY{o}{.}\PY{n}{xlabel}\PY{p}{(}\PY{l+s+s1}{\PYZsq{}}\PY{l+s+s1}{Lambda}\PY{l+s+s1}{\PYZsq{}}\PY{p}{)}
         \PY{n}{plt}\PY{o}{.}\PY{n}{ylabel}\PY{p}{(}\PY{l+s+s1}{\PYZsq{}}\PY{l+s+s1}{Error}\PY{l+s+s1}{\PYZsq{}}\PY{p}{)}
         \PY{n}{plt}\PY{o}{.}\PY{n}{show}\PY{p}{(}\PY{p}{)}
\end{Verbatim}


    \begin{center}
    \adjustimage{max size={0.9\linewidth}{0.9\paperheight}}{output_28_0.png}
    \end{center}
    { \hspace*{\fill} \\}
    
    When \(\lambda\) becomes bigger, reconstructed images are more smooth,
so the error between it with input noisy image also get bigger.


    % Add a bibliography block to the postdoc
    
    
    
    \end{document}
