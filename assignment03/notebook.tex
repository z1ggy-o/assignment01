
% Default to the notebook output style

    


% Inherit from the specified cell style.




    
\documentclass[11pt]{article}

    
    
    \usepackage[T1]{fontenc}
    % Nicer default font (+ math font) than Computer Modern for most use cases
    \usepackage{mathpazo}

    % Basic figure setup, for now with no caption control since it's done
    % automatically by Pandoc (which extracts ![](path) syntax from Markdown).
    \usepackage{graphicx}
    % We will generate all images so they have a width \maxwidth. This means
    % that they will get their normal width if they fit onto the page, but
    % are scaled down if they would overflow the margins.
    \makeatletter
    \def\maxwidth{\ifdim\Gin@nat@width>\linewidth\linewidth
    \else\Gin@nat@width\fi}
    \makeatother
    \let\Oldincludegraphics\includegraphics
    % Set max figure width to be 80% of text width, for now hardcoded.
    \renewcommand{\includegraphics}[1]{\Oldincludegraphics[width=.8\maxwidth]{#1}}
    % Ensure that by default, figures have no caption (until we provide a
    % proper Figure object with a Caption API and a way to capture that
    % in the conversion process - todo).
    \usepackage{caption}
    \DeclareCaptionLabelFormat{nolabel}{}
    \captionsetup{labelformat=nolabel}

    \usepackage{adjustbox} % Used to constrain images to a maximum size 
    \usepackage{xcolor} % Allow colors to be defined
    \usepackage{enumerate} % Needed for markdown enumerations to work
    \usepackage{geometry} % Used to adjust the document margins
    \usepackage{amsmath} % Equations
    \usepackage{amssymb} % Equations
    \usepackage{textcomp} % defines textquotesingle
    % Hack from http://tex.stackexchange.com/a/47451/13684:
    \AtBeginDocument{%
        \def\PYZsq{\textquotesingle}% Upright quotes in Pygmentized code
    }
    \usepackage{upquote} % Upright quotes for verbatim code
    \usepackage{eurosym} % defines \euro
    \usepackage[mathletters]{ucs} % Extended unicode (utf-8) support
    \usepackage[utf8x]{inputenc} % Allow utf-8 characters in the tex document
    \usepackage{fancyvrb} % verbatim replacement that allows latex
    \usepackage{grffile} % extends the file name processing of package graphics 
                         % to support a larger range 
    % The hyperref package gives us a pdf with properly built
    % internal navigation ('pdf bookmarks' for the table of contents,
    % internal cross-reference links, web links for URLs, etc.)
    \usepackage{hyperref}
    \usepackage{longtable} % longtable support required by pandoc >1.10
    \usepackage{booktabs}  % table support for pandoc > 1.12.2
    \usepackage[inline]{enumitem} % IRkernel/repr support (it uses the enumerate* environment)
    \usepackage[normalem]{ulem} % ulem is needed to support strikethroughs (\sout)
                                % normalem makes italics be italics, not underlines
    

    
    
    % Colors for the hyperref package
    \definecolor{urlcolor}{rgb}{0,.145,.698}
    \definecolor{linkcolor}{rgb}{.71,0.21,0.01}
    \definecolor{citecolor}{rgb}{.12,.54,.11}

    % ANSI colors
    \definecolor{ansi-black}{HTML}{3E424D}
    \definecolor{ansi-black-intense}{HTML}{282C36}
    \definecolor{ansi-red}{HTML}{E75C58}
    \definecolor{ansi-red-intense}{HTML}{B22B31}
    \definecolor{ansi-green}{HTML}{00A250}
    \definecolor{ansi-green-intense}{HTML}{007427}
    \definecolor{ansi-yellow}{HTML}{DDB62B}
    \definecolor{ansi-yellow-intense}{HTML}{B27D12}
    \definecolor{ansi-blue}{HTML}{208FFB}
    \definecolor{ansi-blue-intense}{HTML}{0065CA}
    \definecolor{ansi-magenta}{HTML}{D160C4}
    \definecolor{ansi-magenta-intense}{HTML}{A03196}
    \definecolor{ansi-cyan}{HTML}{60C6C8}
    \definecolor{ansi-cyan-intense}{HTML}{258F8F}
    \definecolor{ansi-white}{HTML}{C5C1B4}
    \definecolor{ansi-white-intense}{HTML}{A1A6B2}

    % commands and environments needed by pandoc snippets
    % extracted from the output of `pandoc -s`
    \providecommand{\tightlist}{%
      \setlength{\itemsep}{0pt}\setlength{\parskip}{0pt}}
    \DefineVerbatimEnvironment{Highlighting}{Verbatim}{commandchars=\\\{\}}
    % Add ',fontsize=\small' for more characters per line
    \newenvironment{Shaded}{}{}
    \newcommand{\KeywordTok}[1]{\textcolor[rgb]{0.00,0.44,0.13}{\textbf{{#1}}}}
    \newcommand{\DataTypeTok}[1]{\textcolor[rgb]{0.56,0.13,0.00}{{#1}}}
    \newcommand{\DecValTok}[1]{\textcolor[rgb]{0.25,0.63,0.44}{{#1}}}
    \newcommand{\BaseNTok}[1]{\textcolor[rgb]{0.25,0.63,0.44}{{#1}}}
    \newcommand{\FloatTok}[1]{\textcolor[rgb]{0.25,0.63,0.44}{{#1}}}
    \newcommand{\CharTok}[1]{\textcolor[rgb]{0.25,0.44,0.63}{{#1}}}
    \newcommand{\StringTok}[1]{\textcolor[rgb]{0.25,0.44,0.63}{{#1}}}
    \newcommand{\CommentTok}[1]{\textcolor[rgb]{0.38,0.63,0.69}{\textit{{#1}}}}
    \newcommand{\OtherTok}[1]{\textcolor[rgb]{0.00,0.44,0.13}{{#1}}}
    \newcommand{\AlertTok}[1]{\textcolor[rgb]{1.00,0.00,0.00}{\textbf{{#1}}}}
    \newcommand{\FunctionTok}[1]{\textcolor[rgb]{0.02,0.16,0.49}{{#1}}}
    \newcommand{\RegionMarkerTok}[1]{{#1}}
    \newcommand{\ErrorTok}[1]{\textcolor[rgb]{1.00,0.00,0.00}{\textbf{{#1}}}}
    \newcommand{\NormalTok}[1]{{#1}}
    
    % Additional commands for more recent versions of Pandoc
    \newcommand{\ConstantTok}[1]{\textcolor[rgb]{0.53,0.00,0.00}{{#1}}}
    \newcommand{\SpecialCharTok}[1]{\textcolor[rgb]{0.25,0.44,0.63}{{#1}}}
    \newcommand{\VerbatimStringTok}[1]{\textcolor[rgb]{0.25,0.44,0.63}{{#1}}}
    \newcommand{\SpecialStringTok}[1]{\textcolor[rgb]{0.73,0.40,0.53}{{#1}}}
    \newcommand{\ImportTok}[1]{{#1}}
    \newcommand{\DocumentationTok}[1]{\textcolor[rgb]{0.73,0.13,0.13}{\textit{{#1}}}}
    \newcommand{\AnnotationTok}[1]{\textcolor[rgb]{0.38,0.63,0.69}{\textbf{\textit{{#1}}}}}
    \newcommand{\CommentVarTok}[1]{\textcolor[rgb]{0.38,0.63,0.69}{\textbf{\textit{{#1}}}}}
    \newcommand{\VariableTok}[1]{\textcolor[rgb]{0.10,0.09,0.49}{{#1}}}
    \newcommand{\ControlFlowTok}[1]{\textcolor[rgb]{0.00,0.44,0.13}{\textbf{{#1}}}}
    \newcommand{\OperatorTok}[1]{\textcolor[rgb]{0.40,0.40,0.40}{{#1}}}
    \newcommand{\BuiltInTok}[1]{{#1}}
    \newcommand{\ExtensionTok}[1]{{#1}}
    \newcommand{\PreprocessorTok}[1]{\textcolor[rgb]{0.74,0.48,0.00}{{#1}}}
    \newcommand{\AttributeTok}[1]{\textcolor[rgb]{0.49,0.56,0.16}{{#1}}}
    \newcommand{\InformationTok}[1]{\textcolor[rgb]{0.38,0.63,0.69}{\textbf{\textit{{#1}}}}}
    \newcommand{\WarningTok}[1]{\textcolor[rgb]{0.38,0.63,0.69}{\textbf{\textit{{#1}}}}}
    
    
    % Define a nice break command that doesn't care if a line doesn't already
    % exist.
    \def\br{\hspace*{\fill} \\* }
    % Math Jax compatability definitions
    \def\gt{>}
    \def\lt{<}
    % Document parameters
    \title{assignment03}
    
    
    

    % Pygments definitions
    
\makeatletter
\def\PY@reset{\let\PY@it=\relax \let\PY@bf=\relax%
    \let\PY@ul=\relax \let\PY@tc=\relax%
    \let\PY@bc=\relax \let\PY@ff=\relax}
\def\PY@tok#1{\csname PY@tok@#1\endcsname}
\def\PY@toks#1+{\ifx\relax#1\empty\else%
    \PY@tok{#1}\expandafter\PY@toks\fi}
\def\PY@do#1{\PY@bc{\PY@tc{\PY@ul{%
    \PY@it{\PY@bf{\PY@ff{#1}}}}}}}
\def\PY#1#2{\PY@reset\PY@toks#1+\relax+\PY@do{#2}}

\expandafter\def\csname PY@tok@w\endcsname{\def\PY@tc##1{\textcolor[rgb]{0.73,0.73,0.73}{##1}}}
\expandafter\def\csname PY@tok@c\endcsname{\let\PY@it=\textit\def\PY@tc##1{\textcolor[rgb]{0.25,0.50,0.50}{##1}}}
\expandafter\def\csname PY@tok@cp\endcsname{\def\PY@tc##1{\textcolor[rgb]{0.74,0.48,0.00}{##1}}}
\expandafter\def\csname PY@tok@k\endcsname{\let\PY@bf=\textbf\def\PY@tc##1{\textcolor[rgb]{0.00,0.50,0.00}{##1}}}
\expandafter\def\csname PY@tok@kp\endcsname{\def\PY@tc##1{\textcolor[rgb]{0.00,0.50,0.00}{##1}}}
\expandafter\def\csname PY@tok@kt\endcsname{\def\PY@tc##1{\textcolor[rgb]{0.69,0.00,0.25}{##1}}}
\expandafter\def\csname PY@tok@o\endcsname{\def\PY@tc##1{\textcolor[rgb]{0.40,0.40,0.40}{##1}}}
\expandafter\def\csname PY@tok@ow\endcsname{\let\PY@bf=\textbf\def\PY@tc##1{\textcolor[rgb]{0.67,0.13,1.00}{##1}}}
\expandafter\def\csname PY@tok@nb\endcsname{\def\PY@tc##1{\textcolor[rgb]{0.00,0.50,0.00}{##1}}}
\expandafter\def\csname PY@tok@nf\endcsname{\def\PY@tc##1{\textcolor[rgb]{0.00,0.00,1.00}{##1}}}
\expandafter\def\csname PY@tok@nc\endcsname{\let\PY@bf=\textbf\def\PY@tc##1{\textcolor[rgb]{0.00,0.00,1.00}{##1}}}
\expandafter\def\csname PY@tok@nn\endcsname{\let\PY@bf=\textbf\def\PY@tc##1{\textcolor[rgb]{0.00,0.00,1.00}{##1}}}
\expandafter\def\csname PY@tok@ne\endcsname{\let\PY@bf=\textbf\def\PY@tc##1{\textcolor[rgb]{0.82,0.25,0.23}{##1}}}
\expandafter\def\csname PY@tok@nv\endcsname{\def\PY@tc##1{\textcolor[rgb]{0.10,0.09,0.49}{##1}}}
\expandafter\def\csname PY@tok@no\endcsname{\def\PY@tc##1{\textcolor[rgb]{0.53,0.00,0.00}{##1}}}
\expandafter\def\csname PY@tok@nl\endcsname{\def\PY@tc##1{\textcolor[rgb]{0.63,0.63,0.00}{##1}}}
\expandafter\def\csname PY@tok@ni\endcsname{\let\PY@bf=\textbf\def\PY@tc##1{\textcolor[rgb]{0.60,0.60,0.60}{##1}}}
\expandafter\def\csname PY@tok@na\endcsname{\def\PY@tc##1{\textcolor[rgb]{0.49,0.56,0.16}{##1}}}
\expandafter\def\csname PY@tok@nt\endcsname{\let\PY@bf=\textbf\def\PY@tc##1{\textcolor[rgb]{0.00,0.50,0.00}{##1}}}
\expandafter\def\csname PY@tok@nd\endcsname{\def\PY@tc##1{\textcolor[rgb]{0.67,0.13,1.00}{##1}}}
\expandafter\def\csname PY@tok@s\endcsname{\def\PY@tc##1{\textcolor[rgb]{0.73,0.13,0.13}{##1}}}
\expandafter\def\csname PY@tok@sd\endcsname{\let\PY@it=\textit\def\PY@tc##1{\textcolor[rgb]{0.73,0.13,0.13}{##1}}}
\expandafter\def\csname PY@tok@si\endcsname{\let\PY@bf=\textbf\def\PY@tc##1{\textcolor[rgb]{0.73,0.40,0.53}{##1}}}
\expandafter\def\csname PY@tok@se\endcsname{\let\PY@bf=\textbf\def\PY@tc##1{\textcolor[rgb]{0.73,0.40,0.13}{##1}}}
\expandafter\def\csname PY@tok@sr\endcsname{\def\PY@tc##1{\textcolor[rgb]{0.73,0.40,0.53}{##1}}}
\expandafter\def\csname PY@tok@ss\endcsname{\def\PY@tc##1{\textcolor[rgb]{0.10,0.09,0.49}{##1}}}
\expandafter\def\csname PY@tok@sx\endcsname{\def\PY@tc##1{\textcolor[rgb]{0.00,0.50,0.00}{##1}}}
\expandafter\def\csname PY@tok@m\endcsname{\def\PY@tc##1{\textcolor[rgb]{0.40,0.40,0.40}{##1}}}
\expandafter\def\csname PY@tok@gh\endcsname{\let\PY@bf=\textbf\def\PY@tc##1{\textcolor[rgb]{0.00,0.00,0.50}{##1}}}
\expandafter\def\csname PY@tok@gu\endcsname{\let\PY@bf=\textbf\def\PY@tc##1{\textcolor[rgb]{0.50,0.00,0.50}{##1}}}
\expandafter\def\csname PY@tok@gd\endcsname{\def\PY@tc##1{\textcolor[rgb]{0.63,0.00,0.00}{##1}}}
\expandafter\def\csname PY@tok@gi\endcsname{\def\PY@tc##1{\textcolor[rgb]{0.00,0.63,0.00}{##1}}}
\expandafter\def\csname PY@tok@gr\endcsname{\def\PY@tc##1{\textcolor[rgb]{1.00,0.00,0.00}{##1}}}
\expandafter\def\csname PY@tok@ge\endcsname{\let\PY@it=\textit}
\expandafter\def\csname PY@tok@gs\endcsname{\let\PY@bf=\textbf}
\expandafter\def\csname PY@tok@gp\endcsname{\let\PY@bf=\textbf\def\PY@tc##1{\textcolor[rgb]{0.00,0.00,0.50}{##1}}}
\expandafter\def\csname PY@tok@go\endcsname{\def\PY@tc##1{\textcolor[rgb]{0.53,0.53,0.53}{##1}}}
\expandafter\def\csname PY@tok@gt\endcsname{\def\PY@tc##1{\textcolor[rgb]{0.00,0.27,0.87}{##1}}}
\expandafter\def\csname PY@tok@err\endcsname{\def\PY@bc##1{\setlength{\fboxsep}{0pt}\fcolorbox[rgb]{1.00,0.00,0.00}{1,1,1}{\strut ##1}}}
\expandafter\def\csname PY@tok@kc\endcsname{\let\PY@bf=\textbf\def\PY@tc##1{\textcolor[rgb]{0.00,0.50,0.00}{##1}}}
\expandafter\def\csname PY@tok@kd\endcsname{\let\PY@bf=\textbf\def\PY@tc##1{\textcolor[rgb]{0.00,0.50,0.00}{##1}}}
\expandafter\def\csname PY@tok@kn\endcsname{\let\PY@bf=\textbf\def\PY@tc##1{\textcolor[rgb]{0.00,0.50,0.00}{##1}}}
\expandafter\def\csname PY@tok@kr\endcsname{\let\PY@bf=\textbf\def\PY@tc##1{\textcolor[rgb]{0.00,0.50,0.00}{##1}}}
\expandafter\def\csname PY@tok@bp\endcsname{\def\PY@tc##1{\textcolor[rgb]{0.00,0.50,0.00}{##1}}}
\expandafter\def\csname PY@tok@fm\endcsname{\def\PY@tc##1{\textcolor[rgb]{0.00,0.00,1.00}{##1}}}
\expandafter\def\csname PY@tok@vc\endcsname{\def\PY@tc##1{\textcolor[rgb]{0.10,0.09,0.49}{##1}}}
\expandafter\def\csname PY@tok@vg\endcsname{\def\PY@tc##1{\textcolor[rgb]{0.10,0.09,0.49}{##1}}}
\expandafter\def\csname PY@tok@vi\endcsname{\def\PY@tc##1{\textcolor[rgb]{0.10,0.09,0.49}{##1}}}
\expandafter\def\csname PY@tok@vm\endcsname{\def\PY@tc##1{\textcolor[rgb]{0.10,0.09,0.49}{##1}}}
\expandafter\def\csname PY@tok@sa\endcsname{\def\PY@tc##1{\textcolor[rgb]{0.73,0.13,0.13}{##1}}}
\expandafter\def\csname PY@tok@sb\endcsname{\def\PY@tc##1{\textcolor[rgb]{0.73,0.13,0.13}{##1}}}
\expandafter\def\csname PY@tok@sc\endcsname{\def\PY@tc##1{\textcolor[rgb]{0.73,0.13,0.13}{##1}}}
\expandafter\def\csname PY@tok@dl\endcsname{\def\PY@tc##1{\textcolor[rgb]{0.73,0.13,0.13}{##1}}}
\expandafter\def\csname PY@tok@s2\endcsname{\def\PY@tc##1{\textcolor[rgb]{0.73,0.13,0.13}{##1}}}
\expandafter\def\csname PY@tok@sh\endcsname{\def\PY@tc##1{\textcolor[rgb]{0.73,0.13,0.13}{##1}}}
\expandafter\def\csname PY@tok@s1\endcsname{\def\PY@tc##1{\textcolor[rgb]{0.73,0.13,0.13}{##1}}}
\expandafter\def\csname PY@tok@mb\endcsname{\def\PY@tc##1{\textcolor[rgb]{0.40,0.40,0.40}{##1}}}
\expandafter\def\csname PY@tok@mf\endcsname{\def\PY@tc##1{\textcolor[rgb]{0.40,0.40,0.40}{##1}}}
\expandafter\def\csname PY@tok@mh\endcsname{\def\PY@tc##1{\textcolor[rgb]{0.40,0.40,0.40}{##1}}}
\expandafter\def\csname PY@tok@mi\endcsname{\def\PY@tc##1{\textcolor[rgb]{0.40,0.40,0.40}{##1}}}
\expandafter\def\csname PY@tok@il\endcsname{\def\PY@tc##1{\textcolor[rgb]{0.40,0.40,0.40}{##1}}}
\expandafter\def\csname PY@tok@mo\endcsname{\def\PY@tc##1{\textcolor[rgb]{0.40,0.40,0.40}{##1}}}
\expandafter\def\csname PY@tok@ch\endcsname{\let\PY@it=\textit\def\PY@tc##1{\textcolor[rgb]{0.25,0.50,0.50}{##1}}}
\expandafter\def\csname PY@tok@cm\endcsname{\let\PY@it=\textit\def\PY@tc##1{\textcolor[rgb]{0.25,0.50,0.50}{##1}}}
\expandafter\def\csname PY@tok@cpf\endcsname{\let\PY@it=\textit\def\PY@tc##1{\textcolor[rgb]{0.25,0.50,0.50}{##1}}}
\expandafter\def\csname PY@tok@c1\endcsname{\let\PY@it=\textit\def\PY@tc##1{\textcolor[rgb]{0.25,0.50,0.50}{##1}}}
\expandafter\def\csname PY@tok@cs\endcsname{\let\PY@it=\textit\def\PY@tc##1{\textcolor[rgb]{0.25,0.50,0.50}{##1}}}

\def\PYZbs{\char`\\}
\def\PYZus{\char`\_}
\def\PYZob{\char`\{}
\def\PYZcb{\char`\}}
\def\PYZca{\char`\^}
\def\PYZam{\char`\&}
\def\PYZlt{\char`\<}
\def\PYZgt{\char`\>}
\def\PYZsh{\char`\#}
\def\PYZpc{\char`\%}
\def\PYZdl{\char`\$}
\def\PYZhy{\char`\-}
\def\PYZsq{\char`\'}
\def\PYZdq{\char`\"}
\def\PYZti{\char`\~}
% for compatibility with earlier versions
\def\PYZat{@}
\def\PYZlb{[}
\def\PYZrb{]}
\makeatother


    % Exact colors from NB
    \definecolor{incolor}{rgb}{0.0, 0.0, 0.5}
    \definecolor{outcolor}{rgb}{0.545, 0.0, 0.0}



    
    % Prevent overflowing lines due to hard-to-break entities
    \sloppy 
    % Setup hyperref package
    \hypersetup{
      breaklinks=true,  % so long urls are correctly broken across lines
      colorlinks=true,
      urlcolor=urlcolor,
      linkcolor=linkcolor,
      citecolor=citecolor,
      }
    % Slightly bigger margins than the latex defaults
    
    \geometry{verbose,tmargin=1in,bmargin=1in,lmargin=1in,rmargin=1in}
    
    

    \begin{document}
    
    
    \maketitle
    
    

    
    \section{K-means Algorithm}\label{k-means-algorithm}

\textbf{Name}: ZHU GUANGYU\\
\textbf{Student ID}: 20165953\\
\textbf{Github Repo}:
\href{https://github.com/z1ggy-o/cv_assignment/tree/master/assignment03}{assignment03}

\begin{center}\rule{0.5\linewidth}{\linethickness}\end{center}

\subsection{Clustering}\label{clustering}

the goal of \emph{clustering} is to group or partition the vectors into
\emph{k} groups or clusters, with the vectors in each group close to
each other.

the best clustering: - can find the best slustering, if the
representatives are fixed; - can find the vest representatives, if the
clustering is fixed.

We use a single number to judge a choice of clustering, along with a
choice of the group. We define:

\[
J^{clust}=(||x_{1}-z_{c_{1}}||^{2}+ \cdots +||x_{N}-z_{c_{N}}||^{2})/N
\]

Here \(x_{N}\) is vector, \(z_{c_{N}}\) is correspond representatives.
We call this value \emph{energy} or \emph{cost}.

\subsubsection{When representatives
fixed}\label{when-representatives-fixed}

We assign each date vector \(x_{i}\) to its nearest representatives.
Since the representatives are fixed, we actually re-grouped vectors into
different partitions. We have:
\[ ||x_{i}-z_{c_{i}}|| = \min_{j=1,\cdots\,k}||x_{i}-z_{c_{i}}|| \]

This gives us the minimum \(J^{clust}\).

\subsubsection{When group assignment
fixed}\label{when-group-assignment-fixed}

This means the element vectors of each group are fixed. We need to find
the group representatives to minimize our cost \(J^{clust}\).

Simply, choose the average of the vectros in its group:
\[ z_{j} = (1/|G_{j}|)\sum_{i\in G_{j}}x_{i}\]

since this makes the sum of distance between points and its
representative minimum.

\subsection{\texorpdfstring{\emph{k}-means
Algorithm}{k-means Algorithm}}\label{k-means-algorithm-1}

Previous two methods can help us get the best clustering. But the two
methods are depend on each other. To solve this problem, we can use
\emph{k-means algorithm}.

\emph{k-means algorithm}'s idea is simple. We repeatedly alternate
between updating the group assignments, then updating the
representatives. In each iteration we get a better \(J^{clust}\) until
the step does not change the choice.

Have to be aware of is k-means algorithm \textbf{cannot} guarantee that
the partition it finds minimizes \(J^{clust}\). Commonly, we run it
several times with different initial representatives, and choose the one
with the smallest cost.

There is another problem is to determin the optimal number of clusters
(here is the \emph{k}).\\
If you have given number of clusters, that's fine. If you don't, there
are few methods can help us: -
\href{https://en.wikipedia.org/wiki/Elbow_method_\%28clustering\%29}{Elbow
method} -
\href{https://en.wikipedia.org/wiki/Silhouette_\%28clustering\%29}{The
silhouette method} -
\href{http://web.stanford.edu/~hastie/Papers/gap.pdf}{Gap statistic}

\begin{center}\rule{0.5\linewidth}{\linethickness}\end{center}

    \subsection{Implementation}\label{implementation}

Now, let's try implement the algorithm.

\begin{center}\rule{0.5\linewidth}{\linethickness}\end{center}

First, import some packages.\\
- \texttt{numpy} for scientific computing - \texttt{matplotlib} for
visualization - \texttt{math} is the python build in math packages.

    \begin{Verbatim}[commandchars=\\\{\}]
{\color{incolor}In [{\color{incolor}7}]:} \PY{k+kn}{import} \PY{n+nn}{numpy} \PY{k}{as} \PY{n+nn}{np}
        \PY{k+kn}{from} \PY{n+nn}{matplotlib} \PY{k}{import} \PY{n}{pyplot} \PY{k}{as} \PY{n}{plt}
        \PY{k+kn}{from} \PY{n+nn}{matplotlib} \PY{k}{import} \PY{n}{cm}
        \PY{k+kn}{import} \PY{n+nn}{math}
\end{Verbatim}


    \begin{center}\rule{0.5\linewidth}{\linethickness}\end{center}

Create a class call \texttt{KMeans} to combine the functions we need
together.

Here we define \texttt{\_\_init\_\_} function to receive 3 arguments: -
\texttt{k} is number of clusters we want. - \texttt{num\_points} is the
number of random points we want to generate. - \texttt{num\_dims} is the
dimention of the data. Default value is 2 (point)

    \begin{Verbatim}[commandchars=\\\{\}]
{\color{incolor}In [{\color{incolor}8}]:} \PY{k}{class} \PY{n+nc}{KMeans}\PY{p}{:}
            \PY{l+s+sd}{\PYZdq{}\PYZdq{}\PYZdq{}k\PYZhy{}means algorithm}
        
        \PY{l+s+sd}{    generate random 2d points,}
        \PY{l+s+sd}{    group these points into k clusters}
        \PY{l+s+sd}{    \PYZdq{}\PYZdq{}\PYZdq{}}
        
            \PY{k}{def} \PY{n+nf}{\PYZus{}\PYZus{}init\PYZus{}\PYZus{}}\PY{p}{(}\PY{n+nb+bp}{self}\PY{p}{,} \PY{n}{k}\PY{o}{=}\PY{l+m+mi}{3}\PY{p}{,} \PY{n}{num\PYZus{}points}\PY{o}{=}\PY{l+m+mi}{100}\PY{p}{,} \PY{n}{num\PYZus{}dims}\PY{o}{=}\PY{l+m+mi}{2}\PY{p}{)}\PY{p}{:}
                \PY{l+s+sd}{\PYZdq{}\PYZdq{}\PYZdq{}Constructer}
        \PY{l+s+sd}{        }
        \PY{l+s+sd}{        Parameters:}
        \PY{l+s+sd}{            k(int): number of clusters we want}
        \PY{l+s+sd}{            num\PYZus{}points: number of random points we want to generate}
        \PY{l+s+sd}{            num\PYZus{}dims: the dimention of the data (for point, is 2)}
        \PY{l+s+sd}{        \PYZdq{}\PYZdq{}\PYZdq{}}
                
                \PY{n+nb+bp}{self}\PY{o}{.}\PY{n}{k} \PY{o}{=} \PY{n}{k}
                \PY{n+nb+bp}{self}\PY{o}{.}\PY{n}{num\PYZus{}points} \PY{o}{=} \PY{n}{num\PYZus{}points}
                \PY{n+nb+bp}{self}\PY{o}{.}\PY{n}{num\PYZus{}dims} \PY{o}{=} \PY{n}{num\PYZus{}dims}
                \PY{n+nb+bp}{self}\PY{o}{.}\PY{n}{energy\PYZus{}history} \PY{o}{=} \PY{p}{[}\PY{p}{]}
                \PY{n+nb+bp}{self}\PY{o}{.}\PY{n}{points} \PY{o}{=} \PY{p}{[}\PY{p}{]}
                \PY{n+nb+bp}{self}\PY{o}{.}\PY{n}{centroids} \PY{o}{=} \PY{p}{[}\PY{p}{]}
                \PY{n+nb+bp}{self}\PY{o}{.}\PY{n}{colors} \PY{o}{=} \PY{n+nb}{list}\PY{p}{(}\PY{n}{cm}\PY{o}{.}\PY{n}{rainbow}\PY{p}{(}\PY{n}{np}\PY{o}{.}\PY{n}{linspace}\PY{p}{(}\PY{l+m+mi}{0}\PY{p}{,} \PY{l+m+mi}{1}\PY{p}{,} \PY{n}{k}\PY{p}{)}\PY{p}{)}\PY{p}{)}
                \PY{n+nb+bp}{self}\PY{o}{.}\PY{n}{clusters} \PY{o}{=} \PY{p}{[}\PY{p}{]}
                \PY{k}{for} \PY{n}{\PYZus{}} \PY{o+ow}{in} \PY{n+nb}{range}\PY{p}{(}\PY{n}{k}\PY{p}{)}\PY{p}{:}
                    \PY{n+nb+bp}{self}\PY{o}{.}\PY{n}{clusters}\PY{o}{.}\PY{n}{append}\PY{p}{(}\PY{p}{[}\PY{p}{]}\PY{p}{)}
\end{Verbatim}


    \begin{center}\rule{0.5\linewidth}{\linethickness}\end{center}

Define some getter function at here.\\
Most of them are for visualization.

\begin{quote}
Because the code be seperated into different cells, I need to add some
extra codes to bind them together, just after the
\texttt{\#\ for\ juypter\ notebook} command, You do not need them when
run this code at local.
\end{quote}

    \begin{Verbatim}[commandchars=\\\{\}]
{\color{incolor}In [{\color{incolor}29}]:}     \PY{k}{def} \PY{n+nf}{getNumOfClusters}\PY{p}{(}\PY{n+nb+bp}{self}\PY{p}{)}\PY{p}{:}
                 \PY{k}{return} \PY{n+nb+bp}{self}\PY{o}{.}\PY{n}{k}
         
             \PY{k}{def} \PY{n+nf}{getColors}\PY{p}{(}\PY{n+nb+bp}{self}\PY{p}{)}\PY{p}{:}
                 \PY{k}{return} \PY{n+nb+bp}{self}\PY{o}{.}\PY{n}{colors}
         
             \PY{k}{def} \PY{n+nf}{getEnergyHistory}\PY{p}{(}\PY{n+nb+bp}{self}\PY{p}{)}\PY{p}{:}
                 \PY{k}{return} \PY{n+nb+bp}{self}\PY{o}{.}\PY{n}{energy\PYZus{}history}
         
             \PY{k}{def} \PY{n+nf}{getPoints}\PY{p}{(}\PY{n+nb+bp}{self}\PY{p}{)}\PY{p}{:}
                 \PY{k}{return} \PY{n+nb+bp}{self}\PY{o}{.}\PY{n}{points}
         
             \PY{k}{def} \PY{n+nf}{getClusters}\PY{p}{(}\PY{n+nb+bp}{self}\PY{p}{)}\PY{p}{:}
                 \PY{k}{return} \PY{n+nb+bp}{self}\PY{o}{.}\PY{n}{clusters}
         
             \PY{k}{def} \PY{n+nf}{getCentroids}\PY{p}{(}\PY{n+nb+bp}{self}\PY{p}{)}\PY{p}{:}
                 \PY{k}{return} \PY{n+nb+bp}{self}\PY{o}{.}\PY{n}{centroids}
             
         \PY{c+c1}{\PYZsh{} for jupyter notebook}
         \PY{n}{KMeans}\PY{o}{.}\PY{n}{getNumOfClusters} \PY{o}{=} \PY{n}{getNumOfClusters}
         \PY{n}{KMeans}\PY{o}{.}\PY{n}{getColors} \PY{o}{=} \PY{n}{getColors}
         \PY{n}{KMeans}\PY{o}{.}\PY{n}{getEnergyHistory} \PY{o}{=} \PY{n}{getEnergyHistory}
         \PY{n}{KMeans}\PY{o}{.}\PY{n}{getPoints} \PY{o}{=} \PY{n}{getPoints}
         \PY{n}{KMeans}\PY{o}{.}\PY{n}{getClusters} \PY{o}{=} \PY{n}{getClusters}
         \PY{n}{KMeans}\PY{o}{.}\PY{n}{getCentroids} \PY{o}{=} \PY{n}{getCentroids}
\end{Verbatim}


    \begin{center}\rule{0.5\linewidth}{\linethickness}\end{center}

Define the trigger function.

There are different way to determine when to stop the iteration.\\
Here we run the algorithm until the energy not change.

In practice, often stops the algorithm earlier, as soon as the
improvement becomes very small.

Here we seperate the trigger function and the cluster function to make
the future modification easier.

    \begin{Verbatim}[commandchars=\\\{\}]
{\color{incolor}In [{\color{incolor}10}]:}     \PY{k}{def} \PY{n+nf}{run}\PY{p}{(}\PY{n+nb+bp}{self}\PY{p}{)}\PY{p}{:}
                 \PY{l+s+sd}{\PYZdq{}\PYZdq{}\PYZdq{}Run algorithm}
         
         \PY{l+s+sd}{        Repeatly assign labels to points and compute new centroids.}
         \PY{l+s+sd}{        Iterate until the energy of the result not change.}
         \PY{l+s+sd}{        \PYZdq{}\PYZdq{}\PYZdq{}}
         
                 \PY{n}{energy\PYZus{}prev} \PY{o}{=} \PY{l+m+mi}{0}
                 \PY{n}{energy\PYZus{}thisTurn} \PY{o}{=} \PY{n+nb+bp}{self}\PY{o}{.}\PY{n}{\PYZus{}computeEnergy}\PY{p}{(}\PY{p}{)}  \PY{c+c1}{\PYZsh{} initialization result}
                 \PY{k}{while}\PY{p}{(}\PY{n}{energy\PYZus{}thisTurn} \PY{o}{!=} \PY{n}{energy\PYZus{}prev}\PY{p}{)}\PY{p}{:}
                     \PY{n}{energy\PYZus{}prev} \PY{o}{=} \PY{n}{energy\PYZus{}thisTurn}
                     \PY{n}{energy\PYZus{}thisTurn} \PY{o}{=} \PY{n+nb+bp}{self}\PY{o}{.}\PY{n}{\PYZus{}clustering}\PY{p}{(}\PY{p}{)}
                     
         \PY{c+c1}{\PYZsh{} for jupyter notebook}
         \PY{n}{KMeans}\PY{o}{.}\PY{n}{run} \PY{o}{=} \PY{n}{run}
\end{Verbatim}


    About \textbf{energy} computation, like we said previously, use
function: \[
J^{clust}=(||x_{1}-z_{c_{1}}||^{2}+ \cdots +||x_{N}-z_{c_{N}}||^{2})/N
\]

    \begin{Verbatim}[commandchars=\\\{\}]
{\color{incolor}In [{\color{incolor}11}]:}     \PY{k}{def} \PY{n+nf}{\PYZus{}computeEnergy}\PY{p}{(}\PY{n+nb+bp}{self}\PY{p}{)}\PY{p}{:}
                 \PY{l+s+sd}{\PYZdq{}\PYZdq{}\PYZdq{} Compute the cost of the clustering result}
         
         \PY{l+s+sd}{        Return:}
         \PY{l+s+sd}{            energy(float): the energy of this clustering.}
         \PY{l+s+sd}{        \PYZdq{}\PYZdq{}\PYZdq{}}
         
                 \PY{n}{energy} \PY{o}{=} \PY{l+m+mi}{0}
                 \PY{k}{for} \PY{n}{i} \PY{o+ow}{in} \PY{n+nb}{range}\PY{p}{(}\PY{n+nb}{len}\PY{p}{(}\PY{n+nb+bp}{self}\PY{o}{.}\PY{n}{centroids}\PY{p}{)}\PY{p}{)}\PY{p}{:}
                     \PY{n}{centroid} \PY{o}{=} \PY{n+nb+bp}{self}\PY{o}{.}\PY{n}{centroids}\PY{p}{[}\PY{n}{i}\PY{p}{]}
                     \PY{n}{part\PYZus{}energy} \PY{o}{=} \PY{l+m+mi}{0}
                     \PY{k}{for} \PY{n}{point} \PY{o+ow}{in} \PY{n+nb+bp}{self}\PY{o}{.}\PY{n}{clusters}\PY{p}{[}\PY{n}{i}\PY{p}{]}\PY{p}{:}
                         \PY{n}{part\PYZus{}energy} \PY{o}{+}\PY{o}{=} \PY{p}{(}
                             \PY{n}{math}\PY{o}{.}\PY{n}{pow}\PY{p}{(}\PY{n+nb+bp}{self}\PY{o}{.}\PY{n}{\PYZus{}computeDistance}\PY{p}{(}\PY{n}{point}\PY{p}{,} \PY{n}{centroid}\PY{p}{)}\PY{p}{,} \PY{l+m+mi}{2}\PY{p}{)}
                         \PY{p}{)}
                     \PY{n}{energy} \PY{o}{+}\PY{o}{=} \PY{n}{part\PYZus{}energy}
                 \PY{n}{energy} \PY{o}{=} \PY{n}{energy} \PY{o}{/} \PY{n+nb}{len}\PY{p}{(}\PY{n+nb+bp}{self}\PY{o}{.}\PY{n}{points}\PY{p}{)}
                 \PY{n+nb+bp}{self}\PY{o}{.}\PY{n}{energy\PYZus{}history}\PY{o}{.}\PY{n}{append}\PY{p}{(}\PY{n}{energy}\PY{p}{)}
         
                 \PY{k}{return} \PY{n}{energy}
             
         \PY{c+c1}{\PYZsh{} for jupyter notebook}
         \PY{n}{KMeans}\PY{o}{.}\PY{n}{\PYZus{}computeEnergy} \PY{o}{=} \PY{n}{\PYZus{}computeEnergy}
\end{Verbatim}


    \begin{center}\rule{0.5\linewidth}{\linethickness}\end{center}

Define the initialization function.

This function will: 1. Generate random points 2. Assign label to each
points 3. Randomly move each cluster 4. Compute centroids

We seperate each step to different function.

    \begin{Verbatim}[commandchars=\\\{\}]
{\color{incolor}In [{\color{incolor}12}]:}     \PY{k}{def} \PY{n+nf}{generatePointCluster}\PY{p}{(}\PY{n+nb+bp}{self}\PY{p}{)}\PY{p}{:}
                     \PY{l+s+sd}{\PYZdq{}\PYZdq{}\PYZdq{}Generate random points}
         
         \PY{l+s+sd}{            1. Generate random points}
         \PY{l+s+sd}{            2. Randomly assign label to each points}
         \PY{l+s+sd}{            3. Disperse points by cluster one more time}
         \PY{l+s+sd}{            4. Compute the initial centroids}
         \PY{l+s+sd}{            \PYZdq{}\PYZdq{}\PYZdq{}}
         
                     \PY{n+nb+bp}{self}\PY{o}{.}\PY{n}{\PYZus{}generatePoints}\PY{p}{(}\PY{p}{)}
                     \PY{n+nb+bp}{self}\PY{o}{.}\PY{n}{\PYZus{}initialLabel}\PY{p}{(}\PY{p}{)}
                     \PY{n+nb+bp}{self}\PY{o}{.}\PY{n}{\PYZus{}dispersePoints}\PY{p}{(}\PY{p}{)}
                     \PY{n+nb+bp}{self}\PY{o}{.}\PY{n}{\PYZus{}computeCentroid}\PY{p}{(}\PY{p}{)}
                     
         \PY{c+c1}{\PYZsh{} for jupyter notebook}
         \PY{n}{KMeans}\PY{o}{.}\PY{n}{generatePointCluster} \PY{o}{=} \PY{n}{generatePointCluster}
\end{Verbatim}


    Depends on the \texttt{num\_points} and \texttt{num\_dims}, we use
\texttt{numpy}'s \texttt{random} function to generate the random data
set, then put them into a \texttt{points} list.

    \begin{Verbatim}[commandchars=\\\{\}]
{\color{incolor}In [{\color{incolor}13}]:}     \PY{k}{def} \PY{n+nf}{\PYZus{}generatePoints}\PY{p}{(}\PY{n+nb+bp}{self}\PY{p}{)}\PY{p}{:}
                     \PY{n}{randoms} \PY{o}{=} \PY{n}{np}\PY{o}{.}\PY{n}{random}\PY{o}{.}\PY{n}{rand}\PY{p}{(}\PY{n+nb+bp}{self}\PY{o}{.}\PY{n}{num\PYZus{}points}\PY{p}{,} \PY{n+nb+bp}{self}\PY{o}{.}\PY{n}{num\PYZus{}dims}\PY{p}{)}
                     \PY{k}{for} \PY{n}{x}\PY{p}{,} \PY{n}{y} \PY{o+ow}{in} \PY{n}{randoms}\PY{p}{:}
                         \PY{n}{point} \PY{o}{=} \PY{p}{[}\PY{n+nb}{int}\PY{p}{(}\PY{n}{x}\PY{o}{*}\PY{l+m+mi}{100}\PY{p}{)}\PY{p}{,} \PY{n+nb}{int}\PY{p}{(}\PY{n}{y}\PY{o}{*}\PY{l+m+mi}{100}\PY{p}{)}\PY{p}{]}
                         \PY{n+nb+bp}{self}\PY{o}{.}\PY{n}{points}\PY{o}{.}\PY{n}{append}\PY{p}{(}\PY{n}{point}\PY{p}{)}
                         
         \PY{c+c1}{\PYZsh{} for jupyter notebook}
         \PY{n}{KMeans}\PY{o}{.}\PY{n}{\PYZus{}generatePoints} \PY{o}{=} \PY{n}{\PYZus{}generatePoints}
\end{Verbatim}


    For intialing labels, we just simply do it in order.

    \begin{Verbatim}[commandchars=\\\{\}]
{\color{incolor}In [{\color{incolor}14}]:}     \PY{k}{def} \PY{n+nf}{\PYZus{}initialLabel}\PY{p}{(}\PY{n+nb+bp}{self}\PY{p}{)}\PY{p}{:}
                     \PY{k}{for} \PY{n}{i} \PY{o+ow}{in} \PY{n+nb}{range}\PY{p}{(}\PY{n+nb}{len}\PY{p}{(}\PY{n+nb+bp}{self}\PY{o}{.}\PY{n}{points}\PY{p}{)}\PY{p}{)}\PY{p}{:}
                         \PY{n}{index} \PY{o}{=} \PY{n}{i} \PY{o}{\PYZpc{}} \PY{n+nb+bp}{self}\PY{o}{.}\PY{n}{k}
                         \PY{n+nb+bp}{self}\PY{o}{.}\PY{n}{clusters}\PY{p}{[}\PY{n}{index}\PY{p}{]}\PY{o}{.}\PY{n}{append}\PY{p}{(}\PY{n+nb+bp}{self}\PY{o}{.}\PY{n}{points}\PY{p}{[}\PY{n}{i}\PY{p}{]}\PY{p}{)}
                         
         \PY{c+c1}{\PYZsh{} for jupyter notebook}
         \PY{n}{KMeans}\PY{o}{.}\PY{n}{\PYZus{}initialLabel} \PY{o}{=} \PY{n}{\PYZus{}initialLabel}
\end{Verbatim}


    To make data set looks clusters, we move each cluster of a random
offset.

    \begin{Verbatim}[commandchars=\\\{\}]
{\color{incolor}In [{\color{incolor}15}]:}     \PY{k}{def} \PY{n+nf}{\PYZus{}dispersePoints}\PY{p}{(}\PY{n+nb+bp}{self}\PY{p}{)}\PY{p}{:}
                     \PY{c+c1}{\PYZsh{} move each cluster\PYZsq{}s point with random offset}
                     \PY{k}{for} \PY{n}{i} \PY{o+ow}{in} \PY{n+nb}{range}\PY{p}{(}\PY{n+nb+bp}{self}\PY{o}{.}\PY{n}{k}\PY{p}{)}\PY{p}{:}
                         \PY{n}{x\PYZus{}off} \PY{o}{=} \PY{n}{np}\PY{o}{.}\PY{n}{random}\PY{o}{.}\PY{n}{randint}\PY{p}{(}\PY{o}{\PYZhy{}}\PY{l+m+mi}{50}\PY{p}{,} \PY{l+m+mi}{50}\PY{p}{)}
                         \PY{n}{y\PYZus{}off} \PY{o}{=} \PY{n}{np}\PY{o}{.}\PY{n}{random}\PY{o}{.}\PY{n}{randint}\PY{p}{(}\PY{o}{\PYZhy{}}\PY{l+m+mi}{50}\PY{p}{,} \PY{l+m+mi}{50}\PY{p}{)}
                         \PY{n}{points\PYZus{}moved} \PY{o}{=} \PY{p}{[}\PY{p}{]}
                         \PY{k}{for} \PY{n}{x}\PY{p}{,} \PY{n}{y} \PY{o+ow}{in} \PY{n+nb+bp}{self}\PY{o}{.}\PY{n}{clusters}\PY{p}{[}\PY{n}{i}\PY{p}{]}\PY{p}{:}
                             \PY{n}{points\PYZus{}moved}\PY{o}{.}\PY{n}{append}\PY{p}{(}\PY{p}{[}\PY{n}{x}\PY{o}{+}\PY{n}{x\PYZus{}off}\PY{p}{,} \PY{n}{y}\PY{o}{+}\PY{n}{y\PYZus{}off}\PY{p}{]}\PY{p}{)}
                         \PY{n+nb+bp}{self}\PY{o}{.}\PY{n}{clusters}\PY{p}{[}\PY{n}{i}\PY{p}{]} \PY{o}{=} \PY{n}{points\PYZus{}moved}
         
                     \PY{c+c1}{\PYZsh{} update result back to points list}
                     \PY{n}{new\PYZus{}points} \PY{o}{=} \PY{p}{[}\PY{p}{]}
                     \PY{k}{for} \PY{n}{i} \PY{o+ow}{in} \PY{n+nb}{range}\PY{p}{(}\PY{n+nb+bp}{self}\PY{o}{.}\PY{n}{k}\PY{p}{)}\PY{p}{:}
                         \PY{k}{for} \PY{n}{point} \PY{o+ow}{in} \PY{n+nb+bp}{self}\PY{o}{.}\PY{n}{clusters}\PY{p}{[}\PY{n}{i}\PY{p}{]}\PY{p}{:}
                             \PY{n}{new\PYZus{}points}\PY{o}{.}\PY{n}{append}\PY{p}{(}\PY{n}{point}\PY{p}{)}
                     \PY{n+nb+bp}{self}\PY{o}{.}\PY{n}{points} \PY{o}{=} \PY{n}{new\PYZus{}points}
                     
         \PY{c+c1}{\PYZsh{} for jupyter notebook}
         \PY{n}{KMeans}\PY{o}{.}\PY{n}{\PYZus{}dispersePoints} \PY{o}{=} \PY{n}{\PYZus{}dispersePoints}
\end{Verbatim}


    For centroid computing, here we just use each cluster's \emph{mean}
value as the centroid.

    \begin{Verbatim}[commandchars=\\\{\}]
{\color{incolor}In [{\color{incolor}16}]:}     \PY{k}{def} \PY{n+nf}{\PYZus{}computeCentroid}\PY{p}{(}\PY{n+nb+bp}{self}\PY{p}{)}\PY{p}{:}
                     \PY{l+s+sd}{\PYZdq{}\PYZdq{}\PYZdq{} Compute each groups centroid then update self.controids \PYZdq{}\PYZdq{}\PYZdq{}}
         
                     \PY{n}{new\PYZus{}centroids} \PY{o}{=} \PY{p}{[}\PY{p}{]}
                     \PY{k}{for} \PY{n}{cluster} \PY{o+ow}{in} \PY{n+nb+bp}{self}\PY{o}{.}\PY{n}{clusters}\PY{p}{:}
                         \PY{n}{x\PYZus{}cod} \PY{o}{=} \PY{p}{[}\PY{n}{point}\PY{p}{[}\PY{l+m+mi}{0}\PY{p}{]} \PY{k}{for} \PY{n}{point} \PY{o+ow}{in} \PY{n}{cluster}\PY{p}{]}
                         \PY{n}{y\PYZus{}cod} \PY{o}{=} \PY{p}{[}\PY{n}{point}\PY{p}{[}\PY{l+m+mi}{1}\PY{p}{]} \PY{k}{for} \PY{n}{point} \PY{o+ow}{in} \PY{n}{cluster}\PY{p}{]}
         
                         \PY{n}{centroid\PYZus{}x} \PY{o}{=} \PY{n+nb}{int}\PY{p}{(}\PY{n+nb}{sum}\PY{p}{(}\PY{n}{x\PYZus{}cod}\PY{p}{)}\PY{o}{/}\PY{n+nb}{len}\PY{p}{(}\PY{n}{x\PYZus{}cod}\PY{p}{)}\PY{p}{)}
                         \PY{n}{centroid\PYZus{}y} \PY{o}{=} \PY{n+nb}{int}\PY{p}{(}\PY{n+nb}{sum}\PY{p}{(}\PY{n}{y\PYZus{}cod}\PY{p}{)}\PY{o}{/}\PY{n+nb}{len}\PY{p}{(}\PY{n}{y\PYZus{}cod}\PY{p}{)}\PY{p}{)}
                         
                         \PY{n}{new\PYZus{}centroids}\PY{o}{.}\PY{n}{append}\PY{p}{(}\PY{p}{[}\PY{n}{centroid\PYZus{}x}\PY{p}{,} \PY{n}{centroid\PYZus{}y}\PY{p}{]}\PY{p}{)}
         
                     \PY{n+nb+bp}{self}\PY{o}{.}\PY{n}{centroids} \PY{o}{=} \PY{n}{new\PYZus{}centroids}
                     
         \PY{c+c1}{\PYZsh{} for jupyter notebook}
         \PY{n}{KMeans}\PY{o}{.}\PY{n}{\PYZus{}computeCentroid} \PY{o}{=} \PY{n}{\PYZus{}computeCentroid}
\end{Verbatim}


    \begin{center}\rule{0.5\linewidth}{\linethickness}\end{center}

Now, let's define the iteration part.

Since at the intialization part we've already did once label assignment
and centroids computation. In the iteration part, we just repeat on this
order.

A thing have to be aware of is, some time there will have empty cluster
be generated. If empty cluster occurs, we just remove it. That means the
number of clusters may not always equal to the \texttt{k} that we give
to the \texttt{\_\_init\_\_} function.

Label assignment and centroids computation are also seperated into
different functions.

    \begin{Verbatim}[commandchars=\\\{\}]
{\color{incolor}In [{\color{incolor}17}]:}     \PY{k}{def} \PY{n+nf}{\PYZus{}clustering}\PY{p}{(}\PY{n+nb+bp}{self}\PY{p}{)}\PY{p}{:}
                 \PY{l+s+sd}{\PYZdq{}\PYZdq{}\PYZdq{}Run algorithm one iteration}
         
         \PY{l+s+sd}{        Return:}
         \PY{l+s+sd}{            energy: This iteration\PYZsq{}s result energy}
         \PY{l+s+sd}{        \PYZdq{}\PYZdq{}\PYZdq{}}
         
                 \PY{n+nb+bp}{self}\PY{o}{.}\PY{n}{\PYZus{}assignLabel}\PY{p}{(}\PY{p}{)}
                 \PY{c+c1}{\PYZsh{} After re\PYZhy{}grouping if there are any empty cluster, delete it from }
                 \PY{c+c1}{\PYZsh{} list also delete corresponding centroid.}
                 \PY{k}{for} \PY{n}{i} \PY{o+ow}{in} \PY{n+nb}{range}\PY{p}{(}\PY{n+nb}{len}\PY{p}{(}\PY{n+nb+bp}{self}\PY{o}{.}\PY{n}{clusters}\PY{p}{)}\PY{p}{)}\PY{p}{:}
                     \PY{k}{if} \PY{p}{(}\PY{n+nb}{len}\PY{p}{(}\PY{n+nb+bp}{self}\PY{o}{.}\PY{n}{clusters}\PY{p}{[}\PY{n}{i}\PY{p}{]}\PY{p}{)} \PY{o}{==} \PY{l+m+mi}{0}\PY{p}{)}\PY{p}{:}
                         \PY{n+nb+bp}{self}\PY{o}{.}\PY{n}{clusters}\PY{o}{.}\PY{n}{pop}\PY{p}{(}\PY{n}{i}\PY{p}{)}
                         \PY{n+nb+bp}{self}\PY{o}{.}\PY{n}{centroids}\PY{o}{.}\PY{n}{pop}\PY{p}{(}\PY{n}{i}\PY{p}{)}
                         \PY{n+nb+bp}{self}\PY{o}{.}\PY{n}{colors}\PY{o}{.}\PY{n}{pop}\PY{p}{(}\PY{n}{i}\PY{p}{)}
                 \PY{n+nb+bp}{self}\PY{o}{.}\PY{n}{\PYZus{}computeCentroid}\PY{p}{(}\PY{p}{)}
                 \PY{n}{energy} \PY{o}{=} \PY{n+nb+bp}{self}\PY{o}{.}\PY{n}{\PYZus{}computeEnergy}\PY{p}{(}\PY{p}{)}
         
                 \PY{k}{return} \PY{n}{energy}
             
         \PY{c+c1}{\PYZsh{} for jupyter notebook}
         \PY{n}{KMeans}\PY{o}{.}\PY{n}{\PYZus{}clustering} \PY{o}{=} \PY{n}{\PYZus{}clustering}
\end{Verbatim}


    To assign label to points. Compute the distance between each point with
all centroids, find the closest one, then move it into that centroid's
cluster.

After all computation, update the \texttt{clusters} list.

    \begin{Verbatim}[commandchars=\\\{\}]
{\color{incolor}In [{\color{incolor}18}]:}     \PY{k}{def} \PY{n+nf}{\PYZus{}assignLabel}\PY{p}{(}\PY{n+nb+bp}{self}\PY{p}{)}\PY{p}{:}
                 \PY{l+s+sd}{\PYZdq{}\PYZdq{}\PYZdq{} Assign labels to points for generating new groups}
         
         \PY{l+s+sd}{        Compute distance between each point with each centroid,}
         \PY{l+s+sd}{        assign it to the closest centroid\PYZsq{}s group.}
         \PY{l+s+sd}{        \PYZdq{}\PYZdq{}\PYZdq{}}
         
                 \PY{c+c1}{\PYZsh{} for each point, compute the distance, get the closest centroid}
                 \PY{c+c1}{\PYZsh{} generate k new cluster}
                 \PY{n}{new\PYZus{}clusters} \PY{o}{=} \PY{p}{[}\PY{p}{]}
                 \PY{k}{for} \PY{n}{\PYZus{}} \PY{o+ow}{in} \PY{n+nb}{range}\PY{p}{(}\PY{n+nb}{len}\PY{p}{(}\PY{n+nb+bp}{self}\PY{o}{.}\PY{n}{clusters}\PY{p}{)}\PY{p}{)}\PY{p}{:}
                     \PY{n}{new\PYZus{}clusters}\PY{o}{.}\PY{n}{append}\PY{p}{(}\PY{p}{[}\PY{p}{]}\PY{p}{)}
         
                 \PY{k}{for} \PY{n}{point} \PY{o+ow}{in} \PY{n+nb+bp}{self}\PY{o}{.}\PY{n}{points}\PY{p}{:}
                     \PY{n+nb}{min} \PY{o}{=} \PY{n}{math}\PY{o}{.}\PY{n}{inf}
                     \PY{n}{closest} \PY{o}{=} \PY{l+m+mi}{0}
                     \PY{c+c1}{\PYZsh{} find the closest centroid}
                     \PY{k}{for} \PY{n}{i} \PY{o+ow}{in} \PY{n+nb}{range}\PY{p}{(}\PY{n+nb}{len}\PY{p}{(}\PY{n+nb+bp}{self}\PY{o}{.}\PY{n}{centroids}\PY{p}{)}\PY{p}{)}\PY{p}{:}
                         \PY{n}{dist} \PY{o}{=} \PY{n+nb+bp}{self}\PY{o}{.}\PY{n}{\PYZus{}computeDistance}\PY{p}{(}\PY{n}{point}\PY{p}{,} \PY{n+nb+bp}{self}\PY{o}{.}\PY{n}{centroids}\PY{p}{[}\PY{n}{i}\PY{p}{]}\PY{p}{)}
                         \PY{k}{if} \PY{n}{dist} \PY{o}{\PYZlt{}} \PY{n+nb}{min}\PY{p}{:}
                             \PY{n+nb}{min} \PY{o}{=} \PY{n}{dist}
                             \PY{n}{closest} \PY{o}{=} \PY{n}{i}
                     \PY{c+c1}{\PYZsh{} put point into new group}
                     \PY{n}{new\PYZus{}clusters}\PY{p}{[}\PY{n}{closest}\PY{p}{]}\PY{o}{.}\PY{n}{append}\PY{p}{(}\PY{n}{point}\PY{p}{)}
         
                 \PY{n+nb+bp}{self}\PY{o}{.}\PY{n}{clusters} \PY{o}{=} \PY{n}{new\PYZus{}clusters}
                 
         \PY{c+c1}{\PYZsh{} for jupyter notebook}
         \PY{n}{KMeans}\PY{o}{.}\PY{n}{\PYZus{}assignLabel} \PY{o}{=} \PY{n}{\PYZus{}assignLabel}
\end{Verbatim}


    For computing distances, we use \texttt{numpy}'s \texttt{linalg.norm}
directly, it gives us the \emph{Euclidean distance} between two vectors.

    \begin{Verbatim}[commandchars=\\\{\}]
{\color{incolor}In [{\color{incolor}19}]:}     \PY{k}{def} \PY{n+nf}{\PYZus{}computeDistance}\PY{p}{(}\PY{n+nb+bp}{self}\PY{p}{,} \PY{n}{x}\PY{p}{,} \PY{n}{y}\PY{p}{)}\PY{p}{:}
                 \PY{l+s+sd}{\PYZdq{}\PYZdq{}\PYZdq{}Compute the distance between two points}
         
         \PY{l+s+sd}{        root of (x1\PYZhy{}x2)\PYZca{}2 + (y1\PYZhy{}y2)\PYZca{}2}
         \PY{l+s+sd}{        \PYZdq{}\PYZdq{}\PYZdq{}}
         
                 \PY{n}{a} \PY{o}{=} \PY{n}{np}\PY{o}{.}\PY{n}{array}\PY{p}{(}\PY{n}{x}\PY{p}{)}
                 \PY{n}{b} \PY{o}{=} \PY{n}{np}\PY{o}{.}\PY{n}{array}\PY{p}{(}\PY{n}{y}\PY{p}{)}
         
                 \PY{k}{return} \PY{n}{np}\PY{o}{.}\PY{n}{linalg}\PY{o}{.}\PY{n}{norm}\PY{p}{(}\PY{n}{a} \PY{o}{\PYZhy{}} \PY{n}{b}\PY{p}{)}
             
         \PY{c+c1}{\PYZsh{} for jupyter notebook}
         \PY{n}{KMeans}\PY{o}{.}\PY{n}{\PYZus{}computeDistance} \PY{o}{=} \PY{n}{\PYZus{}computeDistance}
\end{Verbatim}


    \begin{center}\rule{0.5\linewidth}{\linethickness}\end{center}

Above is the end of the algorithm part.

Next, we define some functions to visualise the result.

    \begin{Verbatim}[commandchars=\\\{\}]
{\color{incolor}In [{\color{incolor}37}]:} \PY{c+c1}{\PYZsh{} Function to plot the result of the result, both clusters and centroids}
         \PY{k}{def} \PY{n+nf}{plotGraph}\PY{p}{(}\PY{n}{kmeans}\PY{p}{)}\PY{p}{:}
             \PY{l+s+sd}{\PYZdq{}\PYZdq{}\PYZdq{}Plot current clustering result\PYZdq{}\PYZdq{}\PYZdq{}}
         
             \PY{n}{plt}\PY{o}{.}\PY{n}{title}\PY{p}{(}\PY{l+s+s1}{\PYZsq{}}\PY{l+s+s1}{k\PYZhy{}means}\PY{l+s+s1}{\PYZsq{}}\PY{p}{)}
             \PY{n}{plt}\PY{o}{.}\PY{n}{xlabel}\PY{p}{(}\PY{l+s+s1}{\PYZsq{}}\PY{l+s+s1}{X}\PY{l+s+s1}{\PYZsq{}}\PY{p}{)}
             \PY{n}{plt}\PY{o}{.}\PY{n}{ylabel}\PY{p}{(}\PY{l+s+s1}{\PYZsq{}}\PY{l+s+s1}{Y}\PY{l+s+s1}{\PYZsq{}}\PY{p}{)}
         
             \PY{k}{for} \PY{n}{i} \PY{o+ow}{in} \PY{n+nb}{range}\PY{p}{(}\PY{n+nb}{len}\PY{p}{(}\PY{n}{kmeans}\PY{o}{.}\PY{n}{clusters}\PY{p}{)}\PY{p}{)}\PY{p}{:}
                 \PY{k}{for} \PY{n}{x}\PY{p}{,} \PY{n}{y} \PY{o+ow}{in} \PY{n}{kmeans}\PY{o}{.}\PY{n}{clusters}\PY{p}{[}\PY{n}{i}\PY{p}{]}\PY{p}{:}
                     \PY{n}{plt}\PY{o}{.}\PY{n}{scatter}\PY{p}{(}\PY{n}{x}\PY{p}{,} \PY{n}{y}\PY{p}{,} \PY{n}{marker}\PY{o}{=}\PY{l+s+s1}{\PYZsq{}}\PY{l+s+s1}{.}\PY{l+s+s1}{\PYZsq{}}\PY{p}{,} \PY{n}{color}\PY{o}{=}\PY{n}{kmeans}\PY{o}{.}\PY{n}{colors}\PY{p}{[}\PY{n}{i}\PY{p}{]}\PY{p}{)}
                 \PY{n}{x\PYZus{}centroid} \PY{o}{=} \PY{n}{kmeans}\PY{o}{.}\PY{n}{centroids}\PY{p}{[}\PY{n}{i}\PY{p}{]}\PY{p}{[}\PY{l+m+mi}{0}\PY{p}{]}
                 \PY{n}{y\PYZus{}centroid} \PY{o}{=} \PY{n}{kmeans}\PY{o}{.}\PY{n}{centroids}\PY{p}{[}\PY{n}{i}\PY{p}{]}\PY{p}{[}\PY{l+m+mi}{1}\PY{p}{]}
                 \PY{n}{plt}\PY{o}{.}\PY{n}{scatter}\PY{p}{(}
                     \PY{n}{x\PYZus{}centroid}\PY{p}{,}
                     \PY{n}{y\PYZus{}centroid}\PY{p}{,}
                     \PY{n}{marker}\PY{o}{=}\PY{l+s+s1}{\PYZsq{}}\PY{l+s+s1}{s}\PY{l+s+s1}{\PYZsq{}}\PY{p}{,}
                     \PY{n}{s}\PY{o}{=}\PY{l+m+mi}{100}\PY{p}{,}
                     \PY{n}{color}\PY{o}{=}\PY{n}{kmeans}\PY{o}{.}\PY{n}{colors}\PY{p}{[}\PY{n}{i}\PY{p}{]}\PY{p}{)}
             \PY{n}{plt}\PY{o}{.}\PY{n}{show}\PY{p}{(}\PY{p}{)}
         
         
         \PY{c+c1}{\PYZsh{} Input data plotting}
         \PY{k}{def} \PY{n+nf}{plotInputData}\PY{p}{(}\PY{n}{data}\PY{p}{)}\PY{p}{:}
             \PY{l+s+sd}{\PYZdq{}\PYZdq{}\PYZdq{}Plot the randomly generated points from KMeans}
         
         \PY{l+s+sd}{    Parameter:}
         \PY{l+s+sd}{        data(list): a list of 2d vectors}
         \PY{l+s+sd}{    \PYZdq{}\PYZdq{}\PYZdq{}}
         
             \PY{n}{plt}\PY{o}{.}\PY{n}{title}\PY{p}{(}\PY{l+s+s1}{\PYZsq{}}\PY{l+s+s1}{Input Data}\PY{l+s+s1}{\PYZsq{}}\PY{p}{)}
             \PY{n}{plt}\PY{o}{.}\PY{n}{xlabel}\PY{p}{(}\PY{l+s+s1}{\PYZsq{}}\PY{l+s+s1}{X}\PY{l+s+s1}{\PYZsq{}}\PY{p}{)}
             \PY{n}{plt}\PY{o}{.}\PY{n}{ylabel}\PY{p}{(}\PY{l+s+s1}{\PYZsq{}}\PY{l+s+s1}{Y}\PY{l+s+s1}{\PYZsq{}}\PY{p}{)}
         
             \PY{k}{for} \PY{n}{x}\PY{p}{,} \PY{n}{y} \PY{o+ow}{in} \PY{n}{data}\PY{p}{:}
                 \PY{n}{plt}\PY{o}{.}\PY{n}{scatter}\PY{p}{(}\PY{n}{x}\PY{p}{,} \PY{n}{y}\PY{p}{,} \PY{n}{color}\PY{o}{=}\PY{l+s+s1}{\PYZsq{}}\PY{l+s+s1}{black}\PY{l+s+s1}{\PYZsq{}}\PY{p}{,} \PY{n}{marker}\PY{o}{=}\PY{l+s+s1}{\PYZsq{}}\PY{l+s+s1}{.}\PY{l+s+s1}{\PYZsq{}}\PY{p}{)}
             \PY{n}{plt}\PY{o}{.}\PY{n}{show}\PY{p}{(}\PY{p}{)}
         
         
         \PY{c+c1}{\PYZsh{} Plot the labeled clusters}
         \PY{k}{def} \PY{n+nf}{plotLabel}\PY{p}{(}\PY{n}{kmeans}\PY{p}{)}\PY{p}{:}
             \PY{l+s+sd}{\PYZdq{}\PYZdq{}\PYZdq{}Plot initial clusters\PYZdq{}\PYZdq{}\PYZdq{}}
         
             \PY{n}{plt}\PY{o}{.}\PY{n}{title}\PY{p}{(}\PY{l+s+s1}{\PYZsq{}}\PY{l+s+s1}{Label}\PY{l+s+s1}{\PYZsq{}}\PY{p}{)}
             \PY{n}{plt}\PY{o}{.}\PY{n}{xlabel}\PY{p}{(}\PY{l+s+s1}{\PYZsq{}}\PY{l+s+s1}{X}\PY{l+s+s1}{\PYZsq{}}\PY{p}{)}
             \PY{n}{plt}\PY{o}{.}\PY{n}{ylabel}\PY{p}{(}\PY{l+s+s1}{\PYZsq{}}\PY{l+s+s1}{Y}\PY{l+s+s1}{\PYZsq{}}\PY{p}{)}
         
             \PY{k}{for} \PY{n}{i} \PY{o+ow}{in} \PY{n+nb}{range}\PY{p}{(}\PY{n}{kmeans}\PY{o}{.}\PY{n}{getNumOfClusters}\PY{p}{(}\PY{p}{)}\PY{p}{)}\PY{p}{:}
                 \PY{n}{clusters} \PY{o}{=} \PY{n}{kmeans}\PY{o}{.}\PY{n}{getClusters}\PY{p}{(}\PY{p}{)}
                 \PY{n}{colors} \PY{o}{=} \PY{n}{kmeans}\PY{o}{.}\PY{n}{getColors}\PY{p}{(}\PY{p}{)}
                 \PY{k}{for} \PY{n}{x}\PY{p}{,} \PY{n}{y} \PY{o+ow}{in} \PY{n}{clusters}\PY{p}{[}\PY{n}{i}\PY{p}{]}\PY{p}{:}
                     \PY{n}{plt}\PY{o}{.}\PY{n}{scatter}\PY{p}{(}\PY{n}{x}\PY{p}{,} \PY{n}{y}\PY{p}{,} \PY{n}{marker}\PY{o}{=}\PY{l+s+s1}{\PYZsq{}}\PY{l+s+s1}{.}\PY{l+s+s1}{\PYZsq{}}\PY{p}{,} \PY{n}{color}\PY{o}{=}\PY{n}{colors}\PY{p}{[}\PY{n}{i}\PY{p}{]}\PY{p}{)}
             \PY{n}{plt}\PY{o}{.}\PY{n}{show}\PY{p}{(}\PY{p}{)}
         
             
         \PY{c+c1}{\PYZsh{} Plot each iteration\PYZsq{}s energy}
         \PY{k}{def} \PY{n+nf}{plotEnergy}\PY{p}{(}\PY{n}{data}\PY{p}{)}\PY{p}{:}
             \PY{l+s+sd}{\PYZdq{}\PYZdq{}\PYZdq{}Plot energy history}
         
         \PY{l+s+sd}{    Parameter:}
         \PY{l+s+sd}{        data(list): a list of energy}
         \PY{l+s+sd}{    \PYZdq{}\PYZdq{}\PYZdq{}}
         
             \PY{n}{plt}\PY{o}{.}\PY{n}{rc}\PY{p}{(}\PY{l+s+s1}{\PYZsq{}}\PY{l+s+s1}{text}\PY{l+s+s1}{\PYZsq{}}\PY{p}{,} \PY{n}{usetex}\PY{o}{=}\PY{k+kc}{True}\PY{p}{)}
             \PY{n}{plt}\PY{o}{.}\PY{n}{rc}\PY{p}{(}\PY{l+s+s1}{\PYZsq{}}\PY{l+s+s1}{font}\PY{l+s+s1}{\PYZsq{}}\PY{p}{,} \PY{n}{family}\PY{o}{=}\PY{l+s+s1}{\PYZsq{}}\PY{l+s+s1}{serif}\PY{l+s+s1}{\PYZsq{}}\PY{p}{)}
         
             \PY{n}{plt}\PY{o}{.}\PY{n}{title}\PY{p}{(}\PY{l+s+s1}{\PYZsq{}}\PY{l+s+s1}{Energy Variation}\PY{l+s+s1}{\PYZsq{}}\PY{p}{)}
             \PY{n}{plt}\PY{o}{.}\PY{n}{xlabel}\PY{p}{(}\PY{l+s+s1}{\PYZsq{}}\PY{l+s+s1}{Iteration}\PY{l+s+s1}{\PYZsq{}}\PY{p}{)}
             \PY{n}{plt}\PY{o}{.}\PY{n}{ylabel}\PY{p}{(}\PY{l+s+sa}{r}\PY{l+s+s1}{\PYZsq{}}\PY{l+s+s1}{\PYZdl{}J\PYZca{}}\PY{l+s+si}{\PYZob{}clust\PYZcb{}}\PY{l+s+s1}{\PYZdl{}}\PY{l+s+s1}{\PYZsq{}}\PY{p}{)}
             \PY{n}{plt}\PY{o}{.}\PY{n}{xticks}\PY{p}{(}\PY{n+nb}{range}\PY{p}{(}\PY{n+nb}{len}\PY{p}{(}\PY{n}{data}\PY{p}{)}\PY{o}{+}\PY{l+m+mi}{1}\PY{p}{)}\PY{p}{)}
         
             \PY{n}{ite} \PY{o}{=} \PY{p}{[}\PY{n}{i} \PY{k}{for} \PY{n}{i} \PY{o+ow}{in} \PY{n}{np}\PY{o}{.}\PY{n}{arange}\PY{p}{(}\PY{l+m+mi}{1}\PY{p}{,} \PY{n+nb}{len}\PY{p}{(}\PY{n}{data}\PY{p}{)}\PY{o}{+}\PY{l+m+mi}{1}\PY{p}{,} \PY{l+m+mi}{1}\PY{p}{)}\PY{p}{]}
             \PY{n}{plt}\PY{o}{.}\PY{n}{plot}\PY{p}{(}\PY{n}{ite}\PY{p}{,} \PY{n}{data}\PY{p}{,} \PY{l+s+s1}{\PYZsq{}}\PY{l+s+s1}{o\PYZhy{}}\PY{l+s+s1}{\PYZsq{}}\PY{p}{,} \PY{n}{color}\PY{o}{=}\PY{l+s+s1}{\PYZsq{}}\PY{l+s+s1}{b}\PY{l+s+s1}{\PYZsq{}}\PY{p}{)}
             \PY{n}{plt}\PY{o}{.}\PY{n}{show}\PY{p}{(}\PY{p}{)}
\end{Verbatim}


    \subsection{Result}\label{result}

We have done all the implementation, let's check the result togher.

First try `k=3, num\_points=100'

    \begin{Verbatim}[commandchars=\\\{\}]
{\color{incolor}In [{\color{incolor}38}]:} \PY{k}{if} \PY{n+nv+vm}{\PYZus{}\PYZus{}name\PYZus{}\PYZus{}} \PY{o}{==} \PY{l+s+s1}{\PYZsq{}}\PY{l+s+s1}{\PYZus{}\PYZus{}main\PYZus{}\PYZus{}}\PY{l+s+s1}{\PYZsq{}}\PY{p}{:}
             \PY{n}{kmeans\PYZus{}1} \PY{o}{=} \PY{n}{KMeans}\PY{p}{(}\PY{l+m+mi}{3}\PY{p}{,} \PY{l+m+mi}{100}\PY{p}{)}
             \PY{n}{kmeans\PYZus{}1}\PY{o}{.}\PY{n}{generatePointCluster}\PY{p}{(}\PY{p}{)}
             \PY{n+nb}{print}\PY{p}{(}\PY{l+s+s1}{\PYZsq{}}\PY{l+s+s1}{Input Data:}\PY{l+s+s1}{\PYZsq{}}\PY{p}{)}
             \PY{n}{plotInputData}\PY{p}{(}\PY{n}{kmeans\PYZus{}1}\PY{o}{.}\PY{n}{getPoints}\PY{p}{(}\PY{p}{)}\PY{p}{)}  \PY{c+c1}{\PYZsh{} input data}
             \PY{n+nb}{print}\PY{p}{(}\PY{l+s+s1}{\PYZsq{}}\PY{l+s+se}{\PYZbs{}n}\PY{l+s+s1}{Initial Label:}\PY{l+s+s1}{\PYZsq{}}\PY{p}{)}
             \PY{n}{plotLabel}\PY{p}{(}\PY{n}{kmeans\PYZus{}1}\PY{p}{)}  \PY{c+c1}{\PYZsh{} initial label}
             \PY{n+nb}{print}\PY{p}{(}\PY{l+s+s1}{\PYZsq{}}\PY{l+s+se}{\PYZbs{}n}\PY{l+s+s1}{Initial Centroids:}\PY{l+s+s1}{\PYZsq{}}\PY{p}{)}
             \PY{n}{plotGraph}\PY{p}{(}\PY{n}{kmeans\PYZus{}1}\PY{p}{)}  \PY{c+c1}{\PYZsh{} initial centroids}
         
             \PY{n}{kmeans\PYZus{}1}\PY{o}{.}\PY{n}{run}\PY{p}{(}\PY{p}{)}
             
             \PY{n+nb}{print}\PY{p}{(}\PY{l+s+s1}{\PYZsq{}}\PY{l+s+se}{\PYZbs{}n}\PY{l+s+s1}{Final Label:}\PY{l+s+s1}{\PYZsq{}}\PY{p}{)}
             \PY{n}{plotLabel}\PY{p}{(}\PY{n}{kmeans\PYZus{}1}\PY{p}{)}  \PY{c+c1}{\PYZsh{} final label}
             \PY{n+nb}{print}\PY{p}{(}\PY{l+s+s1}{\PYZsq{}}\PY{l+s+se}{\PYZbs{}n}\PY{l+s+s1}{Final Centroids:}\PY{l+s+s1}{\PYZsq{}}\PY{p}{)}
             \PY{n}{plotGraph}\PY{p}{(}\PY{n}{kmeans\PYZus{}1}\PY{p}{)}  \PY{c+c1}{\PYZsh{} final result}
             \PY{n+nb}{print}\PY{p}{(}\PY{l+s+s1}{\PYZsq{}}\PY{l+s+se}{\PYZbs{}n}\PY{l+s+s1}{Energy change:}\PY{l+s+s1}{\PYZsq{}}\PY{p}{)}
             \PY{n}{plotEnergy}\PY{p}{(}\PY{n}{kmeans\PYZus{}1}\PY{o}{.}\PY{n}{getEnergyHistory}\PY{p}{(}\PY{p}{)}\PY{p}{)}
\end{Verbatim}


    \begin{Verbatim}[commandchars=\\\{\}]
Input Data:

    \end{Verbatim}

    \begin{center}
    \adjustimage{max size={0.9\linewidth}{0.9\paperheight}}{output_30_1.png}
    \end{center}
    { \hspace*{\fill} \\}
    
    \begin{Verbatim}[commandchars=\\\{\}]

Initial Label:

    \end{Verbatim}

    \begin{center}
    \adjustimage{max size={0.9\linewidth}{0.9\paperheight}}{output_30_3.png}
    \end{center}
    { \hspace*{\fill} \\}
    
    \begin{Verbatim}[commandchars=\\\{\}]

Initial Centroids:

    \end{Verbatim}

    \begin{center}
    \adjustimage{max size={0.9\linewidth}{0.9\paperheight}}{output_30_5.png}
    \end{center}
    { \hspace*{\fill} \\}
    
    \begin{Verbatim}[commandchars=\\\{\}]

Final Label:

    \end{Verbatim}

    \begin{center}
    \adjustimage{max size={0.9\linewidth}{0.9\paperheight}}{output_30_7.png}
    \end{center}
    { \hspace*{\fill} \\}
    
    \begin{Verbatim}[commandchars=\\\{\}]

Final Centroids:

    \end{Verbatim}

    \begin{center}
    \adjustimage{max size={0.9\linewidth}{0.9\paperheight}}{output_30_9.png}
    \end{center}
    { \hspace*{\fill} \\}
    
    \begin{Verbatim}[commandchars=\\\{\}]

Energy change:

    \end{Verbatim}

    \begin{center}
    \adjustimage{max size={0.9\linewidth}{0.9\paperheight}}{output_30_11.png}
    \end{center}
    { \hspace*{\fill} \\}
    
    \begin{center}\rule{0.5\linewidth}{\linethickness}\end{center}

Let's try some other inputs: \texttt{k} = 5, \texttt{num\_points} = 300

    \begin{Verbatim}[commandchars=\\\{\}]
{\color{incolor}In [{\color{incolor}40}]:} \PY{n}{kmeans\PYZus{}2} \PY{o}{=} \PY{n}{KMeans}\PY{p}{(}\PY{l+m+mi}{5}\PY{p}{,} \PY{l+m+mi}{300}\PY{p}{)}
         \PY{n}{kmeans\PYZus{}2}\PY{o}{.}\PY{n}{generatePointCluster}\PY{p}{(}\PY{p}{)}
         \PY{n+nb}{print}\PY{p}{(}\PY{l+s+s1}{\PYZsq{}}\PY{l+s+s1}{Input Data:}\PY{l+s+s1}{\PYZsq{}}\PY{p}{)}
         \PY{n}{plotInputData}\PY{p}{(}\PY{n}{kmeans\PYZus{}2}\PY{o}{.}\PY{n}{getPoints}\PY{p}{(}\PY{p}{)}\PY{p}{)}  \PY{c+c1}{\PYZsh{} input data}
         \PY{n+nb}{print}\PY{p}{(}\PY{l+s+s1}{\PYZsq{}}\PY{l+s+se}{\PYZbs{}n}\PY{l+s+s1}{Initial Label:}\PY{l+s+s1}{\PYZsq{}}\PY{p}{)}
         \PY{n}{plotLabel}\PY{p}{(}\PY{n}{kmeans\PYZus{}2}\PY{p}{)}  \PY{c+c1}{\PYZsh{} initial label}
         \PY{n+nb}{print}\PY{p}{(}\PY{l+s+s1}{\PYZsq{}}\PY{l+s+se}{\PYZbs{}n}\PY{l+s+s1}{Initial Centroids:}\PY{l+s+s1}{\PYZsq{}}\PY{p}{)}
         \PY{n}{plotGraph}\PY{p}{(}\PY{n}{kmeans\PYZus{}2}\PY{p}{)}  \PY{c+c1}{\PYZsh{} initial centroids}
         
         \PY{n}{kmeans\PYZus{}2}\PY{o}{.}\PY{n}{run}\PY{p}{(}\PY{p}{)}
         
         \PY{n+nb}{print}\PY{p}{(}\PY{l+s+s1}{\PYZsq{}}\PY{l+s+se}{\PYZbs{}n}\PY{l+s+s1}{Final Label:}\PY{l+s+s1}{\PYZsq{}}\PY{p}{)}
         \PY{n}{plotLabel}\PY{p}{(}\PY{n}{kmeans\PYZus{}2}\PY{p}{)}  \PY{c+c1}{\PYZsh{} final label}
         \PY{n+nb}{print}\PY{p}{(}\PY{l+s+s1}{\PYZsq{}}\PY{l+s+se}{\PYZbs{}n}\PY{l+s+s1}{Final Centroids:}\PY{l+s+s1}{\PYZsq{}}\PY{p}{)}
         \PY{n}{plotGraph}\PY{p}{(}\PY{n}{kmeans\PYZus{}2}\PY{p}{)}  \PY{c+c1}{\PYZsh{} final result}
         \PY{n+nb}{print}\PY{p}{(}\PY{l+s+s1}{\PYZsq{}}\PY{l+s+se}{\PYZbs{}n}\PY{l+s+s1}{Energy change:}\PY{l+s+s1}{\PYZsq{}}\PY{p}{)}
         \PY{n}{plotEnergy}\PY{p}{(}\PY{n}{kmeans\PYZus{}2}\PY{o}{.}\PY{n}{getEnergyHistory}\PY{p}{(}\PY{p}{)}\PY{p}{)}
\end{Verbatim}


    \begin{Verbatim}[commandchars=\\\{\}]
Input Data:

    \end{Verbatim}

    \begin{center}
    \adjustimage{max size={0.9\linewidth}{0.9\paperheight}}{output_32_1.png}
    \end{center}
    { \hspace*{\fill} \\}
    
    \begin{Verbatim}[commandchars=\\\{\}]

Initial Label:

    \end{Verbatim}

    \begin{center}
    \adjustimage{max size={0.9\linewidth}{0.9\paperheight}}{output_32_3.png}
    \end{center}
    { \hspace*{\fill} \\}
    
    \begin{Verbatim}[commandchars=\\\{\}]

Initial Centroids:

    \end{Verbatim}

    \begin{center}
    \adjustimage{max size={0.9\linewidth}{0.9\paperheight}}{output_32_5.png}
    \end{center}
    { \hspace*{\fill} \\}
    
    \begin{Verbatim}[commandchars=\\\{\}]

Final Label:

    \end{Verbatim}

    \begin{center}
    \adjustimage{max size={0.9\linewidth}{0.9\paperheight}}{output_32_7.png}
    \end{center}
    { \hspace*{\fill} \\}
    
    \begin{Verbatim}[commandchars=\\\{\}]

Final Centroids:

    \end{Verbatim}

    \begin{center}
    \adjustimage{max size={0.9\linewidth}{0.9\paperheight}}{output_32_9.png}
    \end{center}
    { \hspace*{\fill} \\}
    
    \begin{Verbatim}[commandchars=\\\{\}]

Energy change:

    \end{Verbatim}

    \begin{center}
    \adjustimage{max size={0.9\linewidth}{0.9\paperheight}}{output_32_11.png}
    \end{center}
    { \hspace*{\fill} \\}
    
    From above example, we can find the energy of each iteration is going
down, and at some point the value won't change again.


    % Add a bibliography block to the postdoc
    
    
    
    \end{document}
