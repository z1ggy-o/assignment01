
% Default to the notebook output style

    


% Inherit from the specified cell style.




    
\documentclass[11pt]{article}

    
    
    \usepackage[T1]{fontenc}
    % Nicer default font (+ math font) than Computer Modern for most use cases
    \usepackage{mathpazo}

    % Basic figure setup, for now with no caption control since it's done
    % automatically by Pandoc (which extracts ![](path) syntax from Markdown).
    \usepackage{graphicx}
    % We will generate all images so they have a width \maxwidth. This means
    % that they will get their normal width if they fit onto the page, but
    % are scaled down if they would overflow the margins.
    \makeatletter
    \def\maxwidth{\ifdim\Gin@nat@width>\linewidth\linewidth
    \else\Gin@nat@width\fi}
    \makeatother
    \let\Oldincludegraphics\includegraphics
    % Set max figure width to be 80% of text width, for now hardcoded.
    \renewcommand{\includegraphics}[1]{\Oldincludegraphics[width=.8\maxwidth]{#1}}
    % Ensure that by default, figures have no caption (until we provide a
    % proper Figure object with a Caption API and a way to capture that
    % in the conversion process - todo).
    \usepackage{caption}
    \DeclareCaptionLabelFormat{nolabel}{}
    \captionsetup{labelformat=nolabel}

    \usepackage{adjustbox} % Used to constrain images to a maximum size 
    \usepackage{xcolor} % Allow colors to be defined
    \usepackage{enumerate} % Needed for markdown enumerations to work
    \usepackage{geometry} % Used to adjust the document margins
    \usepackage{amsmath} % Equations
    \usepackage{amssymb} % Equations
    \usepackage{textcomp} % defines textquotesingle
    % Hack from http://tex.stackexchange.com/a/47451/13684:
    \AtBeginDocument{%
        \def\PYZsq{\textquotesingle}% Upright quotes in Pygmentized code
    }
    \usepackage{upquote} % Upright quotes for verbatim code
    \usepackage{eurosym} % defines \euro
    \usepackage[mathletters]{ucs} % Extended unicode (utf-8) support
    \usepackage[utf8x]{inputenc} % Allow utf-8 characters in the tex document
    \usepackage{fancyvrb} % verbatim replacement that allows latex
    \usepackage{grffile} % extends the file name processing of package graphics 
                         % to support a larger range 
    % The hyperref package gives us a pdf with properly built
    % internal navigation ('pdf bookmarks' for the table of contents,
    % internal cross-reference links, web links for URLs, etc.)
    \usepackage{hyperref}
    \usepackage{longtable} % longtable support required by pandoc >1.10
    \usepackage{booktabs}  % table support for pandoc > 1.12.2
    \usepackage[inline]{enumitem} % IRkernel/repr support (it uses the enumerate* environment)
    \usepackage[normalem]{ulem} % ulem is needed to support strikethroughs (\sout)
                                % normalem makes italics be italics, not underlines
    

    
    
    % Colors for the hyperref package
    \definecolor{urlcolor}{rgb}{0,.145,.698}
    \definecolor{linkcolor}{rgb}{.71,0.21,0.01}
    \definecolor{citecolor}{rgb}{.12,.54,.11}

    % ANSI colors
    \definecolor{ansi-black}{HTML}{3E424D}
    \definecolor{ansi-black-intense}{HTML}{282C36}
    \definecolor{ansi-red}{HTML}{E75C58}
    \definecolor{ansi-red-intense}{HTML}{B22B31}
    \definecolor{ansi-green}{HTML}{00A250}
    \definecolor{ansi-green-intense}{HTML}{007427}
    \definecolor{ansi-yellow}{HTML}{DDB62B}
    \definecolor{ansi-yellow-intense}{HTML}{B27D12}
    \definecolor{ansi-blue}{HTML}{208FFB}
    \definecolor{ansi-blue-intense}{HTML}{0065CA}
    \definecolor{ansi-magenta}{HTML}{D160C4}
    \definecolor{ansi-magenta-intense}{HTML}{A03196}
    \definecolor{ansi-cyan}{HTML}{60C6C8}
    \definecolor{ansi-cyan-intense}{HTML}{258F8F}
    \definecolor{ansi-white}{HTML}{C5C1B4}
    \definecolor{ansi-white-intense}{HTML}{A1A6B2}

    % commands and environments needed by pandoc snippets
    % extracted from the output of `pandoc -s`
    \providecommand{\tightlist}{%
      \setlength{\itemsep}{0pt}\setlength{\parskip}{0pt}}
    \DefineVerbatimEnvironment{Highlighting}{Verbatim}{commandchars=\\\{\}}
    % Add ',fontsize=\small' for more characters per line
    \newenvironment{Shaded}{}{}
    \newcommand{\KeywordTok}[1]{\textcolor[rgb]{0.00,0.44,0.13}{\textbf{{#1}}}}
    \newcommand{\DataTypeTok}[1]{\textcolor[rgb]{0.56,0.13,0.00}{{#1}}}
    \newcommand{\DecValTok}[1]{\textcolor[rgb]{0.25,0.63,0.44}{{#1}}}
    \newcommand{\BaseNTok}[1]{\textcolor[rgb]{0.25,0.63,0.44}{{#1}}}
    \newcommand{\FloatTok}[1]{\textcolor[rgb]{0.25,0.63,0.44}{{#1}}}
    \newcommand{\CharTok}[1]{\textcolor[rgb]{0.25,0.44,0.63}{{#1}}}
    \newcommand{\StringTok}[1]{\textcolor[rgb]{0.25,0.44,0.63}{{#1}}}
    \newcommand{\CommentTok}[1]{\textcolor[rgb]{0.38,0.63,0.69}{\textit{{#1}}}}
    \newcommand{\OtherTok}[1]{\textcolor[rgb]{0.00,0.44,0.13}{{#1}}}
    \newcommand{\AlertTok}[1]{\textcolor[rgb]{1.00,0.00,0.00}{\textbf{{#1}}}}
    \newcommand{\FunctionTok}[1]{\textcolor[rgb]{0.02,0.16,0.49}{{#1}}}
    \newcommand{\RegionMarkerTok}[1]{{#1}}
    \newcommand{\ErrorTok}[1]{\textcolor[rgb]{1.00,0.00,0.00}{\textbf{{#1}}}}
    \newcommand{\NormalTok}[1]{{#1}}
    
    % Additional commands for more recent versions of Pandoc
    \newcommand{\ConstantTok}[1]{\textcolor[rgb]{0.53,0.00,0.00}{{#1}}}
    \newcommand{\SpecialCharTok}[1]{\textcolor[rgb]{0.25,0.44,0.63}{{#1}}}
    \newcommand{\VerbatimStringTok}[1]{\textcolor[rgb]{0.25,0.44,0.63}{{#1}}}
    \newcommand{\SpecialStringTok}[1]{\textcolor[rgb]{0.73,0.40,0.53}{{#1}}}
    \newcommand{\ImportTok}[1]{{#1}}
    \newcommand{\DocumentationTok}[1]{\textcolor[rgb]{0.73,0.13,0.13}{\textit{{#1}}}}
    \newcommand{\AnnotationTok}[1]{\textcolor[rgb]{0.38,0.63,0.69}{\textbf{\textit{{#1}}}}}
    \newcommand{\CommentVarTok}[1]{\textcolor[rgb]{0.38,0.63,0.69}{\textbf{\textit{{#1}}}}}
    \newcommand{\VariableTok}[1]{\textcolor[rgb]{0.10,0.09,0.49}{{#1}}}
    \newcommand{\ControlFlowTok}[1]{\textcolor[rgb]{0.00,0.44,0.13}{\textbf{{#1}}}}
    \newcommand{\OperatorTok}[1]{\textcolor[rgb]{0.40,0.40,0.40}{{#1}}}
    \newcommand{\BuiltInTok}[1]{{#1}}
    \newcommand{\ExtensionTok}[1]{{#1}}
    \newcommand{\PreprocessorTok}[1]{\textcolor[rgb]{0.74,0.48,0.00}{{#1}}}
    \newcommand{\AttributeTok}[1]{\textcolor[rgb]{0.49,0.56,0.16}{{#1}}}
    \newcommand{\InformationTok}[1]{\textcolor[rgb]{0.38,0.63,0.69}{\textbf{\textit{{#1}}}}}
    \newcommand{\WarningTok}[1]{\textcolor[rgb]{0.38,0.63,0.69}{\textbf{\textit{{#1}}}}}
    
    
    % Define a nice break command that doesn't care if a line doesn't already
    % exist.
    \def\br{\hspace*{\fill} \\* }
    % Math Jax compatability definitions
    \def\gt{>}
    \def\lt{<}
    % Document parameters
    \title{assignment04}
    
    
    

    % Pygments definitions
    
\makeatletter
\def\PY@reset{\let\PY@it=\relax \let\PY@bf=\relax%
    \let\PY@ul=\relax \let\PY@tc=\relax%
    \let\PY@bc=\relax \let\PY@ff=\relax}
\def\PY@tok#1{\csname PY@tok@#1\endcsname}
\def\PY@toks#1+{\ifx\relax#1\empty\else%
    \PY@tok{#1}\expandafter\PY@toks\fi}
\def\PY@do#1{\PY@bc{\PY@tc{\PY@ul{%
    \PY@it{\PY@bf{\PY@ff{#1}}}}}}}
\def\PY#1#2{\PY@reset\PY@toks#1+\relax+\PY@do{#2}}

\expandafter\def\csname PY@tok@w\endcsname{\def\PY@tc##1{\textcolor[rgb]{0.73,0.73,0.73}{##1}}}
\expandafter\def\csname PY@tok@c\endcsname{\let\PY@it=\textit\def\PY@tc##1{\textcolor[rgb]{0.25,0.50,0.50}{##1}}}
\expandafter\def\csname PY@tok@cp\endcsname{\def\PY@tc##1{\textcolor[rgb]{0.74,0.48,0.00}{##1}}}
\expandafter\def\csname PY@tok@k\endcsname{\let\PY@bf=\textbf\def\PY@tc##1{\textcolor[rgb]{0.00,0.50,0.00}{##1}}}
\expandafter\def\csname PY@tok@kp\endcsname{\def\PY@tc##1{\textcolor[rgb]{0.00,0.50,0.00}{##1}}}
\expandafter\def\csname PY@tok@kt\endcsname{\def\PY@tc##1{\textcolor[rgb]{0.69,0.00,0.25}{##1}}}
\expandafter\def\csname PY@tok@o\endcsname{\def\PY@tc##1{\textcolor[rgb]{0.40,0.40,0.40}{##1}}}
\expandafter\def\csname PY@tok@ow\endcsname{\let\PY@bf=\textbf\def\PY@tc##1{\textcolor[rgb]{0.67,0.13,1.00}{##1}}}
\expandafter\def\csname PY@tok@nb\endcsname{\def\PY@tc##1{\textcolor[rgb]{0.00,0.50,0.00}{##1}}}
\expandafter\def\csname PY@tok@nf\endcsname{\def\PY@tc##1{\textcolor[rgb]{0.00,0.00,1.00}{##1}}}
\expandafter\def\csname PY@tok@nc\endcsname{\let\PY@bf=\textbf\def\PY@tc##1{\textcolor[rgb]{0.00,0.00,1.00}{##1}}}
\expandafter\def\csname PY@tok@nn\endcsname{\let\PY@bf=\textbf\def\PY@tc##1{\textcolor[rgb]{0.00,0.00,1.00}{##1}}}
\expandafter\def\csname PY@tok@ne\endcsname{\let\PY@bf=\textbf\def\PY@tc##1{\textcolor[rgb]{0.82,0.25,0.23}{##1}}}
\expandafter\def\csname PY@tok@nv\endcsname{\def\PY@tc##1{\textcolor[rgb]{0.10,0.09,0.49}{##1}}}
\expandafter\def\csname PY@tok@no\endcsname{\def\PY@tc##1{\textcolor[rgb]{0.53,0.00,0.00}{##1}}}
\expandafter\def\csname PY@tok@nl\endcsname{\def\PY@tc##1{\textcolor[rgb]{0.63,0.63,0.00}{##1}}}
\expandafter\def\csname PY@tok@ni\endcsname{\let\PY@bf=\textbf\def\PY@tc##1{\textcolor[rgb]{0.60,0.60,0.60}{##1}}}
\expandafter\def\csname PY@tok@na\endcsname{\def\PY@tc##1{\textcolor[rgb]{0.49,0.56,0.16}{##1}}}
\expandafter\def\csname PY@tok@nt\endcsname{\let\PY@bf=\textbf\def\PY@tc##1{\textcolor[rgb]{0.00,0.50,0.00}{##1}}}
\expandafter\def\csname PY@tok@nd\endcsname{\def\PY@tc##1{\textcolor[rgb]{0.67,0.13,1.00}{##1}}}
\expandafter\def\csname PY@tok@s\endcsname{\def\PY@tc##1{\textcolor[rgb]{0.73,0.13,0.13}{##1}}}
\expandafter\def\csname PY@tok@sd\endcsname{\let\PY@it=\textit\def\PY@tc##1{\textcolor[rgb]{0.73,0.13,0.13}{##1}}}
\expandafter\def\csname PY@tok@si\endcsname{\let\PY@bf=\textbf\def\PY@tc##1{\textcolor[rgb]{0.73,0.40,0.53}{##1}}}
\expandafter\def\csname PY@tok@se\endcsname{\let\PY@bf=\textbf\def\PY@tc##1{\textcolor[rgb]{0.73,0.40,0.13}{##1}}}
\expandafter\def\csname PY@tok@sr\endcsname{\def\PY@tc##1{\textcolor[rgb]{0.73,0.40,0.53}{##1}}}
\expandafter\def\csname PY@tok@ss\endcsname{\def\PY@tc##1{\textcolor[rgb]{0.10,0.09,0.49}{##1}}}
\expandafter\def\csname PY@tok@sx\endcsname{\def\PY@tc##1{\textcolor[rgb]{0.00,0.50,0.00}{##1}}}
\expandafter\def\csname PY@tok@m\endcsname{\def\PY@tc##1{\textcolor[rgb]{0.40,0.40,0.40}{##1}}}
\expandafter\def\csname PY@tok@gh\endcsname{\let\PY@bf=\textbf\def\PY@tc##1{\textcolor[rgb]{0.00,0.00,0.50}{##1}}}
\expandafter\def\csname PY@tok@gu\endcsname{\let\PY@bf=\textbf\def\PY@tc##1{\textcolor[rgb]{0.50,0.00,0.50}{##1}}}
\expandafter\def\csname PY@tok@gd\endcsname{\def\PY@tc##1{\textcolor[rgb]{0.63,0.00,0.00}{##1}}}
\expandafter\def\csname PY@tok@gi\endcsname{\def\PY@tc##1{\textcolor[rgb]{0.00,0.63,0.00}{##1}}}
\expandafter\def\csname PY@tok@gr\endcsname{\def\PY@tc##1{\textcolor[rgb]{1.00,0.00,0.00}{##1}}}
\expandafter\def\csname PY@tok@ge\endcsname{\let\PY@it=\textit}
\expandafter\def\csname PY@tok@gs\endcsname{\let\PY@bf=\textbf}
\expandafter\def\csname PY@tok@gp\endcsname{\let\PY@bf=\textbf\def\PY@tc##1{\textcolor[rgb]{0.00,0.00,0.50}{##1}}}
\expandafter\def\csname PY@tok@go\endcsname{\def\PY@tc##1{\textcolor[rgb]{0.53,0.53,0.53}{##1}}}
\expandafter\def\csname PY@tok@gt\endcsname{\def\PY@tc##1{\textcolor[rgb]{0.00,0.27,0.87}{##1}}}
\expandafter\def\csname PY@tok@err\endcsname{\def\PY@bc##1{\setlength{\fboxsep}{0pt}\fcolorbox[rgb]{1.00,0.00,0.00}{1,1,1}{\strut ##1}}}
\expandafter\def\csname PY@tok@kc\endcsname{\let\PY@bf=\textbf\def\PY@tc##1{\textcolor[rgb]{0.00,0.50,0.00}{##1}}}
\expandafter\def\csname PY@tok@kd\endcsname{\let\PY@bf=\textbf\def\PY@tc##1{\textcolor[rgb]{0.00,0.50,0.00}{##1}}}
\expandafter\def\csname PY@tok@kn\endcsname{\let\PY@bf=\textbf\def\PY@tc##1{\textcolor[rgb]{0.00,0.50,0.00}{##1}}}
\expandafter\def\csname PY@tok@kr\endcsname{\let\PY@bf=\textbf\def\PY@tc##1{\textcolor[rgb]{0.00,0.50,0.00}{##1}}}
\expandafter\def\csname PY@tok@bp\endcsname{\def\PY@tc##1{\textcolor[rgb]{0.00,0.50,0.00}{##1}}}
\expandafter\def\csname PY@tok@fm\endcsname{\def\PY@tc##1{\textcolor[rgb]{0.00,0.00,1.00}{##1}}}
\expandafter\def\csname PY@tok@vc\endcsname{\def\PY@tc##1{\textcolor[rgb]{0.10,0.09,0.49}{##1}}}
\expandafter\def\csname PY@tok@vg\endcsname{\def\PY@tc##1{\textcolor[rgb]{0.10,0.09,0.49}{##1}}}
\expandafter\def\csname PY@tok@vi\endcsname{\def\PY@tc##1{\textcolor[rgb]{0.10,0.09,0.49}{##1}}}
\expandafter\def\csname PY@tok@vm\endcsname{\def\PY@tc##1{\textcolor[rgb]{0.10,0.09,0.49}{##1}}}
\expandafter\def\csname PY@tok@sa\endcsname{\def\PY@tc##1{\textcolor[rgb]{0.73,0.13,0.13}{##1}}}
\expandafter\def\csname PY@tok@sb\endcsname{\def\PY@tc##1{\textcolor[rgb]{0.73,0.13,0.13}{##1}}}
\expandafter\def\csname PY@tok@sc\endcsname{\def\PY@tc##1{\textcolor[rgb]{0.73,0.13,0.13}{##1}}}
\expandafter\def\csname PY@tok@dl\endcsname{\def\PY@tc##1{\textcolor[rgb]{0.73,0.13,0.13}{##1}}}
\expandafter\def\csname PY@tok@s2\endcsname{\def\PY@tc##1{\textcolor[rgb]{0.73,0.13,0.13}{##1}}}
\expandafter\def\csname PY@tok@sh\endcsname{\def\PY@tc##1{\textcolor[rgb]{0.73,0.13,0.13}{##1}}}
\expandafter\def\csname PY@tok@s1\endcsname{\def\PY@tc##1{\textcolor[rgb]{0.73,0.13,0.13}{##1}}}
\expandafter\def\csname PY@tok@mb\endcsname{\def\PY@tc##1{\textcolor[rgb]{0.40,0.40,0.40}{##1}}}
\expandafter\def\csname PY@tok@mf\endcsname{\def\PY@tc##1{\textcolor[rgb]{0.40,0.40,0.40}{##1}}}
\expandafter\def\csname PY@tok@mh\endcsname{\def\PY@tc##1{\textcolor[rgb]{0.40,0.40,0.40}{##1}}}
\expandafter\def\csname PY@tok@mi\endcsname{\def\PY@tc##1{\textcolor[rgb]{0.40,0.40,0.40}{##1}}}
\expandafter\def\csname PY@tok@il\endcsname{\def\PY@tc##1{\textcolor[rgb]{0.40,0.40,0.40}{##1}}}
\expandafter\def\csname PY@tok@mo\endcsname{\def\PY@tc##1{\textcolor[rgb]{0.40,0.40,0.40}{##1}}}
\expandafter\def\csname PY@tok@ch\endcsname{\let\PY@it=\textit\def\PY@tc##1{\textcolor[rgb]{0.25,0.50,0.50}{##1}}}
\expandafter\def\csname PY@tok@cm\endcsname{\let\PY@it=\textit\def\PY@tc##1{\textcolor[rgb]{0.25,0.50,0.50}{##1}}}
\expandafter\def\csname PY@tok@cpf\endcsname{\let\PY@it=\textit\def\PY@tc##1{\textcolor[rgb]{0.25,0.50,0.50}{##1}}}
\expandafter\def\csname PY@tok@c1\endcsname{\let\PY@it=\textit\def\PY@tc##1{\textcolor[rgb]{0.25,0.50,0.50}{##1}}}
\expandafter\def\csname PY@tok@cs\endcsname{\let\PY@it=\textit\def\PY@tc##1{\textcolor[rgb]{0.25,0.50,0.50}{##1}}}

\def\PYZbs{\char`\\}
\def\PYZus{\char`\_}
\def\PYZob{\char`\{}
\def\PYZcb{\char`\}}
\def\PYZca{\char`\^}
\def\PYZam{\char`\&}
\def\PYZlt{\char`\<}
\def\PYZgt{\char`\>}
\def\PYZsh{\char`\#}
\def\PYZpc{\char`\%}
\def\PYZdl{\char`\$}
\def\PYZhy{\char`\-}
\def\PYZsq{\char`\'}
\def\PYZdq{\char`\"}
\def\PYZti{\char`\~}
% for compatibility with earlier versions
\def\PYZat{@}
\def\PYZlb{[}
\def\PYZrb{]}
\makeatother


    % Exact colors from NB
    \definecolor{incolor}{rgb}{0.0, 0.0, 0.5}
    \definecolor{outcolor}{rgb}{0.545, 0.0, 0.0}



    
    % Prevent overflowing lines due to hard-to-break entities
    \sloppy 
    % Setup hyperref package
    \hypersetup{
      breaklinks=true,  % so long urls are correctly broken across lines
      colorlinks=true,
      urlcolor=urlcolor,
      linkcolor=linkcolor,
      citecolor=citecolor,
      }
    % Slightly bigger margins than the latex defaults
    
    \geometry{verbose,tmargin=1in,bmargin=1in,lmargin=1in,rmargin=1in}
    
    

    \begin{document}
    
    
    \maketitle
    
    

    
    \section{K-means Algorithm with
Images}\label{k-means-algorithm-with-images}

\textbf{Name}: ZHU GUANGYU\\
\textbf{Student ID}: 20165953\\
\textbf{Github Repo}: \href{}{assignment03}

\begin{center}\rule{0.5\linewidth}{\linethickness}\end{center}

\subsection{Clustering}\label{clustering}

the goal of \emph{clustering} is to group or partition the vectors into
\emph{k} groups or clusters, with the vectors in each group close to
each other.

the best clustering: - can find the best slustering, if the
representatives are fixed; - can find the vest representatives, if the
clustering is fixed.

We use a single number to judge a choice of clustering, along with a
choice of the group. We define:

\[
J^{clust}=(||x_{1}-z_{c_{1}}||^{2}+ \cdots +||x_{N}-z_{c_{N}}||^{2})/N
\]

Here \(x_{N}\) is vector, \(z_{c_{N}}\) is correspond representatives.
We call this value \emph{energy} or \emph{cost}.

\subsubsection{When representatives
fixed}\label{when-representatives-fixed}

We assign each date vector \(x_{i}\) to its nearest representatives.
Since the representatives are fixed, we actually re-grouped vectors into
different partitions. We have:
\[ ||x_{i}-z_{c_{i}}|| = \min_{j=1,\cdots\,k}||x_{i}-z_{c_{i}}|| \]

This gives us the minimum \(J^{clust}\).

\subsubsection{When group assignment
fixed}\label{when-group-assignment-fixed}

This means the element vectors of each group are fixed. We need to find
the group representatives to minimize our cost \(J^{clust}\).

Simply, choose the average of the vectros in its group:
\[ z_{j} = (1/|G_{j}|)\sum_{i\in G_{j}}x_{i}\]

since this makes the sum of distance between points and its
representative minimum.

\subsection{\texorpdfstring{\emph{k}-means
Algorithm}{k-means Algorithm}}\label{k-means-algorithm}

Previous two methods can help us get the best clustering. But the two
methods are depend on each other. To solve this problem, we can use
\emph{k-means algorithm}.

\emph{k-means algorithm}'s idea is simple. We repeatedly alternate
between updating the group assignments, then updating the
representatives. In each iteration we get a better \(J^{clust}\) until
the step does not change the choice.

Have to be aware of is k-means algorithm \textbf{cannot} guarantee that
the partition it finds minimizes \(J^{clust}\). Commonly, we run it
several times with different initial representatives, and choose the one
with the smallest cost.

There is another problem is to determin the optimal number of clusters
(here is the \emph{k}).\\
If you have given number of clusters, that's fine. If you don't, there
are few methods can help us:

\begin{itemize}
\tightlist
\item
  \href{https://en.wikipedia.org/wiki/Elbow_method_\%28clustering\%29}{Elbow
  method}
\item
  \href{https://en.wikipedia.org/wiki/Silhouette_\%28clustering\%29}{The
  silhouette method}
\item
  \href{http://web.stanford.edu/~hastie/Papers/gap.pdf}{Gap statistic}
\end{itemize}

\begin{center}\rule{0.5\linewidth}{\linethickness}\end{center}

    \subsection{Implementation}\label{implementation}

This time we try to modify the \emph{k-means} algorithm we did last time
to cluster image datas.

The algorithm part is almost same, just need to change the data vector
size and plotting functions.

\begin{center}\rule{0.5\linewidth}{\linethickness}\end{center}

First, import packages that we need.

\begin{itemize}
\tightlist
\item
  \texttt{numpy} for scientific computing
\item
  \texttt{matplotlib} for visualization
\item
  \texttt{math} is the python build in math packages.
\end{itemize}

    \begin{Verbatim}[commandchars=\\\{\}]
{\color{incolor}In [{\color{incolor}1}]:} \PY{k+kn}{import} \PY{n+nn}{math}
        \PY{k+kn}{import} \PY{n+nn}{numpy} \PY{k}{as} \PY{n+nn}{np}
        \PY{k+kn}{from} \PY{n+nn}{matplotlib} \PY{k}{import} \PY{n}{pyplot} \PY{k}{as} \PY{n}{plt}
\end{Verbatim}


    \subsubsection{Read file and
pre-processing}\label{read-file-and-pre-processing}

Define a class called \texttt{DataSet} to read image data from file and
normaize the input data.

    \begin{Verbatim}[commandchars=\\\{\}]
{\color{incolor}In [{\color{incolor}2}]:} \PY{k}{class} \PY{n+nc}{DataSet}\PY{p}{:}
            \PY{l+s+sd}{\PYZdq{}\PYZdq{}\PYZdq{}Read data from files}
        
        \PY{l+s+sd}{    Read image datas from data files.}
        \PY{l+s+sd}{    Normalize the data then change the data to vector from}
        \PY{l+s+sd}{    \PYZdq{}\PYZdq{}\PYZdq{}}
        
            \PY{k}{def} \PY{n+nf}{\PYZus{}\PYZus{}init\PYZus{}\PYZus{}}\PY{p}{(}\PY{n+nb+bp}{self}\PY{p}{,} \PY{n}{file}\PY{o}{=}\PY{l+s+s2}{\PYZdq{}}\PY{l+s+s2}{mnist\PYZus{}test.csv}\PY{l+s+s2}{\PYZdq{}}\PY{p}{)}\PY{p}{:}
                \PY{n+nb+bp}{self}\PY{o}{.}\PY{n}{size\PYZus{}row} \PY{o}{=} \PY{l+m+mi}{28}  \PY{c+c1}{\PYZsh{} height of the image data}
                \PY{n+nb+bp}{self}\PY{o}{.}\PY{n}{size\PYZus{}col} \PY{o}{=} \PY{l+m+mi}{28}  \PY{c+c1}{\PYZsh{} width of the image data}
                \PY{n+nb+bp}{self}\PY{o}{.}\PY{n}{len\PYZus{}vec} \PY{o}{=} \PY{n+nb+bp}{self}\PY{o}{.}\PY{n}{size\PYZus{}row} \PY{o}{*} \PY{n+nb+bp}{self}\PY{o}{.}\PY{n}{size\PYZus{}col}
        
                \PY{c+c1}{\PYZsh{} read data from file}
                \PY{n}{\PYZus{}handle\PYZus{}file} \PY{o}{=} \PY{n+nb}{open}\PY{p}{(}\PY{n}{file}\PY{p}{,} \PY{l+s+s1}{\PYZsq{}}\PY{l+s+s1}{r}\PY{l+s+s1}{\PYZsq{}}\PY{p}{)}
                \PY{n}{\PYZus{}data} \PY{o}{=} \PY{n}{\PYZus{}handle\PYZus{}file}\PY{o}{.}\PY{n}{readlines}\PY{p}{(}\PY{p}{)}
                \PY{n}{\PYZus{}handle\PYZus{}file}\PY{o}{.}\PY{n}{close}\PY{p}{(}\PY{p}{)}
        
                \PY{n+nb+bp}{self}\PY{o}{.}\PY{n}{num\PYZus{}image} \PY{o}{=} \PY{n+nb}{len}\PY{p}{(}\PY{n}{\PYZus{}data}\PY{p}{)}
        
                \PY{n+nb+bp}{self}\PY{o}{.}\PY{n}{list\PYZus{}image} \PY{o}{=} \PY{n}{np}\PY{o}{.}\PY{n}{empty}\PY{p}{(}\PY{p}{(}\PY{n+nb+bp}{self}\PY{o}{.}\PY{n}{len\PYZus{}vec}\PY{p}{,} \PY{n+nb+bp}{self}\PY{o}{.}\PY{n}{num\PYZus{}image}\PY{p}{)}\PY{p}{,} \PY{n}{dtype}\PY{o}{=}\PY{n+nb}{float}\PY{p}{)}
                \PY{n+nb+bp}{self}\PY{o}{.}\PY{n}{list\PYZus{}label} \PY{o}{=} \PY{n}{np}\PY{o}{.}\PY{n}{empty}\PY{p}{(}\PY{n+nb+bp}{self}\PY{o}{.}\PY{n}{num\PYZus{}image}\PY{p}{,} \PY{n}{dtype}\PY{o}{=}\PY{n+nb}{int}\PY{p}{)}
        
                \PY{c+c1}{\PYZsh{} change the image datas into vector forms}
                \PY{n}{\PYZus{}count} \PY{o}{=} \PY{l+m+mi}{0}
                \PY{k}{for} \PY{n}{line} \PY{o+ow}{in} \PY{n}{\PYZus{}data}\PY{p}{:}
                    \PY{n}{line\PYZus{}data} \PY{o}{=} \PY{n}{line}\PY{o}{.}\PY{n}{split}\PY{p}{(}\PY{l+s+s1}{\PYZsq{}}\PY{l+s+s1}{,}\PY{l+s+s1}{\PYZsq{}}\PY{p}{)}
                    \PY{n}{label} \PY{o}{=} \PY{n}{line\PYZus{}data}\PY{p}{[}\PY{l+m+mi}{0}\PY{p}{]}
                    \PY{n}{img\PYZus{}vector} \PY{o}{=} \PY{n}{np}\PY{o}{.}\PY{n}{asfarray}\PY{p}{(}\PY{n}{line\PYZus{}data}\PY{p}{[}\PY{l+m+mi}{1}\PY{p}{:}\PY{p}{]}\PY{p}{)}
                    \PY{n}{img\PYZus{}vector} \PY{o}{=} \PY{n+nb+bp}{self}\PY{o}{.}\PY{n}{\PYZus{}normalize}\PY{p}{(}\PY{n}{img\PYZus{}vector}\PY{p}{)}
        
                    \PY{n+nb+bp}{self}\PY{o}{.}\PY{n}{list\PYZus{}label}\PY{p}{[}\PY{n}{\PYZus{}count}\PY{p}{]} \PY{o}{=} \PY{n}{label}
                    \PY{n+nb+bp}{self}\PY{o}{.}\PY{n}{list\PYZus{}image}\PY{p}{[}\PY{p}{:}\PY{p}{,} \PY{n}{\PYZus{}count}\PY{p}{]} \PY{o}{=} \PY{n}{img\PYZus{}vector}  \PY{c+c1}{\PYZsh{} [:, i] get ith col}
                    \PY{n}{\PYZus{}count} \PY{o}{+}\PY{o}{=} \PY{l+m+mi}{1}
\end{Verbatim}


    \emph{normalizing} means transforming the vector so that it has unit
norm.

Here we do the normalizing by formula:
\[z_{i} = \frac{x_{i}-\min{(x)}}{\max{(x)}-\min{(x)}}\]

    \begin{Verbatim}[commandchars=\\\{\}]
{\color{incolor}In [{\color{incolor}4}]:}     \PY{k}{def} \PY{n+nf}{\PYZus{}normalize}\PY{p}{(}\PY{n+nb+bp}{self}\PY{p}{,} \PY{n}{data}\PY{p}{)}\PY{p}{:}
                \PY{l+s+sd}{\PYZdq{}\PYZdq{}\PYZdq{}Normalize the values of the input data into [0, 1]\PYZdq{}\PYZdq{}\PYZdq{}}
                \PY{n}{data\PYZus{}normalized} \PY{o}{=} \PY{p}{(}\PY{n}{data} \PY{o}{\PYZhy{}} \PY{n+nb}{min}\PY{p}{(}\PY{n}{data}\PY{p}{)}\PY{p}{)} \PY{o}{/} \PY{p}{(}\PY{n+nb}{max}\PY{p}{(}\PY{n}{data}\PY{p}{)} \PY{o}{\PYZhy{}} \PY{n+nb}{min}\PY{p}{(}\PY{n}{data}\PY{p}{)}\PY{p}{)}
        
                \PY{k}{return} \PY{n}{data\PYZus{}normalized}
            
        \PY{c+c1}{\PYZsh{} for jupyter notebook}
        \PY{n}{DataSet}\PY{o}{.}\PY{n}{\PYZus{}normalize} \PY{o}{=} \PY{n}{\PYZus{}normalize}
\end{Verbatim}


    \begin{center}\rule{0.5\linewidth}{\linethickness}\end{center}

\subsubsection{k-means algorithm part}\label{k-means-algorithm-part}

Create a class \texttt{KMeans} to group the functions about algorithm
together.

Here the \texttt{\_\_init\_\_} constructor has two parameters:

\begin{itemize}
\tightlist
\item
  \texttt{data}: image data read by \texttt{DataSet}
\item
  \texttt{k}: number of clusters we want
\end{itemize}

    \begin{Verbatim}[commandchars=\\\{\}]
{\color{incolor}In [{\color{incolor}5}]:} \PY{k}{class} \PY{n+nc}{KMeans}\PY{p}{:}
            \PY{l+s+sd}{\PYZdq{}\PYZdq{}\PYZdq{}k\PYZhy{}means algorithm for images}
        
        \PY{l+s+sd}{    Clustring the image datas into k groups.}
        \PY{l+s+sd}{    Compute each iteration\PYZsq{}s energy and accuracy. }
        \PY{l+s+sd}{    \PYZdq{}\PYZdq{}\PYZdq{}}
        
            \PY{k}{def} \PY{n+nf}{\PYZus{}\PYZus{}init\PYZus{}\PYZus{}}\PY{p}{(}\PY{n+nb+bp}{self}\PY{p}{,} \PY{n}{data}\PY{p}{,} \PY{n}{k}\PY{o}{=}\PY{l+m+mi}{10}\PY{p}{)}\PY{p}{:}
                \PY{n+nb+bp}{self}\PY{o}{.}\PY{n}{k} \PY{o}{=} \PY{n}{k}
                \PY{n+nb+bp}{self}\PY{o}{.}\PY{n}{data} \PY{o}{=} \PY{n}{data}
                \PY{n+nb+bp}{self}\PY{o}{.}\PY{n}{energy\PYZus{}history} \PY{o}{=} \PY{p}{[}\PY{p}{]}
                \PY{n+nb+bp}{self}\PY{o}{.}\PY{n}{accuracy\PYZus{}history} \PY{o}{=} \PY{p}{[}\PY{p}{]}
                \PY{n+nb+bp}{self}\PY{o}{.}\PY{n}{centroids} \PY{o}{=} \PY{p}{[}\PY{p}{]}
                \PY{n+nb+bp}{self}\PY{o}{.}\PY{n}{clusters} \PY{o}{=} \PY{p}{[}\PY{p}{[}\PY{p}{]} \PY{k}{for} \PY{n}{\PYZus{}} \PY{o+ow}{in} \PY{n+nb}{range}\PY{p}{(}\PY{n}{k}\PY{p}{)}\PY{p}{]}
                \PY{n+nb+bp}{self}\PY{o}{.}\PY{n}{labels} \PY{o}{=} \PY{p}{[}\PY{p}{]}
\end{Verbatim}


    \paragraph{Define some getter function for plotting
graph.}\label{define-some-getter-function-for-plotting-graph.}

    \begin{Verbatim}[commandchars=\\\{\}]
{\color{incolor}In [{\color{incolor}6}]:}     \PY{k}{def} \PY{n+nf}{getEnergyHistory}\PY{p}{(}\PY{n+nb+bp}{self}\PY{p}{)}\PY{p}{:}
                \PY{k}{return} \PY{n+nb+bp}{self}\PY{o}{.}\PY{n}{energy\PYZus{}history}
        
            \PY{k}{def} \PY{n+nf}{getAccuracyHistory}\PY{p}{(}\PY{n+nb+bp}{self}\PY{p}{)}\PY{p}{:}
                \PY{k}{return} \PY{n+nb+bp}{self}\PY{o}{.}\PY{n}{accuracy\PYZus{}history}
            
        \PY{c+c1}{\PYZsh{} for jupyter notebook}
        \PY{n}{KMeans}\PY{o}{.}\PY{n}{getEnergyHistory} \PY{o}{=} \PY{n}{getEnergyHistory}
        \PY{n}{KMeans}\PY{o}{.}\PY{n}{getAccuracyHistory} \PY{o}{=} \PY{n}{getAccuracyHistory}
\end{Verbatim}


    \paragraph{Define the initialization
function}\label{define-the-initialization-function}

This initialization function will:

\begin{enumerate}
\def\labelenumi{\arabic{enumi}.}
\tightlist
\item
  randomly assign labels to images
\item
  compute the initial centroids
\item
  compute the initial system energy
\item
  compute the initial accuracy
\end{enumerate}

We separate each step into different functions.

    \begin{Verbatim}[commandchars=\\\{\}]
{\color{incolor}In [{\color{incolor}7}]:}     \PY{k}{def} \PY{n+nf}{initialCluster}\PY{p}{(}\PY{n+nb+bp}{self}\PY{p}{)}\PY{p}{:}
                \PY{n+nb+bp}{self}\PY{o}{.}\PY{n}{\PYZus{}initialLabel}\PY{p}{(}\PY{p}{)}
                \PY{n+nb+bp}{self}\PY{o}{.}\PY{n}{\PYZus{}computeCentroid}\PY{p}{(}\PY{p}{)}
                \PY{n+nb+bp}{self}\PY{o}{.}\PY{n}{\PYZus{}computeEnergy}\PY{p}{(}\PY{p}{)}
                \PY{n+nb+bp}{self}\PY{o}{.}\PY{n}{\PYZus{}computeAccuracy}\PY{p}{(}\PY{p}{)}
                
        \PY{c+c1}{\PYZsh{} for jupyter notebook}
        \PY{n}{KMeans}\PY{o}{.}\PY{n}{initialCluster} \PY{o}{=} \PY{n}{initialCluster}
\end{Verbatim}


    Label initialization is very simple.\\
In range \([0,k)\), we assign labels to images in order.

Instead of store the data vector into cluster list, we just append
image's index in \texttt{list\_image}.\\
Use index can let easily get each image's original label.

    \begin{Verbatim}[commandchars=\\\{\}]
{\color{incolor}In [{\color{incolor}8}]:}     \PY{k}{def} \PY{n+nf}{\PYZus{}initialLabel}\PY{p}{(}\PY{n+nb+bp}{self}\PY{p}{)}\PY{p}{:}
                \PY{c+c1}{\PYZsh{} for i in range(100):}
                \PY{k}{for} \PY{n}{i} \PY{o+ow}{in} \PY{n+nb}{range}\PY{p}{(}\PY{n+nb+bp}{self}\PY{o}{.}\PY{n}{data}\PY{o}{.}\PY{n}{num\PYZus{}image}\PY{p}{)}\PY{p}{:}
                    \PY{n}{index} \PY{o}{=} \PY{n}{i} \PY{o}{\PYZpc{}} \PY{n+nb+bp}{self}\PY{o}{.}\PY{n}{k}
                    \PY{n+nb+bp}{self}\PY{o}{.}\PY{n}{clusters}\PY{p}{[}\PY{n}{index}\PY{p}{]}\PY{o}{.}\PY{n}{append}\PY{p}{(}\PY{n}{i}\PY{p}{)}
                    
        \PY{c+c1}{\PYZsh{} for jupyter notebook}
        \PY{n}{KMeans}\PY{o}{.}\PY{n}{\PYZus{}initialLabel} \PY{o}{=} \PY{n}{\PYZus{}initialLabel}
\end{Verbatim}


    We still use vectors' mean value as cluster's centroid.

The compute method is same, but the data structure have been changed.
Here we use nest array to represent a data matrix. Each column of the
matrix is one image's data.

I used the simplest way, traversal the matrix, to access data. It works,
but the time complexity becomes to \(row \cdot col\)

    \begin{Verbatim}[commandchars=\\\{\}]
{\color{incolor}In [{\color{incolor}9}]:}    \PY{k}{def} \PY{n+nf}{\PYZus{}computeCentroid}\PY{p}{(}\PY{n+nb+bp}{self}\PY{p}{)}\PY{p}{:}
                \PY{l+s+sd}{\PYZdq{}\PYZdq{}\PYZdq{} Compute each groups centroid then update self.controids \PYZdq{}\PYZdq{}\PYZdq{}}
        
                \PY{n}{new\PYZus{}centroids} \PY{o}{=} \PY{p}{[}\PY{p}{]}
                \PY{n}{n} \PY{o}{=} \PY{n+nb+bp}{self}\PY{o}{.}\PY{n}{data}\PY{o}{.}\PY{n}{len\PYZus{}vec}
        
                \PY{k}{for} \PY{n}{cluster} \PY{o+ow}{in} \PY{n+nb+bp}{self}\PY{o}{.}\PY{n}{clusters}\PY{p}{:}
                    \PY{n}{centroid} \PY{o}{=} \PY{p}{[}\PY{l+m+mi}{0}\PY{p}{]} \PY{o}{*} \PY{n}{n}
                    \PY{n}{num\PYZus{}elements} \PY{o}{=} \PY{l+m+mi}{0}
        
                    \PY{c+c1}{\PYZsh{} sum of the cluster\PYZsq{}s elements}
                    \PY{k}{for} \PY{n}{img} \PY{o+ow}{in} \PY{n}{cluster}\PY{p}{:}
                        \PY{n}{num\PYZus{}elements} \PY{o}{+}\PY{o}{=} \PY{l+m+mi}{1}
                        \PY{k}{for} \PY{n}{row} \PY{o+ow}{in} \PY{n+nb}{range}\PY{p}{(}\PY{n}{n}\PY{p}{)}\PY{p}{:}
                            \PY{n}{centroid}\PY{p}{[}\PY{n}{row}\PY{p}{]} \PY{o}{+}\PY{o}{=} \PY{n+nb+bp}{self}\PY{o}{.}\PY{n}{data}\PY{o}{.}\PY{n}{list\PYZus{}image}\PY{p}{[}\PY{n}{row}\PY{p}{]}\PY{p}{[}\PY{n}{img}\PY{p}{]}
        
                    \PY{n}{centroid} \PY{o}{=} \PY{p}{[}\PY{n}{row}\PY{o}{/}\PY{n}{num\PYZus{}elements} \PY{k}{for} \PY{n}{row} \PY{o+ow}{in} \PY{n}{centroid}\PY{p}{]}
                    \PY{n}{new\PYZus{}centroids}\PY{o}{.}\PY{n}{append}\PY{p}{(}\PY{n}{centroid}\PY{p}{)}
        
                \PY{n+nb+bp}{self}\PY{o}{.}\PY{n}{centroids} \PY{o}{=} \PY{n}{new\PYZus{}centroids}
                
        \PY{c+c1}{\PYZsh{} for jupyter notebook}
        \PY{n}{KMeans}\PY{o}{.}\PY{n}{\PYZus{}computeCentroid} \PY{o}{=} \PY{n}{\PYZus{}computeCentroid}
\end{Verbatim}


    The \textbf{energy} computation is same as before: \[
J^{clust}=(||x_{1}-z_{c_{1}}||^{2}+ \cdots +||x_{N}-z_{c_{N}}||^{2})/N
\]

Because of the data representation, we use
\texttt{list\_image{[}:,\ index{]}} this syntex to get the column of the
matrix.

    \begin{Verbatim}[commandchars=\\\{\}]
{\color{incolor}In [{\color{incolor}11}]:} \PY{k}{def} \PY{n+nf}{\PYZus{}computeEnergy}\PY{p}{(}\PY{n+nb+bp}{self}\PY{p}{)}\PY{p}{:}
                 \PY{l+s+sd}{\PYZdq{}\PYZdq{}\PYZdq{} Compute the cost of the clustering result}
         
         \PY{l+s+sd}{        Return:}
         \PY{l+s+sd}{            energy(float): the energy of this clustering.}
         \PY{l+s+sd}{        \PYZdq{}\PYZdq{}\PYZdq{}}
         
                 \PY{n}{energy} \PY{o}{=} \PY{l+m+mi}{0}
                 \PY{k}{for} \PY{n}{i} \PY{o+ow}{in} \PY{n+nb}{range}\PY{p}{(}\PY{n+nb}{len}\PY{p}{(}\PY{n+nb+bp}{self}\PY{o}{.}\PY{n}{centroids}\PY{p}{)}\PY{p}{)}\PY{p}{:}
                     \PY{n}{centroid} \PY{o}{=} \PY{n+nb+bp}{self}\PY{o}{.}\PY{n}{centroids}\PY{p}{[}\PY{n}{i}\PY{p}{]}
                     \PY{n}{part\PYZus{}energy} \PY{o}{=} \PY{l+m+mi}{0}
                     \PY{k}{for} \PY{n}{index} \PY{o+ow}{in} \PY{n+nb+bp}{self}\PY{o}{.}\PY{n}{clusters}\PY{p}{[}\PY{n}{i}\PY{p}{]}\PY{p}{:}
                         \PY{n}{part\PYZus{}energy} \PY{o}{+}\PY{o}{=} \PY{n+nb+bp}{self}\PY{o}{.}\PY{n}{\PYZus{}computeDistance}\PY{p}{(}
                             \PY{n}{x}\PY{o}{=}\PY{n+nb+bp}{self}\PY{o}{.}\PY{n}{data}\PY{o}{.}\PY{n}{list\PYZus{}image}\PY{p}{[}\PY{p}{:}\PY{p}{,} \PY{n}{index}\PY{p}{]}\PY{p}{,} \PY{n}{y}\PY{o}{=}\PY{n}{centroid}\PY{p}{)}
                     \PY{n}{energy} \PY{o}{+}\PY{o}{=} \PY{n}{part\PYZus{}energy}
                 \PY{n}{energy} \PY{o}{=} \PY{n}{energy} \PY{o}{/} \PY{n+nb+bp}{self}\PY{o}{.}\PY{n}{data}\PY{o}{.}\PY{n}{num\PYZus{}image}
                 \PY{n+nb+bp}{self}\PY{o}{.}\PY{n}{energy\PYZus{}history}\PY{o}{.}\PY{n}{append}\PY{p}{(}\PY{n}{energy}\PY{p}{)}
         
                 \PY{k}{return} \PY{n}{energy}
             
         \PY{c+c1}{\PYZsh{} for jupyter notebook}
         \PY{n}{KMeans}\PY{o}{.}\PY{n}{\PYZus{}computeEnergy} \PY{o}{=} \PY{n}{\PYZus{}computeEnergy}
\end{Verbatim}


    The \textbf{accuracy} is the ratio of the most frequently occurring
elements in the group.\\
So we need to iterate all the elements then find the most frequently
occurring image and count the number of occurrences.

    \begin{Verbatim}[commandchars=\\\{\}]
{\color{incolor}In [{\color{incolor}12}]:} \PY{k}{def} \PY{n+nf}{\PYZus{}computeAccuracy}\PY{p}{(}\PY{n+nb+bp}{self}\PY{p}{)}\PY{p}{:}
                 \PY{l+s+sd}{\PYZdq{}\PYZdq{}\PYZdq{}Compute the arrcuracy of the result}
         
         \PY{l+s+sd}{        In each group, let the largest number of elements to be the group label.}
         \PY{l+s+sd}{        \PYZdq{}\PYZdq{}\PYZdq{}}
         
                 \PY{n}{accuracy} \PY{o}{=} \PY{l+m+mi}{0}
                 \PY{n}{labels\PYZus{}clusters} \PY{o}{=} \PY{p}{[}\PY{p}{]}
         
                 \PY{k}{for} \PY{n}{cluster} \PY{o+ow}{in} \PY{n+nb+bp}{self}\PY{o}{.}\PY{n}{clusters}\PY{p}{:}
                     \PY{n}{labels} \PY{o}{=} \PY{p}{[}\PY{p}{]}
         
                     \PY{c+c1}{\PYZsh{} get all the labels in the cluster}
                     \PY{k}{for} \PY{n}{index} \PY{o+ow}{in} \PY{n}{cluster}\PY{p}{:}
                         \PY{n}{labels}\PY{o}{.}\PY{n}{append}\PY{p}{(}\PY{n+nb+bp}{self}\PY{o}{.}\PY{n}{data}\PY{o}{.}\PY{n}{list\PYZus{}label}\PY{p}{[}\PY{n}{index}\PY{p}{]}\PY{p}{)}
         
                     \PY{c+c1}{\PYZsh{} get the largest number of lable, count the occurences}
                     \PY{n}{labels}\PY{o}{.}\PY{n}{sort}\PY{p}{(}\PY{p}{)}
                     \PY{n}{count} \PY{o}{=} \PY{l+m+mi}{0}
                     \PY{n}{count\PYZus{}max} \PY{o}{=} \PY{l+m+mi}{0}
                     \PY{n}{label\PYZus{}prev} \PY{o}{=} \PY{o}{\PYZhy{}}\PY{l+m+mi}{1}
                     \PY{n}{label\PYZus{}max} \PY{o}{=} \PY{o}{\PYZhy{}}\PY{l+m+mi}{1}
                     \PY{k}{for} \PY{n}{label} \PY{o+ow}{in} \PY{n}{labels}\PY{p}{:}
                         \PY{k}{if} \PY{n}{label} \PY{o}{==} \PY{n}{label\PYZus{}prev}\PY{p}{:}
                             \PY{n}{count} \PY{o}{+}\PY{o}{=} \PY{l+m+mi}{1}
                         \PY{k}{else}\PY{p}{:}
                             \PY{k}{if} \PY{n}{count} \PY{o}{\PYZgt{}} \PY{n}{count\PYZus{}max}\PY{p}{:}
                                 \PY{n}{count\PYZus{}max} \PY{o}{=} \PY{n}{count}
                                 \PY{n}{label\PYZus{}max} \PY{o}{=} \PY{n}{label\PYZus{}prev}
                             \PY{n}{label\PYZus{}prev} \PY{o}{=} \PY{n}{label}
                             \PY{n}{count} \PY{o}{=} \PY{l+m+mi}{1}
                     \PY{c+c1}{\PYZsh{} check the last item}
                     \PY{k}{if} \PY{n}{count} \PY{o}{\PYZgt{}} \PY{n}{count\PYZus{}max}\PY{p}{:}
                         \PY{n}{count\PYZus{}max} \PY{o}{=} \PY{n}{count}
                         \PY{n}{label\PYZus{}max} \PY{o}{=} \PY{n}{label\PYZus{}prev}
         
                     \PY{n}{labels\PYZus{}clusters}\PY{o}{.}\PY{n}{append}\PY{p}{(}\PY{n}{label\PYZus{}max}\PY{p}{)}
                     \PY{n}{accuracy\PYZus{}part} \PY{o}{=} \PY{n}{count\PYZus{}max} \PY{o}{/} \PY{n+nb}{len}\PY{p}{(}\PY{n}{cluster}\PY{p}{)}
                     \PY{n}{accuracy} \PY{o}{+}\PY{o}{=} \PY{n}{accuracy\PYZus{}part}
         
                 \PY{n}{accuracy} \PY{o}{=} \PY{n}{accuracy} \PY{o}{/} \PY{n+nb}{len}\PY{p}{(}\PY{n+nb+bp}{self}\PY{o}{.}\PY{n}{clusters}\PY{p}{)}
                 \PY{n+nb+bp}{self}\PY{o}{.}\PY{n}{labels} \PY{o}{=} \PY{n}{labels\PYZus{}clusters}
                 \PY{n+nb+bp}{self}\PY{o}{.}\PY{n}{accuracy\PYZus{}history}\PY{o}{.}\PY{n}{append}\PY{p}{(}\PY{n}{accuracy}\PY{p}{)}
                 
         \PY{c+c1}{\PYZsh{} for jupyter notebook}
         \PY{n}{KMeans}\PY{o}{.}\PY{n}{\PYZus{}computeAccuracy} \PY{o}{=} \PY{n}{\PYZus{}computeAccuracy}
\end{Verbatim}


    \begin{center}\rule{0.5\linewidth}{\linethickness}\end{center}

\subsubsection{Trigger funtions}\label{trigger-funtions}

In trigger function \texttt{run}, we define the condition to control
when to stop our algorithm.

Here we let the algorithm keep running until the \texttt{energy} not
change.\\
Also computhe \emph{energy} and \emph{accuracy} of each iteration.

    \begin{Verbatim}[commandchars=\\\{\}]
{\color{incolor}In [{\color{incolor}13}]:}     \PY{k}{def} \PY{n+nf}{run}\PY{p}{(}\PY{n+nb+bp}{self}\PY{p}{)}\PY{p}{:}
                 \PY{l+s+sd}{\PYZdq{}\PYZdq{}\PYZdq{}Run algorithm}
         
         \PY{l+s+sd}{        Repeatly assign labels to points and compute new centroids.}
         \PY{l+s+sd}{        Iterate until the energy of the result not change.}
         \PY{l+s+sd}{        \PYZdq{}\PYZdq{}\PYZdq{}}
         
                 \PY{n}{energy\PYZus{}prev} \PY{o}{=} \PY{l+m+mi}{0}
                 \PY{n}{energy\PYZus{}this} \PY{o}{=} \PY{n+nb+bp}{self}\PY{o}{.}\PY{n}{energy\PYZus{}history}\PY{p}{[}\PY{l+m+mi}{0}\PY{p}{]}
                 \PY{k}{while}\PY{p}{(}\PY{n}{energy\PYZus{}this} \PY{o}{!=} \PY{n}{energy\PYZus{}prev}\PY{p}{)}\PY{p}{:}
                     \PY{n}{energy\PYZus{}prev} \PY{o}{=} \PY{n}{energy\PYZus{}this}
                     \PY{n}{energy\PYZus{}this} \PY{o}{=} \PY{n+nb+bp}{self}\PY{o}{.}\PY{n}{\PYZus{}clustering}\PY{p}{(}\PY{p}{)}
                     \PY{n+nb+bp}{self}\PY{o}{.}\PY{n}{\PYZus{}computeAccuracy}\PY{p}{(}\PY{p}{)}
                     
         \PY{c+c1}{\PYZsh{} for jupyter notebook}
         \PY{n}{KMeans}\PY{o}{.}\PY{n}{run} \PY{o}{=} \PY{n}{run}
\end{Verbatim}


    Inside the \texttt{\_clustering} function, it call
\texttt{\_assignLabel} and \texttt{\_computeCentroid} one time to
generate new result.

Here we add one step after \texttt{\_assignLabel} to remove the empty
clusters.

    \begin{Verbatim}[commandchars=\\\{\}]
{\color{incolor}In [{\color{incolor}15}]:}     \PY{k}{def} \PY{n+nf}{\PYZus{}clustering}\PY{p}{(}\PY{n+nb+bp}{self}\PY{p}{)}\PY{p}{:}
                 \PY{l+s+sd}{\PYZdq{}\PYZdq{}\PYZdq{}Run algorithm one iteration}
         
         \PY{l+s+sd}{        Return:}
         \PY{l+s+sd}{            energy: This iteration\PYZsq{}s result energy}
         \PY{l+s+sd}{        \PYZdq{}\PYZdq{}\PYZdq{}}
         
                 \PY{n+nb+bp}{self}\PY{o}{.}\PY{n}{\PYZus{}assignLabel}\PY{p}{(}\PY{p}{)}
                 \PY{c+c1}{\PYZsh{} After re\PYZhy{}grouping if there are any empty cluster, delete it from list}
                 \PY{n+nb+bp}{self}\PY{o}{.}\PY{n}{clusters} \PY{o}{=} \PY{p}{[}\PY{n+nb}{list} \PY{k}{for} \PY{n+nb}{list} \PY{o+ow}{in} \PY{n+nb+bp}{self}\PY{o}{.}\PY{n}{clusters} \PY{k}{if} \PY{n+nb}{list}\PY{p}{]}
                 \PY{n+nb+bp}{self}\PY{o}{.}\PY{n}{\PYZus{}computeCentroid}\PY{p}{(}\PY{p}{)}
         
                 \PY{k}{return} \PY{n+nb+bp}{self}\PY{o}{.}\PY{n}{\PYZus{}computeEnergy}\PY{p}{(}\PY{p}{)}
             
         \PY{c+c1}{\PYZsh{} for jupyter notebook}
         \PY{n}{KMeans}\PY{o}{.}\PY{n}{\PYZus{}clustering} \PY{o}{=} \PY{n}{\PYZus{}clustering}
\end{Verbatim}


    Assign new label to images depends on the Euclidean distance between
each image vector and its centroid vector.

    \begin{Verbatim}[commandchars=\\\{\}]
{\color{incolor}In [{\color{incolor}16}]:}     \PY{k}{def} \PY{n+nf}{\PYZus{}assignLabel}\PY{p}{(}\PY{n+nb+bp}{self}\PY{p}{)}\PY{p}{:}
                 \PY{l+s+sd}{\PYZdq{}\PYZdq{}\PYZdq{} Assign labels to elemetns for generating new groups}
         
         \PY{l+s+sd}{        Compute distance between each element with each centroid,}
         \PY{l+s+sd}{        assign it to the closest centroid\PYZsq{}s group.}
         \PY{l+s+sd}{        \PYZdq{}\PYZdq{}\PYZdq{}}
         
                 \PY{c+c1}{\PYZsh{} for each element, compute the distance, get the closest centroid}
                 \PY{c+c1}{\PYZsh{} generate k new cluster}
                 \PY{n}{new\PYZus{}clusters} \PY{o}{=} \PY{p}{[}\PY{p}{[}\PY{p}{]} \PY{k}{for} \PY{n}{\PYZus{}} \PY{o+ow}{in} \PY{n+nb}{range}\PY{p}{(}\PY{n+nb}{len}\PY{p}{(}\PY{n+nb+bp}{self}\PY{o}{.}\PY{n}{clusters}\PY{p}{)}\PY{p}{)}\PY{p}{]}
         
                 \PY{k}{for} \PY{n}{index} \PY{o+ow}{in} \PY{n+nb}{range}\PY{p}{(}\PY{n+nb+bp}{self}\PY{o}{.}\PY{n}{data}\PY{o}{.}\PY{n}{num\PYZus{}image}\PY{p}{)}\PY{p}{:}
                     \PY{n+nb}{min} \PY{o}{=} \PY{n}{math}\PY{o}{.}\PY{n}{inf}
                     \PY{n}{closest} \PY{o}{=} \PY{l+m+mi}{0}
                     \PY{c+c1}{\PYZsh{} find the closest centroid}
                     \PY{k}{for} \PY{n}{j} \PY{o+ow}{in} \PY{n+nb}{range}\PY{p}{(}\PY{n+nb}{len}\PY{p}{(}\PY{n+nb+bp}{self}\PY{o}{.}\PY{n}{centroids}\PY{p}{)}\PY{p}{)}\PY{p}{:}
                         \PY{n}{dist} \PY{o}{=} \PY{n+nb+bp}{self}\PY{o}{.}\PY{n}{\PYZus{}computeDistance}\PY{p}{(}
                             \PY{n}{x}\PY{o}{=}\PY{n+nb+bp}{self}\PY{o}{.}\PY{n}{data}\PY{o}{.}\PY{n}{list\PYZus{}image}\PY{p}{[}\PY{p}{:}\PY{p}{,} \PY{n}{index}\PY{p}{]}\PY{p}{,} \PY{n}{y}\PY{o}{=}\PY{n+nb+bp}{self}\PY{o}{.}\PY{n}{centroids}\PY{p}{[}\PY{n}{j}\PY{p}{]}\PY{p}{)}
                         \PY{k}{if} \PY{n}{dist} \PY{o}{\PYZlt{}} \PY{n+nb}{min}\PY{p}{:}
                             \PY{n+nb}{min} \PY{o}{=} \PY{n}{dist}
                             \PY{n}{closest} \PY{o}{=} \PY{n}{j}
                     \PY{c+c1}{\PYZsh{} put point into new group}
                     \PY{n}{new\PYZus{}clusters}\PY{p}{[}\PY{n}{closest}\PY{p}{]}\PY{o}{.}\PY{n}{append}\PY{p}{(}\PY{n}{index}\PY{p}{)}
         
                 \PY{n+nb+bp}{self}\PY{o}{.}\PY{n}{clusters} \PY{o}{=} \PY{n}{new\PYZus{}clusters}
                 
         \PY{c+c1}{\PYZsh{} for jupyter notebook}
         \PY{n}{KMeans}\PY{o}{.}\PY{n}{\PYZus{}assignLabel} \PY{o}{=} \PY{n}{\PYZus{}assignLabel}
\end{Verbatim}


    \emph{Euclidean distance} function.\\
Since we just use this value to do the comparison, we do not need the
square root value.

    \begin{Verbatim}[commandchars=\\\{\}]
{\color{incolor}In [{\color{incolor}17}]:}     \PY{k}{def} \PY{n+nf}{\PYZus{}computeDistance}\PY{p}{(}\PY{n+nb+bp}{self}\PY{p}{,} \PY{n}{x}\PY{p}{,} \PY{n}{y}\PY{p}{)}\PY{p}{:}
                 \PY{l+s+sd}{\PYZdq{}\PYZdq{}\PYZdq{}Compute the distance between two points}
         
         \PY{l+s+sd}{        (x1\PYZhy{}x2)\PYZca{}2 + (y1\PYZhy{}y2)\PYZca{}2}
         \PY{l+s+sd}{        \PYZdq{}\PYZdq{}\PYZdq{}}
         
                 \PY{c+c1}{\PYZsh{} convert python list to np.array}
                 \PY{n}{x} \PY{o}{=} \PY{n}{np}\PY{o}{.}\PY{n}{array}\PY{p}{(}\PY{n}{x}\PY{p}{)}
                 \PY{n}{y} \PY{o}{=} \PY{n}{np}\PY{o}{.}\PY{n}{array}\PY{p}{(}\PY{n}{y}\PY{p}{)}
         
                 \PY{n}{d} \PY{o}{=} \PY{p}{(}\PY{n}{x} \PY{o}{\PYZhy{}} \PY{n}{y}\PY{p}{)}\PY{o}{*}\PY{o}{*}\PY{l+m+mi}{2}
                 \PY{n}{s} \PY{o}{=} \PY{n}{np}\PY{o}{.}\PY{n}{sum}\PY{p}{(}\PY{n}{d}\PY{p}{)}
         
                 \PY{k}{return} \PY{n}{s}
             
         \PY{c+c1}{\PYZsh{} for jupyter notebook}
         \PY{n}{KMeans}\PY{o}{.}\PY{n}{\PYZus{}computeDistance} \PY{o}{=} \PY{n}{\PYZus{}computeDistance}
\end{Verbatim}


    \subsubsection{Plotting functions}\label{plotting-functions}

We define two plotting functions here.

\begin{itemize}
\tightlist
\item
  \texttt{plotImages}: plot the clustering result images
\item
  \texttt{plotChanges}: plot each iteration's energy and accuracy
  variations
\end{itemize}

    \begin{Verbatim}[commandchars=\\\{\}]
{\color{incolor}In [{\color{incolor}44}]:} \PY{k}{def} \PY{n+nf}{plotImages}\PY{p}{(}\PY{n}{kmeans}\PY{p}{)}\PY{p}{:}
             \PY{l+s+sd}{\PYZdq{}\PYZdq{}\PYZdq{}Plot the graph of clusters average value\PYZdq{}\PYZdq{}\PYZdq{}}
             \PY{n}{k} \PY{o}{=} \PY{n+nb}{len}\PY{p}{(}\PY{n}{kmeans}\PY{o}{.}\PY{n}{clusters}\PY{p}{)}
             \PY{n}{size} \PY{o}{=} \PY{n}{kmeans}\PY{o}{.}\PY{n}{data}\PY{o}{.}\PY{n}{len\PYZus{}vec}
         
             \PY{n}{im\PYZus{}average} \PY{o}{=} \PY{n}{np}\PY{o}{.}\PY{n}{zeros}\PY{p}{(}\PY{p}{(}\PY{n}{size}\PY{p}{,} \PY{n}{k}\PY{p}{)}\PY{p}{,} \PY{n}{dtype}\PY{o}{=}\PY{n+nb}{float}\PY{p}{)}
             \PY{n}{im\PYZus{}count} \PY{o}{=} \PY{n}{np}\PY{o}{.}\PY{n}{zeros}\PY{p}{(}\PY{n}{k}\PY{p}{,} \PY{n}{dtype}\PY{o}{=}\PY{n+nb}{int}\PY{p}{)}
         
             \PY{c+c1}{\PYZsh{} add each cluster\PYZsq{}s image value together}
             \PY{k}{for} \PY{n}{i} \PY{o+ow}{in} \PY{n+nb}{range}\PY{p}{(}\PY{n}{k}\PY{p}{)}\PY{p}{:}
                 \PY{k}{for} \PY{n}{img} \PY{o+ow}{in} \PY{n}{kmeans}\PY{o}{.}\PY{n}{clusters}\PY{p}{[}\PY{n}{i}\PY{p}{]}\PY{p}{:}
                     \PY{n}{im\PYZus{}average}\PY{p}{[}\PY{p}{:}\PY{p}{,} \PY{n}{i}\PY{p}{]} \PY{o}{+}\PY{o}{=} \PY{n}{kmeans}\PY{o}{.}\PY{n}{data}\PY{o}{.}\PY{n}{list\PYZus{}image}\PY{p}{[}\PY{p}{:}\PY{p}{,} \PY{n}{img}\PY{p}{]}
         
             \PY{c+c1}{\PYZsh{} count number of each cluster\PYZsq{}s elements}
             \PY{k}{for} \PY{n}{i} \PY{o+ow}{in} \PY{n+nb}{range}\PY{p}{(}\PY{n}{k}\PY{p}{)}\PY{p}{:}
                 \PY{n}{im\PYZus{}count}\PY{p}{[}\PY{n}{i}\PY{p}{]} \PY{o}{=} \PY{n+nb}{len}\PY{p}{(}\PY{n}{kmeans}\PY{o}{.}\PY{n}{clusters}\PY{p}{[}\PY{n}{i}\PY{p}{]}\PY{p}{)}
         
             \PY{k}{for} \PY{n}{i} \PY{o+ow}{in} \PY{n+nb}{range}\PY{p}{(}\PY{n}{k}\PY{p}{)}\PY{p}{:}
                 \PY{n}{im\PYZus{}average}\PY{p}{[}\PY{p}{:}\PY{p}{,} \PY{n}{i}\PY{p}{]} \PY{o}{/}\PY{o}{=} \PY{n}{im\PYZus{}count}\PY{p}{[}\PY{n}{i}\PY{p}{]}
         
                 \PY{n}{plt}\PY{o}{.}\PY{n}{subplot}\PY{p}{(}\PY{l+m+mi}{2}\PY{p}{,} \PY{l+m+mi}{5}\PY{p}{,} \PY{n}{i}\PY{o}{+}\PY{l+m+mi}{1}\PY{p}{)}
                 \PY{n}{plt}\PY{o}{.}\PY{n}{title}\PY{p}{(}\PY{n}{kmeans}\PY{o}{.}\PY{n}{labels}\PY{p}{[}\PY{n}{i}\PY{p}{]}\PY{p}{)}
                 \PY{n}{plt}\PY{o}{.}\PY{n}{imshow}\PY{p}{(}\PY{n}{im\PYZus{}average}\PY{p}{[}\PY{p}{:}\PY{p}{,} \PY{n}{i}\PY{p}{]}\PY{o}{.}\PY{n}{reshape}\PY{p}{(}
                                                     \PY{p}{(}\PY{n}{kmeans}\PY{o}{.}\PY{n}{data}\PY{o}{.}\PY{n}{size\PYZus{}row}\PY{p}{,}
                                                      \PY{n}{kmeans}\PY{o}{.}\PY{n}{data}\PY{o}{.}\PY{n}{size\PYZus{}col}\PY{p}{)}\PY{p}{)}\PY{p}{,}
                            \PY{n}{cmap}\PY{o}{=}\PY{l+s+s1}{\PYZsq{}}\PY{l+s+s1}{Greys}\PY{l+s+s1}{\PYZsq{}}\PY{p}{,} \PY{n}{interpolation}\PY{o}{=}\PY{l+s+s1}{\PYZsq{}}\PY{l+s+s1}{None}\PY{l+s+s1}{\PYZsq{}}\PY{p}{)}
         
                 \PY{n}{frame} \PY{o}{=} \PY{n}{plt}\PY{o}{.}\PY{n}{gca}\PY{p}{(}\PY{p}{)}
                 \PY{n}{frame}\PY{o}{.}\PY{n}{axes}\PY{o}{.}\PY{n}{get\PYZus{}xaxis}\PY{p}{(}\PY{p}{)}\PY{o}{.}\PY{n}{set\PYZus{}visible}\PY{p}{(}\PY{k+kc}{False}\PY{p}{)}
                 \PY{n}{frame}\PY{o}{.}\PY{n}{axes}\PY{o}{.}\PY{n}{get\PYZus{}yaxis}\PY{p}{(}\PY{p}{)}\PY{o}{.}\PY{n}{set\PYZus{}visible}\PY{p}{(}\PY{k+kc}{False}\PY{p}{)}
         
             \PY{n}{plt}\PY{o}{.}\PY{n}{show}\PY{p}{(}\PY{p}{)}
             
         
         \PY{k}{def} \PY{n+nf}{plotChanges}\PY{p}{(}\PY{n}{energies}\PY{p}{,} \PY{n}{accuracis}\PY{p}{)}\PY{p}{:}
             \PY{l+s+sd}{\PYZdq{}\PYZdq{}\PYZdq{}Plot energy \PYZam{} accuracy history}
         
         \PY{l+s+sd}{    Parameter:}
         \PY{l+s+sd}{        energies(list): a list of energy}
         \PY{l+s+sd}{        accuracis(list): a list of accuracy}
         \PY{l+s+sd}{    \PYZdq{}\PYZdq{}\PYZdq{}}
         
             \PY{n}{plt}\PY{o}{.}\PY{n}{rc}\PY{p}{(}\PY{l+s+s1}{\PYZsq{}}\PY{l+s+s1}{text}\PY{l+s+s1}{\PYZsq{}}\PY{p}{,} \PY{n}{usetex}\PY{o}{=}\PY{k+kc}{True}\PY{p}{)}
             \PY{n}{plt}\PY{o}{.}\PY{n}{rc}\PY{p}{(}\PY{l+s+s1}{\PYZsq{}}\PY{l+s+s1}{font}\PY{l+s+s1}{\PYZsq{}}\PY{p}{,} \PY{n}{family}\PY{o}{=}\PY{l+s+s1}{\PYZsq{}}\PY{l+s+s1}{serif}\PY{l+s+s1}{\PYZsq{}}\PY{p}{)}
         
             \PY{n}{fig} \PY{o}{=} \PY{n}{plt}\PY{o}{.}\PY{n}{figure}\PY{p}{(}\PY{p}{)}
             
             \PY{n}{ite} \PY{o}{=} \PY{p}{[}\PY{n}{i} \PY{k}{for} \PY{n}{i} \PY{o+ow}{in} \PY{n}{np}\PY{o}{.}\PY{n}{arange}\PY{p}{(}\PY{l+m+mi}{1}\PY{p}{,} \PY{n+nb}{len}\PY{p}{(}\PY{n}{energies}\PY{p}{)}\PY{o}{+}\PY{l+m+mi}{1}\PY{p}{,} \PY{l+m+mi}{1}\PY{p}{)}\PY{p}{]}
                 
             \PY{n}{ax1} \PY{o}{=} \PY{n}{fig}\PY{o}{.}\PY{n}{add\PYZus{}subplot}\PY{p}{(}\PY{l+m+mi}{111}\PY{p}{)}
             \PY{n}{ax1}\PY{o}{.}\PY{n}{set\PYZus{}title}\PY{p}{(}\PY{l+s+s1}{\PYZsq{}}\PY{l+s+s1}{Energy and Accuracy Variasion}\PY{l+s+s1}{\PYZsq{}}\PY{p}{)}
             \PY{n}{ax1}\PY{o}{.}\PY{n}{set\PYZus{}ylabel}\PY{p}{(}\PY{l+s+sa}{r}\PY{l+s+s1}{\PYZsq{}}\PY{l+s+s1}{Energy \PYZdl{}J\PYZca{}}\PY{l+s+si}{\PYZob{}clust\PYZcb{}}\PY{l+s+s1}{\PYZdl{}}\PY{l+s+s1}{\PYZsq{}}\PY{p}{)}
             \PY{n}{ax1}\PY{o}{.}\PY{n}{set\PYZus{}xticks}\PY{p}{(}\PY{n}{np}\PY{o}{.}\PY{n}{arange}\PY{p}{(}\PY{l+m+mi}{1}\PY{p}{,} \PY{n+nb}{len}\PY{p}{(}\PY{n}{energies}\PY{p}{)}\PY{o}{+}\PY{l+m+mi}{1}\PY{p}{,} \PY{n}{step}\PY{o}{=}\PY{l+m+mi}{5}\PY{p}{)}\PY{p}{)}
             \PY{n}{ax1}\PY{o}{.}\PY{n}{plot}\PY{p}{(}\PY{n}{ite}\PY{p}{,} \PY{n}{energies}\PY{p}{,} \PY{l+s+s1}{\PYZsq{}}\PY{l+s+s1}{.\PYZhy{}}\PY{l+s+s1}{\PYZsq{}}\PY{p}{,} \PY{n}{color}\PY{o}{=}\PY{l+s+s1}{\PYZsq{}}\PY{l+s+s1}{b}\PY{l+s+s1}{\PYZsq{}}\PY{p}{)}
             
             \PY{n}{ax2} \PY{o}{=} \PY{n}{ax1}\PY{o}{.}\PY{n}{twinx}\PY{p}{(}\PY{p}{)}
             \PY{n}{ax2}\PY{o}{.}\PY{n}{set\PYZus{}ylabel}\PY{p}{(}\PY{l+s+s1}{\PYZsq{}}\PY{l+s+s1}{Accuray}\PY{l+s+s1}{\PYZsq{}}\PY{p}{)}
             \PY{n}{ax2}\PY{o}{.}\PY{n}{get\PYZus{}xaxis}\PY{p}{(}\PY{p}{)}\PY{o}{.}\PY{n}{set\PYZus{}visible}\PY{p}{(}\PY{k+kc}{False}\PY{p}{)}
             \PY{n}{ax2}\PY{o}{.}\PY{n}{plot}\PY{p}{(}\PY{n}{ite}\PY{p}{,} \PY{n}{accuracis}\PY{p}{,} \PY{l+s+s1}{\PYZsq{}}\PY{l+s+s1}{.\PYZhy{}}\PY{l+s+s1}{\PYZsq{}}\PY{p}{,} \PY{n}{color}\PY{o}{=}\PY{l+s+s1}{\PYZsq{}}\PY{l+s+s1}{r}\PY{l+s+s1}{\PYZsq{}}\PY{p}{)}
             
             \PY{n}{plt}\PY{o}{.}\PY{n}{show}\PY{p}{(}\PY{p}{)}
             
\end{Verbatim}


    \subsection{Result}\label{result}

Now let run the program to check the result.

\subsubsection{K=10}\label{k10}

First, let's try when \texttt{k=10}:

\paragraph{1. Initial Images:}\label{initial-images}

    \begin{Verbatim}[commandchars=\\\{\}]
{\color{incolor}In [{\color{incolor}21}]:} \PY{k}{if} \PY{n+nv+vm}{\PYZus{}\PYZus{}name\PYZus{}\PYZus{}} \PY{o}{==} \PY{l+s+s1}{\PYZsq{}}\PY{l+s+s1}{\PYZus{}\PYZus{}main\PYZus{}\PYZus{}}\PY{l+s+s1}{\PYZsq{}}\PY{p}{:}
             \PY{n}{data} \PY{o}{=} \PY{n}{DataSet}\PY{p}{(}\PY{p}{)}
             \PY{n}{kmeans} \PY{o}{=} \PY{n}{KMeans}\PY{p}{(}\PY{n}{data}\PY{p}{,} \PY{l+m+mi}{10}\PY{p}{)}
             \PY{n}{kmeans}\PY{o}{.}\PY{n}{initialCluster}\PY{p}{(}\PY{p}{)}
             \PY{n}{plotImages}\PY{p}{(}\PY{n}{kmeans}\PY{p}{)}
         
             \PY{n}{kmeans}\PY{o}{.}\PY{n}{run}\PY{p}{(}\PY{p}{)}
             
\end{Verbatim}


    \begin{center}
    \adjustimage{max size={0.9\linewidth}{0.9\paperheight}}{output_32_0.png}
    \end{center}
    { \hspace*{\fill} \\}
    
    \paragraph{2. Final Images}\label{final-images}

    \begin{Verbatim}[commandchars=\\\{\}]
{\color{incolor}In [{\color{incolor}22}]:} \PY{n}{plotImages}\PY{p}{(}\PY{n}{kmeans}\PY{p}{)}
\end{Verbatim}


    \begin{center}
    \adjustimage{max size={0.9\linewidth}{0.9\paperheight}}{output_34_0.png}
    \end{center}
    { \hspace*{\fill} \\}
    
    \paragraph{3. Energy and Accuray
changes}\label{energy-and-accuray-changes}

    \begin{Verbatim}[commandchars=\\\{\}]
{\color{incolor}In [{\color{incolor}45}]:} \PY{n}{plotChanges}\PY{p}{(}\PY{n}{kmeans}\PY{o}{.}\PY{n}{getEnergyHistory}\PY{p}{(}\PY{p}{)}\PY{p}{,} \PY{n}{kmeans}\PY{o}{.}\PY{n}{getAccuracyHistory}\PY{p}{(}\PY{p}{)}\PY{p}{)}
\end{Verbatim}


    \begin{center}
    \adjustimage{max size={0.9\linewidth}{0.9\paperheight}}{output_36_0.png}
    \end{center}
    { \hspace*{\fill} \\}
    

    % Add a bibliography block to the postdoc
    
    
    
    \end{document}
