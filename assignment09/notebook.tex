
% Default to the notebook output style

    


% Inherit from the specified cell style.




    
\documentclass[11pt]{article}

    
    
    \usepackage[T1]{fontenc}
    % Nicer default font (+ math font) than Computer Modern for most use cases
    \usepackage{mathpazo}

    % Basic figure setup, for now with no caption control since it's done
    % automatically by Pandoc (which extracts ![](path) syntax from Markdown).
    \usepackage{graphicx}
    % We will generate all images so they have a width \maxwidth. This means
    % that they will get their normal width if they fit onto the page, but
    % are scaled down if they would overflow the margins.
    \makeatletter
    \def\maxwidth{\ifdim\Gin@nat@width>\linewidth\linewidth
    \else\Gin@nat@width\fi}
    \makeatother
    \let\Oldincludegraphics\includegraphics
    % Set max figure width to be 80% of text width, for now hardcoded.
    \renewcommand{\includegraphics}[1]{\Oldincludegraphics[width=.8\maxwidth]{#1}}
    % Ensure that by default, figures have no caption (until we provide a
    % proper Figure object with a Caption API and a way to capture that
    % in the conversion process - todo).
    \usepackage{caption}
    \DeclareCaptionLabelFormat{nolabel}{}
    \captionsetup{labelformat=nolabel}

    \usepackage{adjustbox} % Used to constrain images to a maximum size 
    \usepackage{xcolor} % Allow colors to be defined
    \usepackage{enumerate} % Needed for markdown enumerations to work
    \usepackage{geometry} % Used to adjust the document margins
    \usepackage{amsmath} % Equations
    \usepackage{amssymb} % Equations
    \usepackage{textcomp} % defines textquotesingle
    % Hack from http://tex.stackexchange.com/a/47451/13684:
    \AtBeginDocument{%
        \def\PYZsq{\textquotesingle}% Upright quotes in Pygmentized code
    }
    \usepackage{upquote} % Upright quotes for verbatim code
    \usepackage{eurosym} % defines \euro
    \usepackage[mathletters]{ucs} % Extended unicode (utf-8) support
    \usepackage[utf8x]{inputenc} % Allow utf-8 characters in the tex document
    \usepackage{fancyvrb} % verbatim replacement that allows latex
    \usepackage{grffile} % extends the file name processing of package graphics 
                         % to support a larger range 
    % The hyperref package gives us a pdf with properly built
    % internal navigation ('pdf bookmarks' for the table of contents,
    % internal cross-reference links, web links for URLs, etc.)
    \usepackage{hyperref}
    \usepackage{longtable} % longtable support required by pandoc >1.10
    \usepackage{booktabs}  % table support for pandoc > 1.12.2
    \usepackage[inline]{enumitem} % IRkernel/repr support (it uses the enumerate* environment)
    \usepackage[normalem]{ulem} % ulem is needed to support strikethroughs (\sout)
                                % normalem makes italics be italics, not underlines
    

    
    
    % Colors for the hyperref package
    \definecolor{urlcolor}{rgb}{0,.145,.698}
    \definecolor{linkcolor}{rgb}{.71,0.21,0.01}
    \definecolor{citecolor}{rgb}{.12,.54,.11}

    % ANSI colors
    \definecolor{ansi-black}{HTML}{3E424D}
    \definecolor{ansi-black-intense}{HTML}{282C36}
    \definecolor{ansi-red}{HTML}{E75C58}
    \definecolor{ansi-red-intense}{HTML}{B22B31}
    \definecolor{ansi-green}{HTML}{00A250}
    \definecolor{ansi-green-intense}{HTML}{007427}
    \definecolor{ansi-yellow}{HTML}{DDB62B}
    \definecolor{ansi-yellow-intense}{HTML}{B27D12}
    \definecolor{ansi-blue}{HTML}{208FFB}
    \definecolor{ansi-blue-intense}{HTML}{0065CA}
    \definecolor{ansi-magenta}{HTML}{D160C4}
    \definecolor{ansi-magenta-intense}{HTML}{A03196}
    \definecolor{ansi-cyan}{HTML}{60C6C8}
    \definecolor{ansi-cyan-intense}{HTML}{258F8F}
    \definecolor{ansi-white}{HTML}{C5C1B4}
    \definecolor{ansi-white-intense}{HTML}{A1A6B2}

    % commands and environments needed by pandoc snippets
    % extracted from the output of `pandoc -s`
    \providecommand{\tightlist}{%
      \setlength{\itemsep}{0pt}\setlength{\parskip}{0pt}}
    \DefineVerbatimEnvironment{Highlighting}{Verbatim}{commandchars=\\\{\}}
    % Add ',fontsize=\small' for more characters per line
    \newenvironment{Shaded}{}{}
    \newcommand{\KeywordTok}[1]{\textcolor[rgb]{0.00,0.44,0.13}{\textbf{{#1}}}}
    \newcommand{\DataTypeTok}[1]{\textcolor[rgb]{0.56,0.13,0.00}{{#1}}}
    \newcommand{\DecValTok}[1]{\textcolor[rgb]{0.25,0.63,0.44}{{#1}}}
    \newcommand{\BaseNTok}[1]{\textcolor[rgb]{0.25,0.63,0.44}{{#1}}}
    \newcommand{\FloatTok}[1]{\textcolor[rgb]{0.25,0.63,0.44}{{#1}}}
    \newcommand{\CharTok}[1]{\textcolor[rgb]{0.25,0.44,0.63}{{#1}}}
    \newcommand{\StringTok}[1]{\textcolor[rgb]{0.25,0.44,0.63}{{#1}}}
    \newcommand{\CommentTok}[1]{\textcolor[rgb]{0.38,0.63,0.69}{\textit{{#1}}}}
    \newcommand{\OtherTok}[1]{\textcolor[rgb]{0.00,0.44,0.13}{{#1}}}
    \newcommand{\AlertTok}[1]{\textcolor[rgb]{1.00,0.00,0.00}{\textbf{{#1}}}}
    \newcommand{\FunctionTok}[1]{\textcolor[rgb]{0.02,0.16,0.49}{{#1}}}
    \newcommand{\RegionMarkerTok}[1]{{#1}}
    \newcommand{\ErrorTok}[1]{\textcolor[rgb]{1.00,0.00,0.00}{\textbf{{#1}}}}
    \newcommand{\NormalTok}[1]{{#1}}
    
    % Additional commands for more recent versions of Pandoc
    \newcommand{\ConstantTok}[1]{\textcolor[rgb]{0.53,0.00,0.00}{{#1}}}
    \newcommand{\SpecialCharTok}[1]{\textcolor[rgb]{0.25,0.44,0.63}{{#1}}}
    \newcommand{\VerbatimStringTok}[1]{\textcolor[rgb]{0.25,0.44,0.63}{{#1}}}
    \newcommand{\SpecialStringTok}[1]{\textcolor[rgb]{0.73,0.40,0.53}{{#1}}}
    \newcommand{\ImportTok}[1]{{#1}}
    \newcommand{\DocumentationTok}[1]{\textcolor[rgb]{0.73,0.13,0.13}{\textit{{#1}}}}
    \newcommand{\AnnotationTok}[1]{\textcolor[rgb]{0.38,0.63,0.69}{\textbf{\textit{{#1}}}}}
    \newcommand{\CommentVarTok}[1]{\textcolor[rgb]{0.38,0.63,0.69}{\textbf{\textit{{#1}}}}}
    \newcommand{\VariableTok}[1]{\textcolor[rgb]{0.10,0.09,0.49}{{#1}}}
    \newcommand{\ControlFlowTok}[1]{\textcolor[rgb]{0.00,0.44,0.13}{\textbf{{#1}}}}
    \newcommand{\OperatorTok}[1]{\textcolor[rgb]{0.40,0.40,0.40}{{#1}}}
    \newcommand{\BuiltInTok}[1]{{#1}}
    \newcommand{\ExtensionTok}[1]{{#1}}
    \newcommand{\PreprocessorTok}[1]{\textcolor[rgb]{0.74,0.48,0.00}{{#1}}}
    \newcommand{\AttributeTok}[1]{\textcolor[rgb]{0.49,0.56,0.16}{{#1}}}
    \newcommand{\InformationTok}[1]{\textcolor[rgb]{0.38,0.63,0.69}{\textbf{\textit{{#1}}}}}
    \newcommand{\WarningTok}[1]{\textcolor[rgb]{0.38,0.63,0.69}{\textbf{\textit{{#1}}}}}
    
    
    % Define a nice break command that doesn't care if a line doesn't already
    % exist.
    \def\br{\hspace*{\fill} \\* }
    % Math Jax compatability definitions
    \def\gt{>}
    \def\lt{<}
    % Document parameters
    \title{assignment09}
    
    
    

    % Pygments definitions
    
\makeatletter
\def\PY@reset{\let\PY@it=\relax \let\PY@bf=\relax%
    \let\PY@ul=\relax \let\PY@tc=\relax%
    \let\PY@bc=\relax \let\PY@ff=\relax}
\def\PY@tok#1{\csname PY@tok@#1\endcsname}
\def\PY@toks#1+{\ifx\relax#1\empty\else%
    \PY@tok{#1}\expandafter\PY@toks\fi}
\def\PY@do#1{\PY@bc{\PY@tc{\PY@ul{%
    \PY@it{\PY@bf{\PY@ff{#1}}}}}}}
\def\PY#1#2{\PY@reset\PY@toks#1+\relax+\PY@do{#2}}

\expandafter\def\csname PY@tok@w\endcsname{\def\PY@tc##1{\textcolor[rgb]{0.73,0.73,0.73}{##1}}}
\expandafter\def\csname PY@tok@c\endcsname{\let\PY@it=\textit\def\PY@tc##1{\textcolor[rgb]{0.25,0.50,0.50}{##1}}}
\expandafter\def\csname PY@tok@cp\endcsname{\def\PY@tc##1{\textcolor[rgb]{0.74,0.48,0.00}{##1}}}
\expandafter\def\csname PY@tok@k\endcsname{\let\PY@bf=\textbf\def\PY@tc##1{\textcolor[rgb]{0.00,0.50,0.00}{##1}}}
\expandafter\def\csname PY@tok@kp\endcsname{\def\PY@tc##1{\textcolor[rgb]{0.00,0.50,0.00}{##1}}}
\expandafter\def\csname PY@tok@kt\endcsname{\def\PY@tc##1{\textcolor[rgb]{0.69,0.00,0.25}{##1}}}
\expandafter\def\csname PY@tok@o\endcsname{\def\PY@tc##1{\textcolor[rgb]{0.40,0.40,0.40}{##1}}}
\expandafter\def\csname PY@tok@ow\endcsname{\let\PY@bf=\textbf\def\PY@tc##1{\textcolor[rgb]{0.67,0.13,1.00}{##1}}}
\expandafter\def\csname PY@tok@nb\endcsname{\def\PY@tc##1{\textcolor[rgb]{0.00,0.50,0.00}{##1}}}
\expandafter\def\csname PY@tok@nf\endcsname{\def\PY@tc##1{\textcolor[rgb]{0.00,0.00,1.00}{##1}}}
\expandafter\def\csname PY@tok@nc\endcsname{\let\PY@bf=\textbf\def\PY@tc##1{\textcolor[rgb]{0.00,0.00,1.00}{##1}}}
\expandafter\def\csname PY@tok@nn\endcsname{\let\PY@bf=\textbf\def\PY@tc##1{\textcolor[rgb]{0.00,0.00,1.00}{##1}}}
\expandafter\def\csname PY@tok@ne\endcsname{\let\PY@bf=\textbf\def\PY@tc##1{\textcolor[rgb]{0.82,0.25,0.23}{##1}}}
\expandafter\def\csname PY@tok@nv\endcsname{\def\PY@tc##1{\textcolor[rgb]{0.10,0.09,0.49}{##1}}}
\expandafter\def\csname PY@tok@no\endcsname{\def\PY@tc##1{\textcolor[rgb]{0.53,0.00,0.00}{##1}}}
\expandafter\def\csname PY@tok@nl\endcsname{\def\PY@tc##1{\textcolor[rgb]{0.63,0.63,0.00}{##1}}}
\expandafter\def\csname PY@tok@ni\endcsname{\let\PY@bf=\textbf\def\PY@tc##1{\textcolor[rgb]{0.60,0.60,0.60}{##1}}}
\expandafter\def\csname PY@tok@na\endcsname{\def\PY@tc##1{\textcolor[rgb]{0.49,0.56,0.16}{##1}}}
\expandafter\def\csname PY@tok@nt\endcsname{\let\PY@bf=\textbf\def\PY@tc##1{\textcolor[rgb]{0.00,0.50,0.00}{##1}}}
\expandafter\def\csname PY@tok@nd\endcsname{\def\PY@tc##1{\textcolor[rgb]{0.67,0.13,1.00}{##1}}}
\expandafter\def\csname PY@tok@s\endcsname{\def\PY@tc##1{\textcolor[rgb]{0.73,0.13,0.13}{##1}}}
\expandafter\def\csname PY@tok@sd\endcsname{\let\PY@it=\textit\def\PY@tc##1{\textcolor[rgb]{0.73,0.13,0.13}{##1}}}
\expandafter\def\csname PY@tok@si\endcsname{\let\PY@bf=\textbf\def\PY@tc##1{\textcolor[rgb]{0.73,0.40,0.53}{##1}}}
\expandafter\def\csname PY@tok@se\endcsname{\let\PY@bf=\textbf\def\PY@tc##1{\textcolor[rgb]{0.73,0.40,0.13}{##1}}}
\expandafter\def\csname PY@tok@sr\endcsname{\def\PY@tc##1{\textcolor[rgb]{0.73,0.40,0.53}{##1}}}
\expandafter\def\csname PY@tok@ss\endcsname{\def\PY@tc##1{\textcolor[rgb]{0.10,0.09,0.49}{##1}}}
\expandafter\def\csname PY@tok@sx\endcsname{\def\PY@tc##1{\textcolor[rgb]{0.00,0.50,0.00}{##1}}}
\expandafter\def\csname PY@tok@m\endcsname{\def\PY@tc##1{\textcolor[rgb]{0.40,0.40,0.40}{##1}}}
\expandafter\def\csname PY@tok@gh\endcsname{\let\PY@bf=\textbf\def\PY@tc##1{\textcolor[rgb]{0.00,0.00,0.50}{##1}}}
\expandafter\def\csname PY@tok@gu\endcsname{\let\PY@bf=\textbf\def\PY@tc##1{\textcolor[rgb]{0.50,0.00,0.50}{##1}}}
\expandafter\def\csname PY@tok@gd\endcsname{\def\PY@tc##1{\textcolor[rgb]{0.63,0.00,0.00}{##1}}}
\expandafter\def\csname PY@tok@gi\endcsname{\def\PY@tc##1{\textcolor[rgb]{0.00,0.63,0.00}{##1}}}
\expandafter\def\csname PY@tok@gr\endcsname{\def\PY@tc##1{\textcolor[rgb]{1.00,0.00,0.00}{##1}}}
\expandafter\def\csname PY@tok@ge\endcsname{\let\PY@it=\textit}
\expandafter\def\csname PY@tok@gs\endcsname{\let\PY@bf=\textbf}
\expandafter\def\csname PY@tok@gp\endcsname{\let\PY@bf=\textbf\def\PY@tc##1{\textcolor[rgb]{0.00,0.00,0.50}{##1}}}
\expandafter\def\csname PY@tok@go\endcsname{\def\PY@tc##1{\textcolor[rgb]{0.53,0.53,0.53}{##1}}}
\expandafter\def\csname PY@tok@gt\endcsname{\def\PY@tc##1{\textcolor[rgb]{0.00,0.27,0.87}{##1}}}
\expandafter\def\csname PY@tok@err\endcsname{\def\PY@bc##1{\setlength{\fboxsep}{0pt}\fcolorbox[rgb]{1.00,0.00,0.00}{1,1,1}{\strut ##1}}}
\expandafter\def\csname PY@tok@kc\endcsname{\let\PY@bf=\textbf\def\PY@tc##1{\textcolor[rgb]{0.00,0.50,0.00}{##1}}}
\expandafter\def\csname PY@tok@kd\endcsname{\let\PY@bf=\textbf\def\PY@tc##1{\textcolor[rgb]{0.00,0.50,0.00}{##1}}}
\expandafter\def\csname PY@tok@kn\endcsname{\let\PY@bf=\textbf\def\PY@tc##1{\textcolor[rgb]{0.00,0.50,0.00}{##1}}}
\expandafter\def\csname PY@tok@kr\endcsname{\let\PY@bf=\textbf\def\PY@tc##1{\textcolor[rgb]{0.00,0.50,0.00}{##1}}}
\expandafter\def\csname PY@tok@bp\endcsname{\def\PY@tc##1{\textcolor[rgb]{0.00,0.50,0.00}{##1}}}
\expandafter\def\csname PY@tok@fm\endcsname{\def\PY@tc##1{\textcolor[rgb]{0.00,0.00,1.00}{##1}}}
\expandafter\def\csname PY@tok@vc\endcsname{\def\PY@tc##1{\textcolor[rgb]{0.10,0.09,0.49}{##1}}}
\expandafter\def\csname PY@tok@vg\endcsname{\def\PY@tc##1{\textcolor[rgb]{0.10,0.09,0.49}{##1}}}
\expandafter\def\csname PY@tok@vi\endcsname{\def\PY@tc##1{\textcolor[rgb]{0.10,0.09,0.49}{##1}}}
\expandafter\def\csname PY@tok@vm\endcsname{\def\PY@tc##1{\textcolor[rgb]{0.10,0.09,0.49}{##1}}}
\expandafter\def\csname PY@tok@sa\endcsname{\def\PY@tc##1{\textcolor[rgb]{0.73,0.13,0.13}{##1}}}
\expandafter\def\csname PY@tok@sb\endcsname{\def\PY@tc##1{\textcolor[rgb]{0.73,0.13,0.13}{##1}}}
\expandafter\def\csname PY@tok@sc\endcsname{\def\PY@tc##1{\textcolor[rgb]{0.73,0.13,0.13}{##1}}}
\expandafter\def\csname PY@tok@dl\endcsname{\def\PY@tc##1{\textcolor[rgb]{0.73,0.13,0.13}{##1}}}
\expandafter\def\csname PY@tok@s2\endcsname{\def\PY@tc##1{\textcolor[rgb]{0.73,0.13,0.13}{##1}}}
\expandafter\def\csname PY@tok@sh\endcsname{\def\PY@tc##1{\textcolor[rgb]{0.73,0.13,0.13}{##1}}}
\expandafter\def\csname PY@tok@s1\endcsname{\def\PY@tc##1{\textcolor[rgb]{0.73,0.13,0.13}{##1}}}
\expandafter\def\csname PY@tok@mb\endcsname{\def\PY@tc##1{\textcolor[rgb]{0.40,0.40,0.40}{##1}}}
\expandafter\def\csname PY@tok@mf\endcsname{\def\PY@tc##1{\textcolor[rgb]{0.40,0.40,0.40}{##1}}}
\expandafter\def\csname PY@tok@mh\endcsname{\def\PY@tc##1{\textcolor[rgb]{0.40,0.40,0.40}{##1}}}
\expandafter\def\csname PY@tok@mi\endcsname{\def\PY@tc##1{\textcolor[rgb]{0.40,0.40,0.40}{##1}}}
\expandafter\def\csname PY@tok@il\endcsname{\def\PY@tc##1{\textcolor[rgb]{0.40,0.40,0.40}{##1}}}
\expandafter\def\csname PY@tok@mo\endcsname{\def\PY@tc##1{\textcolor[rgb]{0.40,0.40,0.40}{##1}}}
\expandafter\def\csname PY@tok@ch\endcsname{\let\PY@it=\textit\def\PY@tc##1{\textcolor[rgb]{0.25,0.50,0.50}{##1}}}
\expandafter\def\csname PY@tok@cm\endcsname{\let\PY@it=\textit\def\PY@tc##1{\textcolor[rgb]{0.25,0.50,0.50}{##1}}}
\expandafter\def\csname PY@tok@cpf\endcsname{\let\PY@it=\textit\def\PY@tc##1{\textcolor[rgb]{0.25,0.50,0.50}{##1}}}
\expandafter\def\csname PY@tok@c1\endcsname{\let\PY@it=\textit\def\PY@tc##1{\textcolor[rgb]{0.25,0.50,0.50}{##1}}}
\expandafter\def\csname PY@tok@cs\endcsname{\let\PY@it=\textit\def\PY@tc##1{\textcolor[rgb]{0.25,0.50,0.50}{##1}}}

\def\PYZbs{\char`\\}
\def\PYZus{\char`\_}
\def\PYZob{\char`\{}
\def\PYZcb{\char`\}}
\def\PYZca{\char`\^}
\def\PYZam{\char`\&}
\def\PYZlt{\char`\<}
\def\PYZgt{\char`\>}
\def\PYZsh{\char`\#}
\def\PYZpc{\char`\%}
\def\PYZdl{\char`\$}
\def\PYZhy{\char`\-}
\def\PYZsq{\char`\'}
\def\PYZdq{\char`\"}
\def\PYZti{\char`\~}
% for compatibility with earlier versions
\def\PYZat{@}
\def\PYZlb{[}
\def\PYZrb{]}
\makeatother


    % Exact colors from NB
    \definecolor{incolor}{rgb}{0.0, 0.0, 0.5}
    \definecolor{outcolor}{rgb}{0.545, 0.0, 0.0}



    
    % Prevent overflowing lines due to hard-to-break entities
    \sloppy 
    % Setup hyperref package
    \hypersetup{
      breaklinks=true,  % so long urls are correctly broken across lines
      colorlinks=true,
      urlcolor=urlcolor,
      linkcolor=linkcolor,
      citecolor=citecolor,
      }
    % Slightly bigger margins than the latex defaults
    
    \geometry{verbose,tmargin=1in,bmargin=1in,lmargin=1in,rmargin=1in}
    
    

    \begin{document}
    
    
    \maketitle
    
    

    
    \section{Binary Classification with Different
Features}\label{binary-classification-with-different-features}

\textbf{Name}: ZHU GUANGYU\\
\textbf{Student ID}: 20165953\\
\textbf{Github Repo}:
\href{https://github.com/z1ggy-o/cv_assignment/tree/master/assignment09}{assignment09}

\begin{center}\rule{0.5\linewidth}{\linethickness}\end{center}

In a \emph{classification problem}, the outcome takes on only a finit
number of values. In the simplest case, outcome has only two values, for
example TRUE or FALSE. This is called the \emph{binary classification
problem}.

As in real-valued data fitting, we assume that an approxomate relation
ship of the form \(y \approx f(x)\) holds, where
\(f: R^{n} \rightarrow {-1, +1}\). The model \(\hat{f}\) is called a
\emph{classifier}.

For a given data point x, y with predicted outcome
\(\hat{y} = \hat{f}(x)\), there are four possibilities:

\begin{itemize}
\tightlist
\item
  \emph{True positive}: \(y = +1\) and \(\hat{y} = +1\).
\item
  \emph{True negative}: \(y = -1\) and \(\hat{y} = -1\).
\item
  \emph{False positive}: \(y = -1\) and \(\hat{y} = +1\).
\item
  \emph{False negative}: \(y = +1\) and \(\hat{y} = -1\).
\end{itemize}

\begin{center}\rule{0.5\linewidth}{\linethickness}\end{center}

Continue with assignmen08, we still use \emph{least squares classifer}
to separate 0 and other numbers in MNIST data set.

Sign function is same: \[ sign(x)=\left\{
\begin{aligned}
+1 & & if  x \geq 0 \\
-1 & & if  x < 0 
\end{aligned}
\right.
\]

But this time, we change feature function to see the differencies.

We define new feature functions as
\(f_{i} = r_{i}^{T}x, r_{i} \sim N(0, \sigma)\), and try with varing the
number of parameter \(p\) with the standard deviation \(\sigma = 1\) of
the random feature vectore \(r\).

\begin{center}\rule{0.5\linewidth}{\linethickness}\end{center}

    \subsection{Create Classifier}\label{create-classifier}

\subsubsection{First, import data sets}\label{first-import-data-sets}

We have two data sets, one for training, one for testing. Each element
is a image that has height 28 and width 28 pixels.

    \begin{Verbatim}[commandchars=\\\{\}]
{\color{incolor}In [{\color{incolor}1}]:} \PY{k+kn}{import} \PY{n+nn}{matplotlib}\PY{n+nn}{.}\PY{n+nn}{pyplot} \PY{k}{as} \PY{n+nn}{plt}
        \PY{k+kn}{import} \PY{n+nn}{numpy} \PY{k}{as} \PY{n+nn}{np}
        
        \PY{n}{file\PYZus{}data\PYZus{}train} \PY{o}{=} \PY{l+s+s2}{\PYZdq{}}\PY{l+s+s2}{mnist\PYZus{}train.csv}\PY{l+s+s2}{\PYZdq{}}
        \PY{n}{file\PYZus{}data\PYZus{}test}  \PY{o}{=} \PY{l+s+s2}{\PYZdq{}}\PY{l+s+s2}{mnist\PYZus{}test.csv}\PY{l+s+s2}{\PYZdq{}}
        
        \PY{n}{h\PYZus{}data\PYZus{}train}    \PY{o}{=} \PY{n+nb}{open}\PY{p}{(}\PY{n}{file\PYZus{}data\PYZus{}train}\PY{p}{,} \PY{l+s+s2}{\PYZdq{}}\PY{l+s+s2}{r}\PY{l+s+s2}{\PYZdq{}}\PY{p}{)}
        \PY{n}{h\PYZus{}data\PYZus{}test}     \PY{o}{=} \PY{n+nb}{open}\PY{p}{(}\PY{n}{file\PYZus{}data\PYZus{}test}\PY{p}{,} \PY{l+s+s2}{\PYZdq{}}\PY{l+s+s2}{r}\PY{l+s+s2}{\PYZdq{}}\PY{p}{)}
        
        \PY{n}{data\PYZus{}train}      \PY{o}{=} \PY{n}{h\PYZus{}data\PYZus{}train}\PY{o}{.}\PY{n}{readlines}\PY{p}{(}\PY{p}{)}
        \PY{n}{data\PYZus{}test}       \PY{o}{=} \PY{n}{h\PYZus{}data\PYZus{}test}\PY{o}{.}\PY{n}{readlines}\PY{p}{(}\PY{p}{)}
        
        \PY{n}{h\PYZus{}data\PYZus{}train}\PY{o}{.}\PY{n}{close}\PY{p}{(}\PY{p}{)}
        \PY{n}{h\PYZus{}data\PYZus{}test}\PY{o}{.}\PY{n}{close}\PY{p}{(}\PY{p}{)}
        
        \PY{n}{size\PYZus{}row}    \PY{o}{=} \PY{l+m+mi}{28}    \PY{c+c1}{\PYZsh{} height of the image}
        \PY{n}{size\PYZus{}col}    \PY{o}{=} \PY{l+m+mi}{28}    \PY{c+c1}{\PYZsh{} width of the image}
        
        \PY{n}{num\PYZus{}train}   \PY{o}{=} \PY{n+nb}{len}\PY{p}{(}\PY{n}{data\PYZus{}train}\PY{p}{)}   \PY{c+c1}{\PYZsh{} number of training images}
        \PY{n}{num\PYZus{}test}    \PY{o}{=} \PY{n+nb}{len}\PY{p}{(}\PY{n}{data\PYZus{}test}\PY{p}{)}    \PY{c+c1}{\PYZsh{} number of testing images}
        
        \PY{c+c1}{\PYZsh{} number of training images: 60000}
        \PY{c+c1}{\PYZsh{} number of testing images: 10000}
\end{Verbatim}


    To reduce the bias, we need to normalize the data.

    \begin{Verbatim}[commandchars=\\\{\}]
{\color{incolor}In [{\color{incolor}2}]:} \PY{c+c1}{\PYZsh{}}
        \PY{c+c1}{\PYZsh{} normalize the values of the input data to be [0, 1]}
        \PY{c+c1}{\PYZsh{}}
        \PY{k}{def} \PY{n+nf}{normalize}\PY{p}{(}\PY{n}{data}\PY{p}{)}\PY{p}{:}
        
            \PY{n}{data\PYZus{}normalized} \PY{o}{=} \PY{p}{(}\PY{n}{data} \PY{o}{\PYZhy{}} \PY{n+nb}{min}\PY{p}{(}\PY{n}{data}\PY{p}{)}\PY{p}{)} \PY{o}{/} \PY{p}{(}\PY{n+nb}{max}\PY{p}{(}\PY{n}{data}\PY{p}{)} \PY{o}{\PYZhy{}} \PY{n+nb}{min}\PY{p}{(}\PY{n}{data}\PY{p}{)}\PY{p}{)}
        
            \PY{k}{return}\PY{p}{(}\PY{n}{data\PYZus{}normalized}\PY{p}{)}
\end{Verbatim}


    Normalize each pixel, and put image data into a 784*num\_image matrix.

    \begin{Verbatim}[commandchars=\\\{\}]
{\color{incolor}In [{\color{incolor}3}]:} \PY{c+c1}{\PYZsh{}}
        \PY{c+c1}{\PYZsh{} make a matrix each column of which represents an images in a vector form }
        \PY{c+c1}{\PYZsh{}}
        \PY{n}{list\PYZus{}image\PYZus{}train}    \PY{o}{=} \PY{n}{np}\PY{o}{.}\PY{n}{empty}\PY{p}{(}\PY{p}{(}\PY{n}{size\PYZus{}row} \PY{o}{*} \PY{n}{size\PYZus{}col}\PY{p}{,} \PY{n}{num\PYZus{}train}\PY{p}{)}\PY{p}{,} \PY{n}{dtype}\PY{o}{=}\PY{n+nb}{float}\PY{p}{)}  \PY{c+c1}{\PYZsh{} 784 * num\PYZus{}trian matrix}
        \PY{n}{list\PYZus{}label\PYZus{}train}    \PY{o}{=} \PY{n}{np}\PY{o}{.}\PY{n}{empty}\PY{p}{(}\PY{n}{num\PYZus{}train}\PY{p}{,} \PY{n}{dtype}\PY{o}{=}\PY{n+nb}{int}\PY{p}{)}
        
        \PY{n}{list\PYZus{}image\PYZus{}test}     \PY{o}{=} \PY{n}{np}\PY{o}{.}\PY{n}{empty}\PY{p}{(}\PY{p}{(}\PY{n}{size\PYZus{}row} \PY{o}{*} \PY{n}{size\PYZus{}col}\PY{p}{,} \PY{n}{num\PYZus{}test}\PY{p}{)}\PY{p}{,} \PY{n}{dtype}\PY{o}{=}\PY{n+nb}{float}\PY{p}{)}
        \PY{n}{list\PYZus{}label\PYZus{}test}     \PY{o}{=} \PY{n}{np}\PY{o}{.}\PY{n}{empty}\PY{p}{(}\PY{n}{num\PYZus{}test}\PY{p}{,} \PY{n}{dtype}\PY{o}{=}\PY{n+nb}{int}\PY{p}{)}
        
        \PY{n}{count} \PY{o}{=} \PY{l+m+mi}{0}
        
        \PY{k}{for} \PY{n}{line} \PY{o+ow}{in} \PY{n}{data\PYZus{}train}\PY{p}{:}
        
            \PY{n}{line\PYZus{}data}   \PY{o}{=} \PY{n}{line}\PY{o}{.}\PY{n}{split}\PY{p}{(}\PY{l+s+s1}{\PYZsq{}}\PY{l+s+s1}{,}\PY{l+s+s1}{\PYZsq{}}\PY{p}{)}
            \PY{n}{label}       \PY{o}{=} \PY{n}{line\PYZus{}data}\PY{p}{[}\PY{l+m+mi}{0}\PY{p}{]}
            \PY{n}{im\PYZus{}vector}   \PY{o}{=} \PY{n}{np}\PY{o}{.}\PY{n}{asfarray}\PY{p}{(}\PY{n}{line\PYZus{}data}\PY{p}{[}\PY{l+m+mi}{1}\PY{p}{:}\PY{p}{]}\PY{p}{)}  \PY{c+c1}{\PYZsh{} convert to float type}
            \PY{n}{im\PYZus{}vector}   \PY{o}{=} \PY{n}{normalize}\PY{p}{(}\PY{n}{im\PYZus{}vector}\PY{p}{)}
        
            \PY{n}{list\PYZus{}label\PYZus{}train}\PY{p}{[}\PY{n}{count}\PY{p}{]}     \PY{o}{=} \PY{n}{label}
            \PY{n}{list\PYZus{}image\PYZus{}train}\PY{p}{[}\PY{p}{:}\PY{p}{,} \PY{n}{count}\PY{p}{]}  \PY{o}{=} \PY{n}{im\PYZus{}vector}  \PY{c+c1}{\PYZsh{} each column is a image}
        
            \PY{n}{count} \PY{o}{+}\PY{o}{=} \PY{l+m+mi}{1}
        
        \PY{n}{count} \PY{o}{=} \PY{l+m+mi}{0}
        
        \PY{k}{for} \PY{n}{line} \PY{o+ow}{in} \PY{n}{data\PYZus{}test}\PY{p}{:}
        
            \PY{n}{line\PYZus{}data}   \PY{o}{=} \PY{n}{line}\PY{o}{.}\PY{n}{split}\PY{p}{(}\PY{l+s+s1}{\PYZsq{}}\PY{l+s+s1}{,}\PY{l+s+s1}{\PYZsq{}}\PY{p}{)}
            \PY{n}{label}       \PY{o}{=} \PY{n}{line\PYZus{}data}\PY{p}{[}\PY{l+m+mi}{0}\PY{p}{]}
            \PY{n}{im\PYZus{}vector}   \PY{o}{=} \PY{n}{np}\PY{o}{.}\PY{n}{asfarray}\PY{p}{(}\PY{n}{line\PYZus{}data}\PY{p}{[}\PY{l+m+mi}{1}\PY{p}{:}\PY{p}{]}\PY{p}{)}
            \PY{n}{im\PYZus{}vector}   \PY{o}{=} \PY{n}{normalize}\PY{p}{(}\PY{n}{im\PYZus{}vector}\PY{p}{)}
        
            \PY{n}{list\PYZus{}label\PYZus{}test}\PY{p}{[}\PY{n}{count}\PY{p}{]}      \PY{o}{=} \PY{n}{label}
            \PY{n}{list\PYZus{}image\PYZus{}test}\PY{p}{[}\PY{p}{:}\PY{p}{,} \PY{n}{count}\PY{p}{]}   \PY{o}{=} \PY{n}{im\PYZus{}vector}    
        
            \PY{n}{count} \PY{o}{+}\PY{o}{=} \PY{l+m+mi}{1}
\end{Verbatim}


    \subsubsection{Define feature functions
generator}\label{define-feature-functions-generator}

This is the core part of this assignment.

Depends on the given number of parameters we should generate correponde
feature functions \(r\). Each element of \(r\) is a vector with length
\(28 * 28\). The element of \(r_{i}\) is random number from a normal
distribution.

    \begin{Verbatim}[commandchars=\\\{\}]
{\color{incolor}In [{\color{incolor}4}]:} \PY{k}{def} \PY{n+nf}{generate\PYZus{}features}\PY{p}{(}\PY{n}{n}\PY{p}{,} \PY{n}{size}\PY{p}{)}\PY{p}{:}
            \PY{l+s+sd}{\PYZdq{}\PYZdq{}\PYZdq{}Generate feature functions}
        \PY{l+s+sd}{    }
        \PY{l+s+sd}{    Argumengs:}
        \PY{l+s+sd}{        n(int): number of parameters}
        \PY{l+s+sd}{        size: number of elements of each vector}
        \PY{l+s+sd}{    Return:}
        \PY{l+s+sd}{        functions(2d matrix): feature function matrix}
        \PY{l+s+sd}{    \PYZdq{}\PYZdq{}\PYZdq{}}
            
            \PY{n}{functions} \PY{o}{=} \PY{p}{[}\PY{p}{]}
            
            \PY{c+c1}{\PYZsh{} for n, generate vectore with \PYZsh{}size elements}
            \PY{n}{mean}\PY{p}{,} \PY{n}{sigma} \PY{o}{=} \PY{l+m+mi}{0}\PY{p}{,} \PY{l+m+mi}{1}
            
            \PY{k}{for} \PY{n}{\PYZus{}} \PY{o+ow}{in} \PY{n+nb}{range}\PY{p}{(}\PY{n}{n}\PY{p}{)}\PY{p}{:}
                \PY{n}{ri} \PY{o}{=} \PY{n}{np}\PY{o}{.}\PY{n}{random}\PY{o}{.}\PY{n}{normal}\PY{p}{(}\PY{n}{mean}\PY{p}{,} \PY{n}{sigma}\PY{p}{,} \PY{n}{size}\PY{p}{)}
                \PY{n}{functions}\PY{o}{.}\PY{n}{append}\PY{p}{(}\PY{n}{ri}\PY{p}{)}
            
            \PY{k}{return} \PY{n}{np}\PY{o}{.}\PY{n}{array}\PY{p}{(}\PY{n}{functions}\PY{p}{)} 
\end{Verbatim}


    \subsubsection{Generate the matrix of feature funtions and
data}\label{generate-the-matrix-of-feature-funtions-and-data}

Since we have got feature functions, now we can apply them on input
data.

    \begin{Verbatim}[commandchars=\\\{\}]
{\color{incolor}In [{\color{incolor}5}]:} \PY{k}{def} \PY{n+nf}{generate\PYZus{}tilde\PYZus{}matrix}\PY{p}{(}\PY{n}{feature\PYZus{}func}\PY{p}{,} \PY{n}{data}\PY{p}{,} \PY{n}{num\PYZus{}data}\PY{p}{)}\PY{p}{:}
            \PY{l+s+sd}{\PYZdq{}\PYZdq{}\PYZdq{}Create matrix of feature function on data}
        \PY{l+s+sd}{    }
        \PY{l+s+sd}{    Arguments:}
        \PY{l+s+sd}{        feature\PYZus{}func(2d matrix): feature function matrix}
        \PY{l+s+sd}{        data: input image data}
        \PY{l+s+sd}{        num\PYZus{}data: number of input data}
        \PY{l+s+sd}{    Return:}
        \PY{l+s+sd}{        matrix of feature functions applied on image data}
        \PY{l+s+sd}{    \PYZdq{}\PYZdq{}\PYZdq{}}
            
            \PY{n}{A} \PY{o}{=} \PY{p}{[}\PY{p}{]}
            
            \PY{k}{for} \PY{n}{i} \PY{o+ow}{in} \PY{n+nb}{range}\PY{p}{(}\PY{n}{num\PYZus{}data}\PY{p}{)}\PY{p}{:}
                \PY{n}{img} \PY{o}{=} \PY{n}{data}\PY{p}{[}\PY{p}{:}\PY{p}{,} \PY{n}{i}\PY{p}{]}  \PY{c+c1}{\PYZsh{} ith image(column)}
                \PY{n}{row} \PY{o}{=} \PY{n}{np}\PY{o}{.}\PY{n}{inner}\PY{p}{(}\PY{n}{feature\PYZus{}func}\PY{p}{,} \PY{n}{img}\PY{p}{)}
                \PY{n}{A}\PY{o}{.}\PY{n}{append}\PY{p}{(}\PY{n}{row}\PY{p}{)}
            
            \PY{k}{return} \PY{n}{np}\PY{o}{.}\PY{n}{array}\PY{p}{(}\PY{n}{A}\PY{p}{)}
\end{Verbatim}


    \subsubsection{\texorpdfstring{Compute
\(\theta\)}{Compute \textbackslash{}theta}}\label{compute-theta}

Depends on \(A\theta = y\), while \(A\) is the matrix of feature
function apply on data set, \(\theta\) is perameters, and \(y\) is the
label.

Because we just want to separate 0 and other numbers, we need to process
label \(y\) which gives 0's image \(+1\) and other number's image
\(-1\).

Through pseudo inverse \((A^{T}A)^{-1}A\) we can compute the \(\theta\).
Here we use \texttt{np.linalg.pinv} to get \(A^{-1}\).

    \begin{Verbatim}[commandchars=\\\{\}]
{\color{incolor}In [{\color{incolor}6}]:} \PY{c+c1}{\PYZsh{} process label array}
        \PY{k}{def} \PY{n+nf}{process\PYZus{}label}\PY{p}{(}\PY{n}{labels}\PY{p}{)}\PY{p}{:}
            \PY{n}{result} \PY{o}{=} \PY{p}{[}\PY{p}{]}
            \PY{k}{for} \PY{n}{label} \PY{o+ow}{in} \PY{n}{labels}\PY{p}{:}
                \PY{k}{if} \PY{n}{label} \PY{o}{==} \PY{l+m+mi}{0}\PY{p}{:}
                    \PY{n}{result}\PY{o}{.}\PY{n}{append}\PY{p}{(}\PY{l+m+mi}{1}\PY{p}{)}
                \PY{k}{else}\PY{p}{:}
                    \PY{n}{result}\PY{o}{.}\PY{n}{append}\PY{p}{(}\PY{o}{\PYZhy{}}\PY{l+m+mi}{1}\PY{p}{)}
            
            \PY{k}{return} \PY{n}{result}
        
                
        \PY{k}{def} \PY{n+nf}{compute\PYZus{}theta}\PY{p}{(}\PY{n}{A}\PY{p}{,} \PY{n}{y}\PY{p}{)}\PY{p}{:}
            \PY{n}{A\PYZus{}inv} \PY{o}{=} \PY{n}{np}\PY{o}{.}\PY{n}{linalg}\PY{o}{.}\PY{n}{pinv}\PY{p}{(}\PY{n}{A}\PY{p}{)}
            \PY{n}{theta} \PY{o}{=} \PY{n}{np}\PY{o}{.}\PY{n}{inner}\PY{p}{(}\PY{n}{A\PYZus{}inv}\PY{p}{,} \PY{n}{y}\PY{p}{)}
            
            \PY{k}{return} \PY{n}{theta}
\end{Verbatim}


    \subsubsection{\texorpdfstring{Define classifier
\(\hat{f}(x)\)}{Define classifier \textbackslash{}hat\{f\}(x)}}\label{define-classifier-hatfx}

Now we have had feature functions, parameters, so we can create our
classifier \(\hat{f}(x) = sign(\tilde{f}(x))\), where

\[ sign(x)=\left\{
\begin{aligned}
+1 & & if  x \geq 0 \\
-1 & & if  x < 0 
\end{aligned}
\right.
\]

\[\tilde{f}(x) = \theta_{1}f_{1}(x) + \theta_{2}f_{2}(x) + \cdots + \theta_{p}f_{p}(x)\]

    \paragraph{sign function}\label{sign-function}

    \begin{Verbatim}[commandchars=\\\{\}]
{\color{incolor}In [{\color{incolor}7}]:} \PY{k}{def} \PY{n+nf}{sign\PYZus{}func}\PY{p}{(}\PY{n}{x}\PY{p}{)}\PY{p}{:}
            \PY{k}{if} \PY{n}{x} \PY{o}{\PYZgt{}}\PY{o}{=} \PY{l+m+mi}{0}\PY{p}{:}
                \PY{k}{return} \PY{l+m+mi}{1}
            \PY{k}{else}\PY{p}{:}
                \PY{k}{return} \PY{o}{\PYZhy{}}\PY{l+m+mi}{1}
\end{Verbatim}


    \paragraph{Combine all components}\label{combine-all-components}

    \begin{Verbatim}[commandchars=\\\{\}]
{\color{incolor}In [{\color{incolor}8}]:} \PY{k}{def} \PY{n+nf}{classifier}\PY{p}{(}\PY{n}{input\PYZus{}data}\PY{p}{,} \PY{n}{data\PYZus{}num}\PY{p}{,} \PY{n}{feature\PYZus{}funcs}\PY{p}{,} \PY{n}{theta}\PY{p}{)}\PY{p}{:}
                \PY{l+s+sd}{\PYZdq{}\PYZdq{}\PYZdq{}Given classify result}
        \PY{l+s+sd}{        }
        \PY{l+s+sd}{        Argument:}
        \PY{l+s+sd}{            input\PYZus{}data(2d matrix): testing image data}
        \PY{l+s+sd}{        Return:}
        \PY{l+s+sd}{            determine result array}
        \PY{l+s+sd}{        \PYZdq{}\PYZdq{}\PYZdq{}}
                
                \PY{c+c1}{\PYZsh{} generate tilde f matrix}
                \PY{n}{tilde\PYZus{}matrix} \PY{o}{=} \PY{n}{generate\PYZus{}tilde\PYZus{}matrix}\PY{p}{(}\PY{n}{feature\PYZus{}funcs}\PY{p}{,} \PY{n}{input\PYZus{}data}\PY{p}{,} \PY{n}{data\PYZus{}num}\PY{p}{)}
                
                \PY{c+c1}{\PYZsh{} get classify result}
                \PY{n}{f\PYZus{}tilde} \PY{o}{=} \PY{n}{np}\PY{o}{.}\PY{n}{inner}\PY{p}{(}\PY{n}{tilde\PYZus{}matrix}\PY{p}{,} \PY{n}{theta}\PY{p}{)}
                \PY{n}{f\PYZus{}hat} \PY{o}{=} \PY{n+nb}{list}\PY{p}{(}\PY{n+nb}{map}\PY{p}{(}\PY{n}{sign\PYZus{}func}\PY{p}{,} \PY{n}{f\PYZus{}tilde}\PY{p}{)}\PY{p}{)}
                
                \PY{k}{return} \PY{n}{np}\PY{o}{.}\PY{n}{array}\PY{p}{(}\PY{n}{f\PYZus{}hat}\PY{p}{)}
        
        
        \PY{k}{def} \PY{n+nf}{eles\PYZus{}classifier}\PY{p}{(}\PY{n}{num\PYZus{}paras}\PY{p}{,} \PY{n}{size}\PY{p}{,} \PY{n}{data\PYZus{}train}\PY{p}{,} \PY{n}{num\PYZus{}data}\PY{p}{,} \PY{n}{labels}\PY{p}{)}\PY{p}{:}
            \PY{l+s+sd}{\PYZdq{}\PYZdq{}\PYZdq{}Generate feature functions, compute parameters}
        \PY{l+s+sd}{    }
        \PY{l+s+sd}{    Arguments:}
        \PY{l+s+sd}{        num\PYZus{}paras(int): number os parameters}
        \PY{l+s+sd}{        size(int): size of input data}
        \PY{l+s+sd}{        data\PYZus{}train(2d matrix): training data set}
        \PY{l+s+sd}{        num\PYZus{}data(int): number of data}
        \PY{l+s+sd}{        labels(array): label of each training data}
        \PY{l+s+sd}{    Returen:}
        \PY{l+s+sd}{        (tuple): feature function, theta}
        \PY{l+s+sd}{        }
        \PY{l+s+sd}{    \PYZdq{}\PYZdq{}\PYZdq{}}
            
            \PY{c+c1}{\PYZsh{} generate feature functions}
            \PY{n}{feature\PYZus{}funcs} \PY{o}{=} \PY{n}{generate\PYZus{}features}\PY{p}{(}\PY{n}{num\PYZus{}paras}\PY{p}{,} \PY{n}{size}\PY{p}{)}
            
            \PY{c+c1}{\PYZsh{} generate tilde f matrix}
            \PY{n}{A} \PY{o}{=} \PY{n}{generate\PYZus{}tilde\PYZus{}matrix}\PY{p}{(}\PY{n}{feature\PYZus{}funcs}\PY{p}{,} \PY{n}{data\PYZus{}train}\PY{p}{,} \PY{n}{num\PYZus{}data}\PY{p}{)}
            
            \PY{c+c1}{\PYZsh{} process label}
            \PY{n}{label\PYZus{}list} \PY{o}{=} \PY{n}{process\PYZus{}label}\PY{p}{(}\PY{n}{labels}\PY{p}{)}
            
            \PY{c+c1}{\PYZsh{} compute parameter}
            \PY{n}{theta} \PY{o}{=} \PY{n}{compute\PYZus{}theta}\PY{p}{(}\PY{n}{A}\PY{p}{,} \PY{n}{label\PYZus{}list}\PY{p}{)}
            
            \PY{k}{return} \PY{n}{feature\PYZus{}funcs}\PY{p}{,} \PY{n}{theta}
\end{Verbatim}


    \subsubsection{Count predicted outcome: TP, FP, TN, and
FN}\label{count-predicted-outcome-tp-fp-tn-and-fn}

We have already got the prediction by our classifier \(\hat{f}(x)\), now
let's compare it with the label of testing data set to check how it
works.

The outcome are

\begin{itemize}
\tightlist
\item
  \emph{True positive}: \(y = +1\) and \(\hat{y} = +1\).
\item
  \emph{True negative}: \(y = -1\) and \(\hat{y} = -1\).
\item
  \emph{False positive}: \(y = -1\) and \(\hat{y} = +1\).
\item
  \emph{False negative}: \(y = +1\) and \(\hat{y} = -1\).
\end{itemize}

    \begin{Verbatim}[commandchars=\\\{\}]
{\color{incolor}In [{\color{incolor}9}]:} \PY{k}{def} \PY{n+nf}{outcomes}\PY{p}{(}\PY{n}{labels}\PY{p}{,} \PY{n}{prediction}\PY{p}{)}\PY{p}{:}
            \PY{l+s+sd}{\PYZdq{}\PYZdq{}\PYZdq{}count outcomes of prediction}
        \PY{l+s+sd}{    }
        \PY{l+s+sd}{    Input:}
        \PY{l+s+sd}{        label(array): correct labels}
        \PY{l+s+sd}{        prediction(array): prediction of classifier}
        \PY{l+s+sd}{    Return:}
        \PY{l+s+sd}{        A dictionary contains the indices of each outcome type}
        \PY{l+s+sd}{        tp: true positive}
        \PY{l+s+sd}{        tn: true negative}
        \PY{l+s+sd}{        fp: false positive}
        \PY{l+s+sd}{        fn: flase negative}
        \PY{l+s+sd}{    \PYZdq{}\PYZdq{}\PYZdq{}}
            
            \PY{c+c1}{\PYZsh{} process labels, let 0 == +1, others == \PYZhy{}1}
            \PY{n}{label\PYZus{}processed} \PY{o}{=} \PY{n}{process\PYZus{}label}\PY{p}{(}\PY{n}{labels}\PY{p}{)}
            
            \PY{n}{length} \PY{o}{=} \PY{n+nb}{len}\PY{p}{(}\PY{n}{label\PYZus{}processed}\PY{p}{)}
            \PY{n}{tp} \PY{o}{=} \PY{p}{[}\PY{p}{]}  
            \PY{n}{fp} \PY{o}{=} \PY{p}{[}\PY{p}{]}
            \PY{n}{tn} \PY{o}{=} \PY{p}{[}\PY{p}{]}
            \PY{n}{fn} \PY{o}{=} \PY{p}{[}\PY{p}{]}
            
            \PY{k}{for} \PY{n}{i} \PY{o+ow}{in} \PY{n+nb}{range}\PY{p}{(}\PY{n}{length}\PY{p}{)}\PY{p}{:}
                \PY{k}{if} \PY{n}{label\PYZus{}processed}\PY{p}{[}\PY{n}{i}\PY{p}{]} \PY{o}{==} \PY{l+m+mi}{1} \PY{o+ow}{and} \PY{n}{prediction}\PY{p}{[}\PY{n}{i}\PY{p}{]} \PY{o}{==} \PY{l+m+mi}{1}\PY{p}{:}
                    \PY{n}{tp}\PY{o}{.}\PY{n}{append}\PY{p}{(}\PY{n}{i}\PY{p}{)}
                \PY{k}{elif} \PY{n}{label\PYZus{}processed}\PY{p}{[}\PY{n}{i}\PY{p}{]} \PY{o}{==} \PY{l+m+mi}{1} \PY{o+ow}{and} \PY{n}{prediction}\PY{p}{[}\PY{n}{i}\PY{p}{]} \PY{o}{==} \PY{o}{\PYZhy{}}\PY{l+m+mi}{1}\PY{p}{:}
                    \PY{n}{fn}\PY{o}{.}\PY{n}{append}\PY{p}{(}\PY{n}{i}\PY{p}{)}
                \PY{k}{elif} \PY{n}{label\PYZus{}processed}\PY{p}{[}\PY{n}{i}\PY{p}{]} \PY{o}{==} \PY{o}{\PYZhy{}}\PY{l+m+mi}{1} \PY{o+ow}{and} \PY{n}{prediction}\PY{p}{[}\PY{n}{i}\PY{p}{]} \PY{o}{==} \PY{o}{\PYZhy{}}\PY{l+m+mi}{1}\PY{p}{:}
                    \PY{n}{tn}\PY{o}{.}\PY{n}{append}\PY{p}{(}\PY{n}{i}\PY{p}{)}
                \PY{k}{else}\PY{p}{:}
                    \PY{n}{fp}\PY{o}{.}\PY{n}{append}\PY{p}{(}\PY{n}{i}\PY{p}{)}
                    
            \PY{n}{outcome} \PY{o}{=} \PY{p}{\PYZob{}}\PY{l+s+s1}{\PYZsq{}}\PY{l+s+s1}{TP}\PY{l+s+s1}{\PYZsq{}}\PY{p}{:} \PY{n}{tp}\PY{p}{,}
                       \PY{l+s+s1}{\PYZsq{}}\PY{l+s+s1}{FP}\PY{l+s+s1}{\PYZsq{}}\PY{p}{:} \PY{n}{fp}\PY{p}{,}
                       \PY{l+s+s1}{\PYZsq{}}\PY{l+s+s1}{TN}\PY{l+s+s1}{\PYZsq{}}\PY{p}{:} \PY{n}{tn}\PY{p}{,}
                       \PY{l+s+s1}{\PYZsq{}}\PY{l+s+s1}{FN}\PY{l+s+s1}{\PYZsq{}}\PY{p}{:} \PY{n}{fn}\PY{p}{\PYZcb{}}
            
            \PY{k}{return} \PY{n}{outcome}
\end{Verbatim}


    \subsubsection{\texorpdfstring{Define \(F_{1}\) score
funtion}{Define F\_\{1\} score funtion}}\label{define-f_1-score-funtion}

\[F_{1} score = 2 \cdot \frac{precision \cdot recall}{precision + recall},\]

where
\(precision = \frac{true \ positives}{true \ positives \ + \ false \ positives}\),
\(recall = \frac{true \ positives}{false \ negative \ + \ true \ positive}\)

    \begin{Verbatim}[commandchars=\\\{\}]
{\color{incolor}In [{\color{incolor}10}]:} \PY{k}{def} \PY{n+nf}{f1\PYZus{}score}\PY{p}{(}\PY{n}{outcome}\PY{p}{)}\PY{p}{:}
             
             \PY{n}{tp} \PY{o}{=} \PY{n+nb}{len}\PY{p}{(}\PY{n}{outcome}\PY{p}{[}\PY{l+s+s1}{\PYZsq{}}\PY{l+s+s1}{TP}\PY{l+s+s1}{\PYZsq{}}\PY{p}{]}\PY{p}{)}
             \PY{n}{fp} \PY{o}{=} \PY{n+nb}{len}\PY{p}{(}\PY{n}{outcome}\PY{p}{[}\PY{l+s+s1}{\PYZsq{}}\PY{l+s+s1}{FP}\PY{l+s+s1}{\PYZsq{}}\PY{p}{]}\PY{p}{)}
             \PY{n}{tn} \PY{o}{=} \PY{n+nb}{len}\PY{p}{(}\PY{n}{outcome}\PY{p}{[}\PY{l+s+s1}{\PYZsq{}}\PY{l+s+s1}{TN}\PY{l+s+s1}{\PYZsq{}}\PY{p}{]}\PY{p}{)}
             \PY{n}{fn} \PY{o}{=} \PY{n+nb}{len}\PY{p}{(}\PY{n}{outcome}\PY{p}{[}\PY{l+s+s1}{\PYZsq{}}\PY{l+s+s1}{FN}\PY{l+s+s1}{\PYZsq{}}\PY{p}{]}\PY{p}{)}
             
             \PY{n}{precision} \PY{o}{=} \PY{n}{tp} \PY{o}{/} \PY{p}{(}\PY{n}{tp} \PY{o}{+} \PY{n}{fp}\PY{p}{)}
             \PY{n}{recall} \PY{o}{=} \PY{n}{tp} \PY{o}{/} \PY{p}{(}\PY{n}{fn} \PY{o}{+} \PY{n}{tp}\PY{p}{)}
             
             \PY{k}{return} \PY{l+m+mi}{2} \PY{o}{*} \PY{n}{precision} \PY{o}{*} \PY{n}{recall} \PY{o}{/} \PY{p}{(}\PY{n}{precision} \PY{o}{+} \PY{n}{recall}\PY{p}{)}
\end{Verbatim}


    \subsubsection{Plotting funcions}\label{plotting-funcions}

    \begin{Verbatim}[commandchars=\\\{\}]
{\color{incolor}In [{\color{incolor}11}]:} \PY{k}{def} \PY{n+nf}{average\PYZus{}img}\PY{p}{(}\PY{n}{data}\PY{p}{,} \PY{n}{indices}\PY{p}{)}\PY{p}{:}
             \PY{c+c1}{\PYZsh{} compute the average value of one outcome type}
             
             \PY{n}{size} \PY{o}{=} \PY{l+m+mi}{28} \PY{o}{*} \PY{l+m+mi}{28}
             \PY{n}{sum\PYZus{}img} \PY{o}{=} \PY{n}{np}\PY{o}{.}\PY{n}{zeros}\PY{p}{(}\PY{n}{size}\PY{p}{)}
             
             \PY{k}{for} \PY{n}{index} \PY{o+ow}{in} \PY{n}{indices}\PY{p}{:}
                 \PY{n}{img} \PY{o}{=} \PY{n}{data}\PY{p}{[}\PY{p}{:}\PY{p}{,} \PY{n}{index}\PY{p}{]}
                 \PY{n}{sum\PYZus{}img} \PY{o}{+}\PY{o}{=} \PY{n}{img}
             
             \PY{n}{num\PYZus{}img} \PY{o}{=} \PY{n+nb}{len}\PY{p}{(}\PY{n}{indices}\PY{p}{)}
             
             \PY{k}{return} \PY{n}{sum\PYZus{}img} \PY{o}{/} \PY{n}{num\PYZus{}img}
             
         
         \PY{k}{def} \PY{n+nf}{plot\PYZus{}all}\PY{p}{(}\PY{n}{data}\PY{p}{,} \PY{n}{outcomes}\PY{p}{)}\PY{p}{:}
             
             \PY{n}{labels} \PY{o}{=} \PY{p}{[}\PY{l+s+s1}{\PYZsq{}}\PY{l+s+s1}{TP}\PY{l+s+s1}{\PYZsq{}}\PY{p}{,} \PY{l+s+s1}{\PYZsq{}}\PY{l+s+s1}{FP}\PY{l+s+s1}{\PYZsq{}}\PY{p}{,} \PY{l+s+s1}{\PYZsq{}}\PY{l+s+s1}{TN}\PY{l+s+s1}{\PYZsq{}}\PY{p}{,} \PY{l+s+s1}{\PYZsq{}}\PY{l+s+s1}{FN}\PY{l+s+s1}{\PYZsq{}}\PY{p}{]}
             
             \PY{k}{for} \PY{n}{i} \PY{o+ow}{in} \PY{n+nb}{range}\PY{p}{(}\PY{l+m+mi}{4}\PY{p}{)}\PY{p}{:}
         
                 \PY{n}{label} \PY{o}{=} \PY{n}{labels}\PY{p}{[}\PY{n}{i}\PY{p}{]}
                 \PY{n}{imgs} \PY{o}{=} \PY{n}{average\PYZus{}img}\PY{p}{(}\PY{n}{data}\PY{p}{,} \PY{n}{outcomes}\PY{p}{[}\PY{n}{label}\PY{p}{]}\PY{p}{)}
                 \PY{n}{img\PYZus{}matrix} \PY{o}{=} \PY{n}{imgs}\PY{o}{.}\PY{n}{reshape}\PY{p}{(}\PY{p}{(}\PY{l+m+mi}{28}\PY{p}{,} \PY{l+m+mi}{28}\PY{p}{)}\PY{p}{)}
         
                 \PY{n}{plt}\PY{o}{.}\PY{n}{subplot}\PY{p}{(}\PY{l+m+mi}{2}\PY{p}{,} \PY{l+m+mi}{2}\PY{p}{,} \PY{n}{i}\PY{o}{+}\PY{l+m+mi}{1}\PY{p}{)}
                 \PY{n}{plt}\PY{o}{.}\PY{n}{title}\PY{p}{(}\PY{l+s+s1}{\PYZsq{}}\PY{l+s+s1}{Average Image of }\PY{l+s+s1}{\PYZsq{}} \PY{o}{+} \PY{n}{label}\PY{p}{)}
                 \PY{n}{plt}\PY{o}{.}\PY{n}{imshow}\PY{p}{(}\PY{n}{img\PYZus{}matrix}\PY{p}{,} \PY{n}{cmap}\PY{o}{=}\PY{l+s+s1}{\PYZsq{}}\PY{l+s+s1}{Greys}\PY{l+s+s1}{\PYZsq{}}\PY{p}{,} \PY{n}{interpolation}\PY{o}{=}\PY{l+s+s1}{\PYZsq{}}\PY{l+s+s1}{None}\PY{l+s+s1}{\PYZsq{}}\PY{p}{)}
         
                 \PY{n}{frame}   \PY{o}{=} \PY{n}{plt}\PY{o}{.}\PY{n}{gca}\PY{p}{(}\PY{p}{)}
                 \PY{n}{frame}\PY{o}{.}\PY{n}{axes}\PY{o}{.}\PY{n}{get\PYZus{}xaxis}\PY{p}{(}\PY{p}{)}\PY{o}{.}\PY{n}{set\PYZus{}visible}\PY{p}{(}\PY{k+kc}{False}\PY{p}{)}
                 \PY{n}{frame}\PY{o}{.}\PY{n}{axes}\PY{o}{.}\PY{n}{get\PYZus{}yaxis}\PY{p}{(}\PY{p}{)}\PY{o}{.}\PY{n}{set\PYZus{}visible}\PY{p}{(}\PY{k+kc}{False}\PY{p}{)}
          
             \PY{n}{plt}\PY{o}{.}\PY{n}{show}\PY{p}{(}\PY{p}{)}
\end{Verbatim}


    \begin{center}\rule{0.5\linewidth}{\linethickness}\end{center}

\subsection{Try different number of
parameters}\label{try-different-number-of-parameters}

Now let's try different number parameters.

Here we use logarithm \(2^{n}, n \in [1, 10]\) number of parameters.

    \begin{Verbatim}[commandchars=\\\{\}]
{\color{incolor}In [{\color{incolor}17}]:} \PY{n}{num\PYZus{}paras} \PY{o}{=} \PY{p}{[}\PY{l+m+mi}{2}\PY{o}{*}\PY{o}{*}\PY{n}{n} \PY{k}{for} \PY{n}{n} \PY{o+ow}{in} \PY{n+nb}{range}\PY{p}{(}\PY{l+m+mi}{1}\PY{p}{,} \PY{l+m+mi}{11}\PY{p}{)}\PY{p}{]}
         
         \PY{n}{f1\PYZus{}history} \PY{o}{=} \PY{p}{[}\PY{p}{]}
         
         \PY{k}{for} \PY{n}{p} \PY{o+ow}{in} \PY{n}{num\PYZus{}paras}\PY{p}{:}
             
             \PY{n}{features}\PY{p}{,} \PY{n}{theta} \PY{o}{=} \PY{n}{eles\PYZus{}classifier}\PY{p}{(}\PY{n}{p}\PY{p}{,} \PY{l+m+mi}{28}\PY{o}{*}\PY{l+m+mi}{28}\PY{p}{,} \PY{n}{list\PYZus{}image\PYZus{}train}\PY{p}{,} \PY{n}{num\PYZus{}train}\PY{p}{,} \PY{n}{list\PYZus{}label\PYZus{}train}\PY{p}{)}
             
             \PY{n}{prediction} \PY{o}{=} \PY{n}{classifier}\PY{p}{(}\PY{n}{list\PYZus{}image\PYZus{}test}\PY{p}{,} \PY{n}{num\PYZus{}test}\PY{p}{,} \PY{n}{features}\PY{p}{,} \PY{n}{theta}\PY{p}{)}
             \PY{n}{outcome} \PY{o}{=} \PY{n}{outcomes}\PY{p}{(}\PY{n}{list\PYZus{}label\PYZus{}test}\PY{p}{,} \PY{n}{prediction}\PY{p}{)}
         
             \PY{n}{f1\PYZus{}history}\PY{o}{.}\PY{n}{append}\PY{p}{(}\PY{n}{f1\PYZus{}score}\PY{p}{(}\PY{n}{outcome}\PY{p}{)}\PY{p}{)}
             
             \PY{n+nb}{print}\PY{p}{(}\PY{l+s+s2}{\PYZdq{}}\PY{l+s+se}{\PYZbs{}n}\PY{l+s+s2}{ }\PY{l+s+si}{\PYZob{}\PYZcb{}}\PY{l+s+s2}{ parameters}\PY{l+s+s2}{\PYZsq{}}\PY{l+s+s2}{ images}\PY{l+s+s2}{\PYZdq{}}\PY{o}{.}\PY{n}{format}\PY{p}{(}\PY{n}{p}\PY{p}{)}\PY{p}{)}
             \PY{n}{plot\PYZus{}all}\PY{p}{(}\PY{n}{list\PYZus{}image\PYZus{}test}\PY{p}{,} \PY{n}{outcome}\PY{p}{)}
\end{Verbatim}


    \begin{Verbatim}[commandchars=\\\{\}]

 2 parameters' images

    \end{Verbatim}

    \begin{center}
    \adjustimage{max size={0.9\linewidth}{0.9\paperheight}}{output_25_1.png}
    \end{center}
    { \hspace*{\fill} \\}
    
    \begin{Verbatim}[commandchars=\\\{\}]

 4 parameters' images

    \end{Verbatim}

    \begin{center}
    \adjustimage{max size={0.9\linewidth}{0.9\paperheight}}{output_25_3.png}
    \end{center}
    { \hspace*{\fill} \\}
    
    \begin{Verbatim}[commandchars=\\\{\}]

 8 parameters' images

    \end{Verbatim}

    \begin{center}
    \adjustimage{max size={0.9\linewidth}{0.9\paperheight}}{output_25_5.png}
    \end{center}
    { \hspace*{\fill} \\}
    
    \begin{Verbatim}[commandchars=\\\{\}]

 16 parameters' images

    \end{Verbatim}

    \begin{center}
    \adjustimage{max size={0.9\linewidth}{0.9\paperheight}}{output_25_7.png}
    \end{center}
    { \hspace*{\fill} \\}
    
    \begin{Verbatim}[commandchars=\\\{\}]

 32 parameters' images

    \end{Verbatim}

    \begin{center}
    \adjustimage{max size={0.9\linewidth}{0.9\paperheight}}{output_25_9.png}
    \end{center}
    { \hspace*{\fill} \\}
    
    \begin{Verbatim}[commandchars=\\\{\}]

 64 parameters' images

    \end{Verbatim}

    \begin{center}
    \adjustimage{max size={0.9\linewidth}{0.9\paperheight}}{output_25_11.png}
    \end{center}
    { \hspace*{\fill} \\}
    
    \begin{Verbatim}[commandchars=\\\{\}]

 128 parameters' images

    \end{Verbatim}

    \begin{center}
    \adjustimage{max size={0.9\linewidth}{0.9\paperheight}}{output_25_13.png}
    \end{center}
    { \hspace*{\fill} \\}
    
    \begin{Verbatim}[commandchars=\\\{\}]

 256 parameters' images

    \end{Verbatim}

    \begin{center}
    \adjustimage{max size={0.9\linewidth}{0.9\paperheight}}{output_25_15.png}
    \end{center}
    { \hspace*{\fill} \\}
    
    \begin{Verbatim}[commandchars=\\\{\}]

 512 parameters' images

    \end{Verbatim}

    \begin{center}
    \adjustimage{max size={0.9\linewidth}{0.9\paperheight}}{output_25_17.png}
    \end{center}
    { \hspace*{\fill} \\}
    
    \begin{Verbatim}[commandchars=\\\{\}]

 1024 parameters' images

    \end{Verbatim}

    \begin{center}
    \adjustimage{max size={0.9\linewidth}{0.9\paperheight}}{output_25_19.png}
    \end{center}
    { \hspace*{\fill} \\}
    
    \begin{Verbatim}[commandchars=\\\{\}]
{\color{incolor}In [{\color{incolor}20}]:} \PY{n}{plt}\PY{o}{.}\PY{n}{title}\PY{p}{(}\PY{l+s+s2}{\PYZdq{}}\PY{l+s+s2}{F1 Score History}\PY{l+s+s2}{\PYZdq{}}\PY{p}{)}
         \PY{n}{plt}\PY{o}{.}\PY{n}{plot}\PY{p}{(}\PY{n}{num\PYZus{}paras}\PY{p}{,} \PY{n}{f1\PYZus{}history}\PY{p}{,} \PY{l+s+s1}{\PYZsq{}}\PY{l+s+s1}{b\PYZhy{}}\PY{l+s+s1}{\PYZsq{}}\PY{p}{)}
         \PY{n}{plt}\PY{o}{.}\PY{n}{xlabel}\PY{p}{(}\PY{l+s+s1}{\PYZsq{}}\PY{l+s+s1}{number of parameters}\PY{l+s+s1}{\PYZsq{}}\PY{p}{)}
         \PY{n}{plt}\PY{o}{.}\PY{n}{ylabel}\PY{p}{(}\PY{l+s+s1}{\PYZsq{}}\PY{l+s+s1}{F1 score}\PY{l+s+s1}{\PYZsq{}}\PY{p}{)}
         \PY{n}{plt}\PY{o}{.}\PY{n}{show}\PY{p}{(}\PY{p}{)}
\end{Verbatim}


    \begin{center}
    \adjustimage{max size={0.9\linewidth}{0.9\paperheight}}{output_26_0.png}
    \end{center}
    { \hspace*{\fill} \\}
    

    % Add a bibliography block to the postdoc
    
    
    
    \end{document}
